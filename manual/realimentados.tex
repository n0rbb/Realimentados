\documentclass[a4paper,10pt]{book}
\usepackage[pdftex]{graphicx}
\usepackage{epstopdf}
\usepackage{subfigure}
\usepackage{amsmath,amsthm}
\usepackage{tikz}
\usepackage{circuitikz}
\usetikzlibrary{babel}
\usetikzlibrary{shapes, arrows, patterns, angles, quotes}
\textwidth= 15cm
\evensidemargin=0cm
\usepackage[spanish]{babel}
\usepackage[utf8]{inputenc}
\usepackage{textcomp}
\usepackage{amstext}
\usepackage{amsfonts}
\usepackage{amssymb}
\usepackage{comment}
\usepackage[hyperindex=true,breaklinks=true,colorlinks=true,linkcolor=blue]{hyperref}
\renewcommand{\tablename}{Tabla}
\renewcommand{\listtablename}{\'Indice de Tablas}

\usepackage{listings}
\usepackage{color} %red, green, blue, yellow, cyan, magenta, black, white
\definecolor{mygreen}{RGB}{28,172,0} % color values Red, Green, Blue
\definecolor{mylilas}{RGB}{170,55,241}

\usepackage{multirow}
\usepackage{makeidx}

%\usepackage{draftwatermark}
%\SetWatermarkText{Borrador,juan.jimenez@fis.ucm.es}
%\SetWatermarkScale{2}

% Atajos para el tikz
\tikzstyle{block} = [draw, rectangle, minimum width=6em]
\tikzstyle{sum} = [draw, fill=blue!20, circle, node distance=1cm]
\tikzstyle{input} = [coordinate]
\tikzstyle{output} = [coordinate]
\tikzstyle{pinstyle} = [pin edge={to-,thin,black}]

% Entornos para los "teoremas"
\newtheorem{algo}{Algoritmo}[section]
\newtheorem{theorem}{Teorema}[section]
\newtheorem{problem}{Problema}[section]
\newtheorem{corollary}{Corolario}[section]
\newtheorem{lemmas}{Lema}[section]
\newcommand*{\lema}{Lema}
\newenvironment{lemma}[1][\lema]{\begin{lemmas}[#1]\renewcommand*{\qedsymbol}{\(\Diamond\)}}{\end{lemmas}}

\theoremstyle{definition}
%\newtheorem{definition}{Definición}[section]
\newtheorem{definitions}{Definición}[section]
\newcommand*{\definicion}{Definición}
\newenvironment{definition}[1][\definicion]{\begin{definitions}[#1]\renewcommand*{\qedsymbol}{\(\bigtriangledown\)}}{\end{definitions}}

\newtheorem{examples}{Ejemplo}[section]
\newcommand*{\ejemplo}{Ejemplo}
\newenvironment{example}[1][\ejemplo]{\begin{examples}[#1]\renewcommand*{\qedsymbol}{\(\maltese\)}}{\end{examples}}
%\renewcommand{\qedsymbol}{\maltese}

\theoremstyle{remark}
\newtheorem{remark}{Atención}[section]

\newcommand{\argmin}{\operatornamewithlimits{arg\,min}}

\newcommand{\gusanito}{
	\xi
}

\graphicspath{{./figuras/}}
\makeindex
\begin{document}
\title{
\begin{flushleft}
\includegraphics[width=2.5cm]{ucm2.eps}
Universidad Complutense de Madrid\\
---------------------------------------------------------------------\
\end{flushleft}
Sistemas din\'amicos y realimentaci\'on}
\author{ Juan Jim\'enez \\ H\'ector Garc\'ia de Marina}

\maketitle\
\
\vspace*{\fill}

\includegraphics[scale=1]{by-sa.eps}\\
El contenido de estos apuntes est\'a bajo licencia Creative Commons Atribution-ShareAlike 4.0\\
\href{http://creativecommons.org/licenses/by-sa/4.0/}{http://creativecommons.org/licenses/by-sa/4.0/}\\
\copyright Juan Jim\'enez

\bigskip
\tableofcontents
\listoffigures
%\listoftables
%\section*{Matlab Code}
\lstset{language=Matlab,%
    %basicstyle=\color{red},
    breaklines=true,%
    morekeywords={matlab2tikz},
    keywordstyle=\color{blue},%
    morekeywords=[2]{1}, keywordstyle=[2]{\color{black}},
    identifierstyle=\color{black},%
    stringstyle=\color{mylilas},
    commentstyle=\color{mygreen},%
    showstringspaces=false,%without this there will be a symbol in the places where there is a space
    numbers=left,%
    numberstyle={\tiny \color{black}},% size of the numbers
    numbersep=9pt, % this defines how far the numbers are from the text
    emph=[1]{for,end,break},emphstyle=[1]\color{red}, %some words to emphasise
    %emph=[2]{word1,word2}, emphstyle=[2]{style},    
}
\chapter{Introducción}
Estas notas de clase conforman el contenido de la asignatura ``Sistemas Dinámicos y Realimentación". La asignatura está especialmente orientada  a los estudiantes del grado en Física y contiene tan solo dos ideas básicas:

\begin{enumerate}
\item Una parte apreciable --si no toda-- la realidad física que conocemos puede modelarse mediante el uso de lo que denominaremos sistemas dinámicos. Estos vendrán caracterizados por un conjunto de variables de estado, que describen aspectos mensurables de la realidad y son dinámicos en el sentido de que el valor de las variables de estado evoluciona en el tiempo. Una manera habitual de describir un sistema dinámico, a partir de sus variables de estado, es mediante el uso de ecuaciones diferenciales\footnote{Para sistemas que evolucionan en tiempo continuo $t\in \mathbb{R}^+$. Para sistemas de tiempo discreto $t \in \mathbb{N}$ se emplean ecuaciones en diferencias. Por falta de tiempo, no estudiaremos sistemas de tiempo discreto en esta asignatura, aunque tienen un gran interés tanto en sí mismos como por su relación con los sistemas de tiempo continuo.}:

\begin{equation}
	\Sigma := \begin{cases}
		\dot x(t) =& f(t,x(t),u(t)) \\ y(t) =& g(t,x(t),u(t))
	\end{cases}, 
\end{equation}
en donde $\dot x := \frac{\mathrm{d}}{\mathrm{dt}}(x(t))$ es la notación para la derivada total con respecto del tiempo, y $f:\mathbb{R} \times \mathbb{R}^n \times \mathbb{R}^k \to \mathbb{R}^n$ y $g: \mathbb{R} \times \mathbb{R}^n \times \mathbb{R}^k \to \mathbb{R}^m$ son funciones. Las variables de estado se agrupan en el vector $x(t)$.  El vector $u(t)$ agrupa las variables de entrada que representan magnitudes físicas externas al sistema cuyos valores influyen en la evolución de las variables de estado. Por último el vector $y(t)$ representa las salidas del sistema: magnitudes físicas, dependientes de las variables de estado y de las entradas, que pueden ser observadas (medidas). 

\item En muchos sistemas es posible manipular los valores de las variables de entrada de modo que se pueda obtener una evolución deseada de sus variables de estado y, por tanto, de su salida. El estudio de los modos de manipular las variables de entrada para conseguir la evolución deseada del sistema constituye lo que se conoce como \emph{Control de Sistemas}.

Una forma particularmente adecuada y poderosa de manipular las variables de entrada para controlar la evolución de un sistema es hacerlas depender de los valores que toman las variables de estado mismas: $u(t) = u(x(t))$ o de las salidas del sistema $u(t) = u(y(t))$. Esta manera de controlar un sistema se conoce como \emph{control por realimentación}.
\end{enumerate}

Podríamos decir que el estudio de los sistemas dinámicos tiene un carácter transversal ya que es aplicable a fenómenos físicos de cualquier naturaleza: mecánicos, eléctricos, ópticos, etc. Además, aplicando el conocido principio ``ecuaciones iguales tienen soluciones iguales'', podemos  con facilidad trasladar los resultados obtenidos en sistemas de una determinada naturaleza a otros distintos que puedan ser modelados con ecuaciones equivalentes. Así, por ejemplo, es posible establecer analogías entre sistemas mecánicos y sistemas eléctricos.\\

Un aspecto particularmente interesante en el estudio de los sistemas dinámicos es el análisis de su estabilidad. En términos puramente intuitivos podríamos dar una definición preliminar de estabilidad diciendo que un sistema es estable cuando la evolución en el tiempo de sus variables de estado no diverge. A lo largo de la asignatura precisaremos este concepto, estudiaremos formas de establecer la estabilidad o no de un sistema y veremos su relación con la posibilidad de controlar un sistema mediante realimentación.\\

El concepto de realimentación está presente en la naturaleza. Los sistemas biológicos son capaces de reaccionar a su estado y modificarlo en beneficio propio.  Ejemplos de ello son la regulación del contenido de azúcar en sangre, de la temperatura corporal, o del equilibrio. En todos los casos, se parte del conocimiento del estado actual para actuar modificando el estado de modo que se alcance la situación deseada; a continuación, se vuelve a observar el estado resultante y se vuelve a actuar, etc. Este ciclo observación-actuación-observación es un principio muy poderoso y está en la base del desarrollo científico y tecnológico. Los sistemas de control, basados en repetir dicho ciclo, son ubicuos. Están presentes en la industria, en los sistemas de comunicaciones, en el transporte y, por supuesto, en la investigación científica. Por ejemplo, el interferómetro de ondas gravitacionales (LIGO) funciona gracias a un sofisticado sistema de control para estabilizarlo; el (gran) colisionador de hadrones (LHC) del CERN solo es una realidad gracias a los sistemas de control que permiten controlar los haces, mantener a baja temperatura sus electroimanes, etc.\\

La asignatura y el contenido de estos apuntes están divididos en dos grandes bloques. Los temas \ref{lineales}, \ref{crllineales} y \ref{ejercicios}, se centran en el estudio de los sistemas lineales y su control por realimentación.   Los capítulos \ref{lyapunov} y \ref{crlnolineal} son una introducción al estudio de los sistemas dinámicos no-lineales, su estabilidad y el diseño de algunos controladores básicos para este tipo de sistemas. Se ha incluido también un apéndice con algunos ejemplos de uso de Matlab para resolver ejercicios de sistemas dinámicos. El tema \ref{modelado} contiene las definiciones y conceptos necesarios para el desarrollo del resto de los apuntes: introduce el concepto de representación de sistemas en variables de estado, así como algunos de los métodos de análisis de dichos sistemas. \\

Por último, los apuntes son para vosotros (los que estudiáis la asignatura). Están escritos en \LaTeX\ y todos los archivos fuente están colgados en el repositorio de GitHub \url{https://github.com/UCM-237/Realimentados}. Así que, si os animáis, podéis \emph{clonaros} el \emph{repo} y añadir cambiar y corregir todo lo que queráis a vuestros apuntes. Además, podéis enviarnos un \emph{pull request} cuando corrijáis errores, y así vamos mejorando los apuntes entre todos. Si no os va mucho git y/o \LaTeX, también podéis enviar sugerencias y correcciones directamente al Campus Virtual o por correo electrónico.



\include{introduccion}
\chapter{Modelado de sistemas dinámicos}\label{modelado}

\section{Sistemas en el espacio de estados}
Nos vamos a centrar en sistemas que puedan ser descritos por características cuantificables. A estas características las vamos a llamar {\bf variables de estado}, como por ejemplo, una temperatura, una velocidad, o un voltaje. Si estas variables de estado dependen del tiempo, llamamos {\bf señal} a la sucesión de valores de las variables de estado en el tiempo. Uno podría interaccionar con el sistema a través de una {\bf entrada} cuantificable, y a su vez medir información del sistema a través de una {\bf salida} cuantificable.\\

Vamos a definir $x(t)\in\mathbb{R}^n$, $y(t)\in\mathbb{R}^m$ y $u(t)\in\mathbb{R}^k$, respectivamente, como el vector apilado de variables estados, la salida y la entrada a un sistema $\Sigma$. En particular, variables de estado, entradas y salidas son señales, e.g. $x :[0,\infty) \to \mathbb{R}^n$.\\

El sistema $\Sigma$ es un modelo que predice el valor de las variables de estado y la salida a lo largo del tiempo. Esta predicción incorpora la interacción de la entrada con los estados y la salida. En particular, vamos a emplear ecuaciones diferenciales como herramienta para predecir la evolución en el tiempo de los estados del sistema $\Sigma$ como se muestra a continuación
\begin{equation}
	\Sigma := \begin{cases}
		\dot x(t) =& f(t,x(t),u(t)) \\ y(t) =& g(t,x(t),u(t))
	\end{cases}, 
\label{eq: sigma}
\end{equation}
en donde $\dot x := \frac{\mathrm{d}}{\mathrm{dt}}(x(t))$ es la notación para la derivada total con respecto del tiempo, y $f:\mathbb{R} \times \mathbb{R}^n \times \mathbb{R}^k \to \mathbb{R}^n$ y $g: \mathbb{R} \times \mathbb{R}^n \times \mathbb{R}^k \to \mathbb{R}^m$ son funciones.\\

Podemos representar el sistema $\Sigma$ como un bloque con puertos de entrada y de salida como se muestra en la figura \ref{fig: sigma}.\\

\begin{figure}[!ht]
\centering
\begin{tikzpicture}[auto, node distance=2cm,>=latex']
	\node [input, name=input] {};
	\node [block, right of=input] (system) {$\Sigma$};
	\node [output, right of=system] (output) {};
	\draw [draw,->] (input) -- node {$u(t)$} (system);
	\draw [->] (system) -- node [name=y] {$y(t)$}(output);
\end{tikzpicture}
	\caption{Diagrama de bloque entrada/salida del sistema $\Sigma$.}
	\label{fig: sigma}
\end{figure}

Dado que el sistema viene descrito por las variables de estado,  con frecuencia es posible prescindir de la salida y centrar el estudio en la primera de las dos ecuaciones anteriores.\\

 En ocasiones podemos emplear ecuaciones de estado en las que no aparece explícitamente la entrada  para describir un sistema,
\begin{equation}
\dot x = f(t,x).
\end{equation}\label{eq: for}\\

En este caso se habla de una ecuación de estado no forzada. Esto no quiere decir necesariamente que la entrada al sistema sea cero. Puede aparecer representada directamente como una función del tiempo, $u=\rho(t)$,  como una función (\emph{realimentación}) de los estados $u = \rho(x)$ o como una función de ambos $u=\rho(t,x)$.\\

Un caso especialmente interesante de (\ref{eq: for}) se obtiene cuando $f$ no depende explícitamente del tiempo,
\begin{equation}
\dot x = f(x).
\end{equation}
Se trata entonces de un sistema \emph{autónomo} o \emph{invariante en el tiempo}. Es decir, el comportamiento del sistema es invariante bajo traslaciones del origen de tiempos; si cambiamos la variable tiempo de $t$ a $t-a$, la parte derecha de la ecuación de estado no experimenta cambios.\\

Para un sistema dinámico descrito mediante variable de estados, un punto de equilibrio $\overline x$  tiene la propiedad de que si el sistema se encuentra en $\overline x$ en un instante de tiempo $t_0$, permanecerá en ese mismo punto para todo instante de tiempo posterior $t>t_0$. En el caso de un sistema autónomo, los puntos de equilibrio son las raíces reales de la ecuación
\begin{equation}
f(x)=0.
\end{equation} \\

Los puntos de equilibrio juegan un papel muy importante en el estudio de los sistemas dinámicos, como tendremos ocasión de ver más adelante.\\

Un caso particular de sistemas dinámicos los constituyen los sistemas lineales. Para ellos la ecuación \ref{eq: sigma} toma la forma
\begin{equation}
	\Sigma := \begin{cases}
		\dot x(t) =& A(t)x(t)+B(t)u(t) \\ y(t) =& C(t)x(t)+D(t)u(t)
	\end{cases}, 
\label{eq: sigmaL}
\end{equation}

En el capítulo 5 se estudiarán en detalle los sistemas lineales, cuyas propiedades permiten emplear herramientas de análisis precisas y potentes para su estudio y caracterización. En el caso del resto de los sistemas, es decir todos aquellos que son no-lineales, el análisis resulta más complejo.\\

Una primera aproximación al estudio de los sistemas no-lineales es linealizarlos en torno a un punto de trabajo y estudiar el sistema lineal resultante. Esta aproximación es muy valiosa, como veremos en la sección (\ref{sec: linear}). Sin embargo, solo es válida en las proximidades del punto de trabajo. Por tanto no es posible predecir a partir de ella el comportamiento del sistema cuando nos alejamos de dicho punto. Por otro lado, los sistemas no-lineales presentan una dinámica mucho mas rica, con fenómenos que no se dan en los sistemas lineales y que no pueden describirse con un modelo linealizado del sistema no-lineal. Algunos de éstos fenómenos son:

\paragraph{Múltiples puntos de equilibrio aislados.} Para un sistema lineal solo es posible encontrar un punto de equilibrio o una región continua de ellos (rectas, planos, etc).  Un sistema lineal es asintóticamente estable cuando, partiendo de valores iniciales arbitrarios, sus variables de estado se aproximan al punto de equilibrio con el tiempo. 

\begin{equation}
\lim_{ t \to \infty} x \to \overline x
\end{equation}
Un sistema no lineal puede presentar más de un punto de equilibrio y aproximarse a uno u otro dependiendo de las condiciones iniciales.

\paragraph{Tiempo de escape finito (\emph{Finite escape time}).} Un sistema lineal es inestable cuando, partiendo de valores arbitrarios, sus variables de estado divergen --se van a infinito-- con el tiempo.
\begin{equation}
\lim_{ t \to \infty} x \to \infty
\end{equation}
Es posible encontrar sistemas no-lineales inestables que se comportan de modo análogo a los lineales pero además, algunos sistemas no-lineales divergen --se van a infinito-- en un tiempo finito. 
\begin{equation}
\lim_{ t \to t_e} x \to \infty
\end{equation}

\paragraph{Ciclos límite.} Se dice que un sistema lineal es marginalmente estable cuando sus variables de estado oscilan periódicamente. Sin embargo, como veremos más adelante, las circunstancias para que se dé una oscilación mantenida en un sistema lineal constituyen una condición no robusta, por lo que en la práctica es imposible que se dé en condiciones reales. Además, la amplitud de las oscilaciones depende de las condiciones iniciales del sistema.\\

Hay sistemas no-lineales que pueden oscilar de modo estable con una amplitud y frecuencia independiente de las condiciones iniciales. Este tipo de oscilaciones son conocidas con el nombre de ciclos límite. 

\paragraph{Caos.} Algunos sistemas no-lineales exhiben comportamientos aún más complicados. No alcanzan un punto de equilibrio, no presentan oscilaciones periódicas y tampoco son inestables. A este comportamiento se le denomina \emph{caótico}. Algunos sistemas presentan incluso un comportamiento aleatorio, a pesar de la naturaleza determinista de las ecuaciones diferenciales que lo definen.\\


 Resulta por tanto necesario buscar herramientas específicas para el análisis de los sistemas no-lineales.
\section{Ejemplos de sistemas dinámicos}\label{sec: ejem}
A continuación se presentan algunos ejemplos clásicos de sistemas dinámicos.
\begin{example}[Péndulo invertido]

Vamos a derivar las funciones $f$ y $g$ en (\ref{eq: sigma}) para el sistema del péndulo invertido.\\

Primero, vamos a hallar la ecuación diferencial que describe la dinámica de una masa $m\in\mathbb{R}_+$ en el extremo de un péndulo de longitud $l\in\mathbb{R}_+$ tal y como se muestra en la figura \ref{fig: invpen}. Vamos a considerar que podemos interactuar con el sistema por medio de un torque $T\in\mathbb{R}$ en el otro extremo del péndulo, que la masa sufre un rozamiento proporcional $b\in\mathbb{R}_+$ a su celeridad y que podemos medir el ángulo $\theta\in\mathbb{R}$ que forma el péndulo con la vertical.

\begin{figure}[!ht]
\centering
\begin{tikzpicture}
    \coordinate (origo) at (0,0);
    \coordinate (pivot) at (1,5);

    % draw axes
	\fill[black] (origo) circle (0.05) ++ (-0.25,0.25) node (T) [black] {$T$} ++(0.75,0.75) node (l) [above] {$l$};
    \draw[thick,gray,->] (origo) -- ++(4,0) node[black,right] {$x$};
    \draw[thick,gray,->] (origo) -- ++(0,4) node (mary) [black,above] {$y$};

    % draw roof
    \fill[pattern = north east lines] ($ (origo) + (-1,0) $) rectangle ($ (origo) + (1,-0.5) $);
    \draw[thick] ($ (origo) + (-1,0) $) -- ($ (origo) + (1,0) $);

    \draw[thick] (origo) -- ++(-300:3) coordinate (bob) ++(0.5,0) node (hola) [black] {$m$};
    \fill (bob) circle (0.2);
    \draw[->] (bob) -- ++(0,-1) node (mg) [right] {$mg$};
    \draw[->] (bob) -- ++(-0.5,0.35) node (fr) [above] {$b\dot\theta$};

    \pic [draw, <-, "$\theta$", angle eccentricity=1.5] {angle = bob--origo--mary};
\end{tikzpicture}
\caption{Péndulo Invertido}
\label{fig: invpen}
\end{figure}

Definimos $g = 9.8$ e $I\in\mathbb{R}_+$ como la aceleración gravitatoria y el momento de inercia del péndulo respectivamente. Es sencillo comprobar que $I = ml^2$, y explotaremos que $I \ddot\theta = \text{suma de torques}$. De hecho, tenemos que considerar tres torques: 1. Torque $T$ ejercido por nosotros en la base del péndulo; 2. Torque $-bl\dot\theta$ ejercido por la fricción en la masa; 3. Torque $mgl \sin\theta$ ejercido por la atracción gravitatoria. Por lo que la ecuación diferencial que modela el comportamiento del péndulo invertido es
\begin{equation}
\ddot\theta = \frac{1}{ml^2}\left(mgl\sin{\theta}-bl\dot\theta + T\right).
\label{eq: dyn}
\end{equation}

Parece razonable escoger $\theta$ como uno de los estados para construir $x(t)$ en (\ref{eq: sigma}). De hecho, la ecuación (\ref{eq: dyn}) es de segundo orden en $\theta$, por lo que es conveniente escoger $\dot\theta$ como un estado también. Por lo tanto, definamos nuestro vector de estados como
\begin{equation}
x := \begin{bmatrix}\theta \\ \dot\theta \end{bmatrix},
\end{equation}
y dado que mediante el torque $T$ es como interactuamos con el sistema, escogemos como entrada $u(t) = T(t)$.\\

Ahora estamos listos para construir las funciones $f$ y $g$ en (\ref{eq: sigma}) para el péndulo invertido. Atención a que $f$ y $g$ solo toman como argumentos los vectores de estados $x$ y de entradas $u$. En el lado izquierdo de (\ref{eq: sigma}) tenemos la derivada temporal de $x(t)$, por lo que

\begin{equation}
	\frac{\mathrm{d}}{\mathrm{dt}}\left(\begin{bmatrix}\theta \\ \dot\theta \end{bmatrix}\right) = f(x(t), u(t)) = \begin{bmatrix}f_1(x(t), u(t)) \\ f_2(x(t), u(t))\end{bmatrix}, \label{eq: fn}
\end{equation}
donde automáticamente obtenemos que $f_1 = \dot\theta$. Nótese que la primera fila de $f$ en (\ref{eq: fn}), a la izquierda tenemos que $\frac{\mathrm{d}}{\mathrm{dt}}\theta$, y a la derecha tenemos que $f_1 = \dot\theta$ porque $\dot\theta$ es un estado o elemento de $x$. Desafortunadamente, no podemos decir que $f_2 = \ddot\theta(t)$ porque $ \ddot\theta$ no es un estado o elemento de $x$. No obstante, tenemos que la segunda fila $f_2$ viene dada por la ecuación (\ref{eq: dyn}), por lo que podemos escribir $f$ como
\begin{equation}
	\frac{\mathrm{d}}{\mathrm{dt}}\left(\begin{bmatrix}\theta \\ \dot\theta \end{bmatrix}\right) =  f(x(t), u(t)) = \begin{bmatrix} \dot\theta \\ \frac{1}{ml^2}\left(mgl\sin{\theta}-bl\dot\theta + T\right) \end{bmatrix}. \label{eq: f}
\end{equation}

El cálculo de $g$ es más sencillo en este caso. Hemos establecido al comienzo que solo podemos medir el ángulo $\theta$, de modo que $y(t) = \theta(t)$, i.e.,
\begin{equation}
g(x(t),u(t)) =  \theta(t).
	\label{eq: g}
\end{equation}

%A Python simulation of this dynamics can be found at \url{https://github.com/noether/aut_course}.
\qed 
\end{example}

\begin{example}[El oscilador de Van der Pol]
Se trata de un oscilador propuesto por primera vez por Balthasar Van der Pol, cuando trabajaba en Philips, para explicar las oscilaciones observadas en tubos de vacío. Podemos obtener la ecuación del oscilador empleando el circuito de la figura \ref{fig:vdp}.\\
\begin{figure}[h!]
\centering
\begin{circuitikz}[american, scale = 0.6]\draw
(0,-4)to[short]
(4,-4)to[short,,i^<= $i_C$]
(5,-4)to[short,i=$i_N$]
(6,-4)to[short](10,-4)
(0,-7.5)to[C = C](0,-4)
(5,-4) to[L = L, i>^= $i_L$,*-* ](5,-7.5)
(10,-4) to[generic=NL,  i= $i_N \equiv h(v)$](10,-7.5) 
(0,-7.5)to[short](10,-7.5)
;
\end{circuitikz}
\caption{Circuito eléctrico no lineal}
\label{fig:vdp}
\end{figure}

Donde el elemento no lineal NL presenta una relación entre voltaje e intensidad caracterizada por la función $h(v)$.\\

El voltaje $v$ en los tres componentes del circuito debe ser igual; además,
\begin{align}
v = L \frac{di_L}{dt}\\
i_C = C\frac{dv}{dt}\\
i_N = h(v)
\end{align}
Si aplicamos la primera ley de Kirchhoff al nodo superior del circuito,
\begin{align}
i_C+i_L+i_N = 0\\
C\frac{dv}{dt}+\frac{1}{L}\int_{-\infty}^{t}v(s)ds +h(v)=0\label{eq:vdp}
\end{align}
Si derivamos (\ref{eq:vdp}) con respecto al tiempo, dividimos por $C$ y reordenamos,
\begin{equation}\label{eq:vdp2}
\frac{d^2v}{dt^2}  + \frac{1}{C}\frac{dh(v)}{dv}\frac{dv}{dt} + \frac{1}{LC}\cdot v= 0
\end{equation}
Se trata de un caso particular de la ecuación de Liénard,
\begin{equation}
\ddot{v} +f(v)\dot{v}+g(v) = 0
\end{equation}
Si definimos ahora $h(v)$,
\begin{align}
h(v) = m\left(\frac{1}{3}v^3-v\right)\\
\frac{dh}{dv} = m(v^2-1)
\end{align}
y sustituimos en (\ref{eq:vdp2}),
\begin{equation}
\ddot{v} +m\frac{1}{C}(v^2-1)\dot{v}+\frac{1}{LC}v = 0
\end{equation}

Podemos representarla finalmente en variables de estado, tomando $x_1=v$ y $x_2=\dot{v}$,
\begin{align}
\dot{x}_1 &= x_2 \label{eq:vpst1} \\
\dot{x}_2 &= -\frac{1}{LC}x_1 - m\frac{1}{C}(x_1^2-1)x_2 \label{eq:vdpst2}
\end{align}
Podemos ahora hacer un primer análisis cualitativo. El primer término a la derecha del igual en la ecuación (\ref{eq:vdpst2}) representa una fuerza recuperadora proporcional al desplazamiento. El segundo término crecerá con la velocidad para $x_1 < 1$, alejando así al sistema del origen y representará un término disipativo para $x_1 > 1$ acercándolo por tanto de nuevo al origen. Es por tanto esperable, que se alcance algún tipo de situación de equilibrio. Más adelante definiremos esta situación rigurosamente como un ciclo límite. \qed
\end{example} 
\begin{example}[Un vehículo de cuatro ruedas.]
La figura \ref{fig:vehir} muestra un esquema de un vehículo terrestre de cuatro ruedas, visto desde arriba. Si consideramos que se mueve en el plano $x,y$, y que su velocidad instántanea $\vec{V}$ está siempre orientada en la dirección de avance del vehículo $\psi$ --asumimos que no derrapa ni se mueve lateralmente--, podemos entonces definir la velocidad, en el sistema de referencia $x,y$ como
\begin{align}
\dot{x} = V_x(t) = V\cos(\psi(t)),\\
\dot{y} = V_y(t) = V\sin(\psi(t)).
\end{align}

\begin{figure}[hbt!]
\centering

\begin{tikzpicture}
\draw(0,0)[rotate around={30:(0,0)},very thick,draw = blue!60!black!40]rectangle(2,1);
\draw(-0.3,-0.2)[rotate around={30:(0,0)},very thick,draw = blue!60!black!40]rectangle(0.3,0.0);
\draw(-0.3,1)[rotate around={30:(0,0)},very thick,draw = blue!60!black!40]rectangle(0.3,1.2);
\draw({2*cos(30)-0.3},{2*sin(30)-0.1})[rotate around={60:({2*cos(30)},{2*sin(30)})},very thick,draw = blue!60!black!40]rectangle({2*cos(30)+0.3},{2*sin(30)+0.1});
\draw({2*cos(30)-sin(30)-0.3},{2*sin(30)+cos(30)-0.1})[rotate around={60:({2*cos(30)-sin(30)},{2*sin(30)+cos(30)})},very thick,draw = blue!60!black!40]rectangle({2*cos(30)-sin(30)+0.3},{2*sin(30)+cos(30)+0.1});
\draw[blue]({cos(30)-0.5*sin(30)},{sin(30)+0.5*cos(30)})--node[above]{$\vec{V}$}({4*cos(30)-0.5*sin(30)},{4*sin(30)+0.5*cos(30)})[-latex];
\draw[red]({cos(30)-0.5*sin(30)},{sin(30)+0.5*cos(30)})--node[above]{$\vec{v'}$}({cos(30)-0.5*sin(30)+3*cos(60},{sin(30)+0.5*cos(30)+3*sin(60})[-latex];

\draw({cos(30)-0.5*sin(30)},{sin(30)+0.5*cos(30)})--({3+cos(30)-0.5*sin(30)},{sin(30)+0.5*cos(30)})[-latex]node[anchor=north]{x};
\draw({cos(30)-0.5*sin(30)},{sin(30)+0.5*cos(30)})--({cos(30)-0.5*sin(30)},{3+sin(30)+0.5*cos(30)})[-latex]node[anchor=east]{y};

\draw[blue]({2+cos(30)-0.5*sin(30)},{sin(30)+0.5*cos(30)})arc(0:30:2);
\draw[red]({3*cos(30)-0.5*sin(30)},{3*sin(30)+0.5*cos(30)})arc(30:60:2);
\draw[blue]({3.2*cos(30)},{3.2*sin(30)}) node[]{$\psi$};
\draw[red]({3.4*cos(50)},{3.3*sin(50)}) node[]{$\phi$};
\draw[latex-latex](0.25,-0.3)--node[below]{l}({2*cos(30)+0.25},{2*sin(30)-0.3});
\end{tikzpicture}
\caption{Esquema de un vehículo terrestre de 4 ruedas}
\label{fig:vehir}
\end{figure}

Además el vehículo girará, siempre que las ruedas delanteras no estén alineadas con las ruedas traseras, cambiando así su dirección de avance. Podemos relacionar la velocidad de giro del vehículo $\dot{\psi}$ con el ángulo de orientación de las ruedas delanteras $\phi$, y la velocidad a la que avanzan $\vec{v}'$.  Podemos obtener las componentes de dicha velocidad en ejes cuerpo (paralela y perpendicular a la dirección de avance del vehículo),
\begin{align}
v_{P} = v'\cos(\phi(t))\\
v_{T} = v'\sin(\phi(t)).
\end{align} 

Pero la rueda está unida al vehículo así que su velocidad en la direccíon de avance debe ser la misma que la del vehículo: $v_P \equiv V$.  A partir de esta relación podemos obtener la velocidad tangencial de las ruedas como
\begin{equation}
v_T = V\frac{\sin(\phi)}{\cos(\phi)} = V\tan(\phi).
\end{equation}

Si tomamos como centro de giro del vehículo el centro de su eje trasero, y la batalla (distancia entre ejes) es $l$, obtenemos una expresión para su velocidad de giro,
\begin{equation}
\dot{\psi} = \frac{V}{l}\tan(\phi)
\end{equation}

En resumen, podemos describir el sistema mediante tres ecuaciones de estado $x_1 \equiv x, x_2\equiv	y, x_3 \equiv \psi$:

%\begin{align}
%\dot{x}_1 =  V\cos(x_3)\\  
%\dot{x}_2 =  V\sin(x_3)\\
%\dot{x}_3 = \frac{V}{l}\tan(\phi)
%\end{align}

\begin{equation}
	\begin{cases}
		\dot x_1 &=  V\cos(x_3)\\  
		\dot x_2 &=  V\sin(x_3)\\
		\dot x_3 &= \frac{V}{l}\tan(\phi)
	\end{cases}
\end{equation}

Si consideramos $V=cte$, la única entrada al sistema sería el ángulo de giro de las ruedas $u(t) = \phi(t)$, controlando su valor, podemos hacer girar al vehículo en la dirección deseada. \qed
\end{example}

\section{Simulación o soluciones numéricas del sistema $\Sigma$}
Dado un punto inicial $x(0)$, podemos predecir o calcular numéricamente $x(t)$. El método de \emph{integración de Euler} es un método numérico sencillo que puede darnos información sobre la evolución temporal de los estados y salidas de $\Sigma$. El siguiente algoritmo describe la integración numérica por Euler:
\begin{algo}   \ \\
	\begin{enumerate}
		\item Define el paso de tiempo $\Delta T$
		\item Define $x = x(0)$
		\item Define $y = g(x,u)$
		\item Registra $x$ and $y$, para poder procesarlos después si fuera necesario
		\item Define $t = 0$
		\item Define un tiempo final $t^*$
		\item Mientras $t \leq t^*$ entonces:
			\begin{enumerate}
				\item $x_{\text{nuevo}} = x_{\text{viejo}} + f(x_{\text{viejo}},u)\Delta T$
				\item $y_{\text{nuevo}} = g(x_{\text{nuevo}},u)$
				\item Representa $x$ gráficamente
				\item Registra $x$ e $y$, para poder procesarlos más adelante si fuera necesario
				\item $t = t + \Delta t$
			\end{enumerate}
		\item Representa $x$ e $y$ a lo largo de $t$
	\end{enumerate}
\end{algo}

Este algoritmo rinde bien cuando $\Delta T$ es suficientemente pequeño en función de como de rápido varíe $f$ en el tiempo. Por ahora hemos considerado $u=0$, es decir, no hay control o interacción alguna con el sistema.

Existen, métodos más robustos para obtener la solución numérica de un sistema. Entre ellos, destacan los métodos de Runge-Kutta, y en particular, el uso de los llamados 'pares encajados' de dichos métodos. Describir los detalles de cálculo de dichos métodos queda fuera del alcance de estas notas. Hay muchos paquetes de software como Matlab o Python que incluyen funciones que implementan dichos métodos. En  el apéndice \ref{apend1}. Se incluyen códigos de Matlab con ejemplos y explicaciones de como construirlos.

\section{Sistemas autónomos de segundo orden}
Los sistema autónomos de segundo orden, son especialmente atractivos para el estudio de los fenómenos no lineales porque sus soluciones se pueden representar fácilmente en el llamado plano de fases.
En general, podemos definir un sistema autónomo de segundo orden a partir de dos ecuaciones escalares, una para la derivada temporal de cada estado,
\begin{align}
\dot x_1 &= f_1(x_1,x_2) \label{eq: sysa1}\\
\dot x_2 &= f_2(x_1,x_2) \label{eq: sysa2}
\end{align}

Supongamos que conocemos una solución al sistema $x(t) = [x_1(t),x_2(t)]^T$, que pasa por el punto $x(0) = x_0$. La solución describe una curva sobre el plano $x_1-x_2$ que pasa por el punto $x_0$. Esta curva recibe el nombre de trayectoria u órbita, desde $x_0$, del sistema. El plano $x_1-x_2$ recibe el nombre de plano de fases. La ecuaciones (\ref{eq: sysa1}) - (\ref{eq: sysa2}) representan un vector tangente $\dot x(t) = [\dot x_1(t),\dot x_2(t)]$ a la trayectoria en el punto $x(t)$.

Además podemos considerar $f(x)=[f_1(x),f_2(x)]$ como un campo vectorial definido sobre el plano de fases., es decir asignamos a cada punto $x$ en el espacio de fases el vector $f(x)$. Es posible obtener una representación aproximada del campo vectorial asociado a un sistema si definimos una región de interés del plano de fase, por ejemplo, una región que contenga sus puntos de equilibrio. Seleccionamos un conjunto de puntos $x$ en posiciones equiespaciadas dentro de la región de interés, y representamos en cada punto $x$ el campo asociado $f(x)$ mediante una flecha que apunte en la dirección de $f(x)$ y cuya longitud sea proporcional al módulo de $f(x)$.

El conjunto de todas las trayectorias en el plano de fases de un sistema se denomina su diagrama de fases. de modo análogo al caso de campo vectorial, podemos obtener una representación aproximada del diagrama de fases obteniendo las trayectorias del sistemas para un conjunto suficientemente grande de condiciones iniciales en una región de interés y dibujándolas en el plano de fases.

\subsection{Algunos ejemplos}
Las representaciones en el plano de fases, nos ofrecen un camino sencillo para observar algunos de los fenómenos propios de los sistemas no lineales. A continuación, vamos a emplearlas con algunos de los ejemplos de la sección \ref{sec: ejem}
\begin{example}[Diagrama de fases para el péndulo invertido \ref{Apinv}]
Podemos obtener un sistema autónomo para el péndulo invertido, eliminando la tensión $T$ de las ecuaciones \ref{eq: f},
\begin{figure}
\includegraphics[scale=0.8]{resp_p_inv.eps}
\caption{Evolución temporal y trayectoria en el plano de fases para el péndulo invertido. La condición inicial, $x_0=[\pi/3,0]^T$, aparece representada con un aspa en el espacio de fases.} \label{fig:trpen}
\end{figure}

\begin{align}
\dot x_1 = &x_2 \label{eq:pi1} \\ 
\dot x_2 = &\frac{g}{l}\sin x_1 - \frac{b}{ml}x_2 \label{eq:pi2}
\end{align}
donde $x_1 = \theta$ y $x_2 = \dot \theta$,. Una primera característica del sistema la obtenemos calculando los puntos de equilibrio, $\overline x=[n\pi,0]^T, n \in \mathbb{Z}$. Tenemos un conjunto infinito de posibles puntos de equilibrio, correspondiente a las posiciones verticales del péndulo.

La figura \ref{fig:trpen} muestra una trayectoria del péndulo invertido para una realización particular; $l=g$,$ml=b$ y una condición inicial $x_0=[\pi/3,0]$. 

En ese caso, es fácil analizar lo que sucede, de un modo cualitativo. Las condiciones iniciales sitúan el péndulo cerca de la vertical, formando un ángulo $\theta_0 =\pi/3$ y en reposo. El péndulo caerá y oscilará con una amplitud y velocidad cada vez menor acercándose al punto de equilibrio $\overline \theta = \pi$.

Podemos obtener más información si dibujamos aproximadamente el campo vectorial definido por las ecuaciones (\ref{eq:pi1})-(\ref{eq:pi2}). La figura \ref{fig:pifield} muestra un ejemplo. La región del plano de fases $\theta \in [-3\pi/2,3\pi/2], \dot \theta \in [-3,3]$. Si consideramos el origen de cada vector como una condición inicial del problema, la flecha indicaría la dirección y magnitud del cambio en el plano de fases.  Es fácil intuir, si seguimos el eje de coordenadas $\theta$, la presencia de tres  puntos de equilibrio en la región mostrada y como el central es un puntos de equilibrio inestable, mientras que los dos laterales corresponden a puntos de equilibrio estables.

Por último podemos obtener una información aún más completa si cabe, si representamos un diagramas de fases del sistema.  La figura \ref{fig:piphp} muestra un ejemplo en el que es posible ver los dos puntos de equilibrios estables (focos) y el punto de silla correspondiente al punto $(0,0)$. 

 
\begin{figure}
\centering
\includegraphics[scale=0.8]{pi_field.eps}
\caption{Campo vectorial definido sobre la región  $\theta \in [-3\pi/2,3\pi/2], \dot \theta \in [-3,3]$ del plano de fases para el péndulo invertido. } \label{fig:pifield}
\end{figure}

 \begin{figure}
\centering
\includegraphics[scale=0.8]{pi_php.eps}
\caption{diagrama de fase para el péndulo invertido, cada aspa representa una condición inicial distinta} \label{fig:piphp}
\end{figure}
\qed
\end{example}

\begin{example}[Ciclo límite del oscilador de Van der Pol \ref{AVdP}.] De modo análogo a lo que hemos hecho en el ejemplo anterior, empezamos por determinar los puntos de equilibrio del sistema. Para ello igualamos a cero las ecuaciones del sistema (\ref{eq:vdpst2}, \ref{eq:vdpst2})y resolvemos el sistema resultante,
\begin{align}
0 &= x_2\\
0 &= -\frac{1}{LC}x_1 - m\frac{1}{C}(x_1^2-1)x_2 
\end{align}
La única solución es el origen $x^*=[0,0]^T$. La figura \ref{fig:vandesol}, muestra la evolución temporal de las soluciones y su trayectoria en el espacio de fases para una realización particular: $m = 0.5, L=1,C=1$. Un aspecto destacable es el carácter periódico de las soluciones alcanzadas. Ambas variables presentan una oscilación mantenida, tras un primer periodo de respuesta transitoria. Si observamos la trayectoria en el espacio de fases, vemos como ésta converge desde la posición inicial $x(0)$ a una trayectoria cerrada y estable. Se trata de un ciclo límite. 

\begin{figure}
\centering
\includegraphics[scale=0.4]{VdPsol.eps}
\caption{Evolución temporal y trayectoria en el plano de fases para el Oscilador de Van der Pol. La condición inicial, $x(0)=[3,-1]^T$, aparece representada con un aspa en el espacio de fases.} \label{fig:vandesol}
\end{figure}

\begin{figure}
\centering
\includegraphics[scale=0.4]{fasesvdp.eps}
\caption{Diagrama de Fases para el oscilador de Van der Pol. Las trayectorias obtenidas desde distintas condiciones iniciales, convergen a un ciclo límite.} \label{fig:fasesvdp}
\end{figure}
\qed
\end{example}

\subsection{Ciclos límite}
Los ciclos límites son un tipo de solución específica de algunos sistemas no lineales. Se trata de una solución periódica $x(t+T) =x(t), \forall t>0$ para un determinado periodo  $T>0$. 

Tienen dos características fundamentales, la primera es que corresponde a soluciones de sistemas estructuralmente estables, es decir, el sistema no cambia su comportamiento bajo pequeñas perturbaciones. La segunda es que la amplitud de la oscilación no depende de las condiciones iniciales, o dicho  de otra manera, todas las condiciones iniciales convergen al o divergen del mismo ciclo límite. Podemos, por tanto, encontrar ciclos límites estables o inestables. En el caso de los inestable, solo permanecen en el ciclo límite las soluciones que empiezan en él, mientras que toda solución que cuyas condiciones iniciales estén próximas al ciclo límite se alejarán de él. . Una manera sencilla de observar un ciclo límite inestable, se obtiene integrando el oscilador del Van der Pol, con el tiempo invertido $t\to -\infty$.





\subsection{Sistemas lti\protect\footnote{Los detalles y demostraciones se incluyen en el capítulo \ref{5:lti}}. }

\begin{figure}[ht] 
\subfigure[Nodo estable]{
\includegraphics[width=0.31\linewidth]{lineal_lr.eps}}
\subfigure[Foco estable]{
\includegraphics[width=0.31\linewidth]{lineal_lc.eps}}
\subfigure[Centro]{
\includegraphics[width=0.31\linewidth]{lineal_imaginario.eps}}\\
\subfigure[Nodo inestable]{
\includegraphics[width=0.31\linewidth]{lineal_mr.eps}}
\subfigure[Foco inestable]{
\includegraphics[width=0.31\linewidth]{lineal_mc.eps}}
\subfigure[Punto de silla (inestable)]{
\includegraphics[width=0.31\linewidth]{lineal_silla.eps}}
\caption{Diagrama de fases para sistemas lineales con autovalores distintos no nulos, reales o complejos conjugados}\label{fig:lin1}
\end{figure}

 Como se ha indicado anteriormente, una primera aproximación al estudio de los sistemas dinámicos es obtener sus puntos de equilibrio, linealizar el sistema en torno a ellos, y estudiar el comportamiento del sistema linealizado. Aunque todo esto, se explicará en detalle en el tema \ref{5:lti}, vamos a adelantar aquí algunos resultados, especialmente útiles para el estudio del plano de fases de sistemas no lineales. 
Tomando como partida las ecuaciones (\ref{eq: sysa1}) - (\ref{eq: sysa2}), y suponiendo que el sistema tiene un punto de equilibrio $x^*= [x_1^*,x_2^*]$. Podemos linealizar el sistema en torno al punto de equilibrio,
\begin{align}
\left.\begin{cases}
	\dot x_1 =& f_1(x_1,x_2) \\ \dot x_2 =& f_2(x_1,x_2)
	\end{cases}\right|_{x\approx x^*} \approx
	\begin{cases}
		x_1(t) &= x^* + \delta x_1(t) \\
		x_2(t) &= x^* + \delta x_2(t) \\
	\delta \dot x(t) &= A(t)\delta x(t)  \\
	\end{cases},
\end{align}
\begin{align}\label{eq:A}
	A = D_xf(x^*) &= \left. \begin{bmatrix}
		\frac{\partial f_1}{\partial x_1} &  \frac{\partial f_1}{\partial x_2} \\
		\ \\
		\frac{\partial f_2}{\partial x_1} & \frac{\partial f_2}{\partial x_2}
	\end{bmatrix} \right |_{x=x^*} \quad
\end{align}

El comportamiento del sistema original, en la proximidades del punto de equilibrio $x^*$, es comparable al del sistema $\delta \dot x(t) = A\delta x(t)$ en torno al origen. Se trata de un sistema autónomo lineal e invariante en el tiempo (lti), cuya solución es,

\begin{equation}
\delta x(t) = P\exp(J_rt)P^{-1}\delta x_0 
\end{equation}

donde  $J_r$ es la forma de Jordan \emph{real} de la matriz A y $P=[v_1,v_2]$ es una matriz formada por los autovectores de $A$, $\ J_r=P^{-1}AP$. La matriz de Jordan, en este caso bidimensional, puede tomar una de las siguientes formas,
\begin{equation}\label{eq:240}
\begin{bmatrix}
\lambda_1 & 0\\ 0 & \lambda_2
\end{bmatrix},\ \begin{bmatrix}
\lambda & k\\ 0 & \lambda
\end{bmatrix},\ \begin{bmatrix}
\alpha & -\beta\\ \beta & \alpha
\end{bmatrix},
\end{equation}
donde $k$ es 0 ó 1.
Comencemos por analizar el primer caso, donde $\lambda_1 \ne \lambda_2 \ne 0$. En este caso, las soluciones del sistema corresponde a la combinación de dos exponenciales,
\begin{equation}
\exp\begin{bmatrix}
\lambda_1t &0 \\
0& \lambda_2t
\end{bmatrix}=\begin{bmatrix}
e^{\lambda_1t} &0 \\
0& e^{\lambda_2t} 
\end{bmatrix}
\end{equation}



Tenemos entonces tres posibilidades distintas. Si los dos autovalores son negativos, ambas exponenciales tienden a cero cuando el tiempos tiende a infinito. Se dice entonces que el sistema es asintóticamente (exponencialmente) estable y el origen es un nodo estable. Si ambos autovalores son positivos, las soluciones tienden a infinito, el sistema es inestable y el origen es  inestable. Si uno de los autovalores es positivo y el otro negativo, es sistema resulta igualmente inestable, una solución tiende a cero y la otra a infinito. El origen es en este caso un punto de silla.

El tercer caso mostrado en la ecuación \ref{eq:240} corresponde a un par de autovalores complejos conjugados,
$\lambda_1 = \alpha + j \beta, \lambda_2 =\alpha -j\beta$.  Las soluciones tomarían la forma,

\begin{equation}
\exp\begin{bmatrix}
\alpha t & -\beta t\\ \beta t & \alpha t
\end{bmatrix} = 
\begin{bmatrix}
e^{\alpha t}\cos(\beta t) & e^{\alpha t}\sin(\beta t)\\
-e^{\alpha t}\sin(\beta t) & e^{\alpha t}\cos(\beta t)
\end{bmatrix}
\end{equation}

De nuevo. Podemos distinguir varios casos atendiendo a la forma que tome la parte real del autovalor. Si $\alpha < 0$, el sistema tenderá asintóticamente a 0, describiendo una espiral. Se dice entonces que el origen es un foco atractivo. Si $\alpha > 0$ las soluciones divergen, y se alejan del origen describiendo una espiral. Se dice entonces que el origen es un foco repulsivo. Por último, si $\alpha = 0$, los autovalores son números imaginarios. Las soluciones son trayectorias elípticas cerradas en torno al origen que se denomina centro. La figura \ref{fig:lin1} muestra diagramas de fases correspondientes a ejemplos de los casos descritos hasta ahora.

\begin{figure}
\centering
\subfigure[Nodo inestable. $\lambda_1 = \lambda_2 > 0$\label{fig:lin2a}]{
\includegraphics[width=0.31\linewidth]{lineal_ji.eps}}
\subfigure[Nodo estable. $\lambda_1 = \lambda_2 < 0$\label{fig:lin2b}]{
\includegraphics[width=0.31\linewidth]{lineal_je.eps}}\\
\subfigure[$\lambda_1=0,\ \lambda_2 > 0$\label{fig:lin2c}]{
\includegraphics[width=0.31\linewidth]{lineal_0i.eps}}
\subfigure[$\lambda_1 = 0,\ \lambda_2 > 0$\label{fig:lin2d}]{
\includegraphics[width=0.31\linewidth]{lineal_0e.eps}}
\subfigure[$\lambda_1 = 0,\ \lambda_2 = 0$\label{fig:lin2e}]{
\includegraphics[width=0.31\linewidth]{lineal_doblecero.eps}}
\caption{Diagrama de fases para sistemas lineales con autovalores iguales y autovalores nulos. En este último caso, figuras (c), (d) y (e) se ha mercado con un aspa el final de las simulaciones para mostrar la recta de equilibrio y el sentido del movimiento de las trayectorias. }\label{fig:lin2}
\end{figure}

En cuanto al segundo caso, si $\lambda_1 = \lambda_2 \ne 0$, las soluciones tomarán la forma,
\begin{equation}
\exp\begin{bmatrix}
\lambda t& kt\\ 0 & \lambda t
\end{bmatrix} = 
\begin{bmatrix}
e^{\lambda t} & e^{\lambda t}kt\\
0 & e^{\lambda t}
\end{bmatrix}
\end{equation}

Cualitativamente las soluciones son parecidas a la de los nodos descritos en el caso en el que ambos autovalores son distintos: dependiendo del signo de $\lambda$ tendremos un nodo repulsivo o un nodo atractivo. Las figuras \ref{fig:lin2a} y \ref{fig:lin2b} muestran un ejemplo de cada caso. 

Quedan por analizar los casos en que los autovalores son nulos. Así, por ejemplo, para $\lambda_1 = 0, \lambda_2 \ne 0$,  

\begin{equation}
\exp\begin{bmatrix}
0 & 0\\ 0 & \lambda_2 t
\end{bmatrix} = 
\begin{bmatrix}
1 & 0\\
0 & e^{\lambda_2 t}
\end{bmatrix}
\end{equation}

En este caso tenemos que el autovector $v_1$ correspondiente al autovalor $\lambda_1=0$ pertenece al kernel de la matriz $A$, $Av_1 = 0\cdot v_1 = 0$. Tenemos por tanto, que todos puntos de la recta, definida por el vector $v_1$ son puntos de equilibrio. Si $\lambda_2 >0$ todas las trayectorias serán perpendiculares a dicha recta y tenderán alejarse de ella, El sistema será inestable. Si por el contrario $\lambda_2 < 0 $ el sistema será estable y las trayectorias serán perpendiculares y terminarán en la recta definida por $v_1$. Las figuras \ref{fig:lin2c} y \ref{fig:lin2d} muestran un ejemplo de cada caso en el que es fácil observar la recta de equilibrio, señalada mediante aspas.

Por último, para el caso en que ambos autovalores son nulos. Si la matriz del sistema es la matriz cero, estamos ante un caso trivial en que todos los puntos del plano son puntos de equilibrio, En otro caso, las soluciones tomarían la forma,

\begin{equation}
\exp\begin{bmatrix}
0 & t\\ 0 & 0
\end{bmatrix} = 
\begin{bmatrix}
1 & t\\
0 & 1
\end{bmatrix}
\end{equation}

De nuevo, la recta definida por el autovector $v_1$ define los puntos de equilibrio del sistema. Las trayectorias en el espacio de las fases serán líneas rectas paralelas a dicha línea. La dirección del movimiento dependerá de las condiciones iniciales. La figura \ref{fig:lin2e} muestra un ejemplo.

Como se decía al principio de esta sección, estudiar el comportamiento de los sistemas lineales cerca de su punto de equilibrio $x=0$, tiene interés para los sistemas no lineales, ya que en muchos casos es posible conocer cual será el comportamiento \emph{local} de un sistema no lineal cerca de un punto de equilibrio, linealizando el sistema en torno a dicho punto de equilibrio. Hasta qué punto los resultados de la linealización son válidos, va a depender en cómo de robustos son los sistemas lineales obtenidos frente a pequeñas perturbaciones. 

En general, para sistemas lineales cuyos autovalores tiene parte real distinta de cero, Pequeñas perturbaciones en la estructura del sistema no cambiarán significativamente el valor de la parte real de los autovalores, por tanto conservarán su signo positivo o negativo y el sistema seguirá siendo inestable o estable. De estos sistemas se dice que son estructuralmente inestables o estructuralmente estables, y su punto de equilibrio $x=0$ se dice que es hiperbólico. 

Si el sistema resultante de linealizar un sistema no lineal da como resultado un sistema lineal estructuralmente estable o inestable, el sistema no lineal será también inestable o estable en las proximidades del punto de equilibrio. Este resultado es generalizable a sistemas de dimensiones mayores que 2.

Para  sistemas lineales cuyos autovalores tienen parte real nula, cualquier perturbación, por pequeña que sea, modificará el valor de la parte real de los autovalores, haciéndolos positivos o negativos. Como resultado, si al linealizar un sistema no lineal se obtiene como resultado un sistema autovalers con parte real negativa y/o  autovalores con parte real nula, no es posible concluir nada sobre el comportamiento del sistema no lineal en las proximidades del punto de equilibrio. 

\subsection{Bifurcaciones}

Como hemos visto, en las secciones anteriores, el comportamiento de un sistema de segundo orden podría determinarse por las características de sus puntos de equilibrio, la existencia de ciclos límites o, en general, órbitas periódicas y por su estabilidad.
Un aspecto interesante que ha salido de pasada, es el estudio de la estabilidad estructural, es decir,  cuando un sistema mantiene el mismo comportamiento bajo pequeñas perturbaciones.

Supongamos que tenemos un sistema dinámico que depende de un parámetro $\lambda$,

\begin{equation}
\dot x = g(x,\lambda), x \in \ \mathbb{R}^n, \lambda \in \ \mathbb{R}^p,
\end{equation}
donde $g$ es una función diferenciable $\mathcal{C}^r$ en algún conjunto abierto en $ \mathbb{R}^n \times  \mathbb{R}^p$.

supongamos además que el sistema tiene un punto de equilibrio $x^*$ para un determinado valor del parámetro $\lambda = \lambda^*$,

\begin{equation}
g(x^*,\lambda^*) = 0.
\end{equation}

En primer lugar, podemos estudiar la estabilidad del punto de equilibrio, linealizando el sistema en torno al punto de equilibrio y estudiando la estabilidad del sistema linealizado. En segundo lugar, podríamos plantearnos cómo afectan a la estabilidad del sistema pequeñas variaciones en el valor del parámetro $\lambda^*$.

Si linealizamos,
\begin{equation}
\delta \dot{x} = A \delta x,
\end{equation}
donde ahora, el jacobiano en el punto de equilibrio depende también del parámetro $\lambda$,

\begin{equation}
A = D_x(x^*,\lambda^*).
\end{equation}

Tendremos que si el punto de equilibrio es hiperbólico --no hay ningún autovalor de $A$ sobre el eje imaginario-- El sistema será estable o inestable según el signo de la parte real de los autovalores de $A$. Esto nos resuelve también la cuestión del efecto de las pequeñas variaciones de $\lambda$. En principio, éstas variaciones no cambiarán la naturaleza del punto de equilibrio.

Sin embargo, si la matriz $A$ tiene autovalores en el eje imaginario, pequeños variaciones de $\lambda$ pueden cambiar completamente el comportamiento del sistema. Dando lugar a la creación y/o destrucción de puntos de equilibrio, a la aparición de ciclos límite e incluso de comportamiento caótico.

El tema es bastante complejo, y nos vamos a limitar a ilustrarlo con algunos casos y ejemplos. Empecemos con sistemas en una dimensión en los que el sistema linealizado tiene un autovalor cero, cuando el parámetro toma el valor cero.

\begin{example}[Bifurcación silla-nodo]
\begin{figure}
\subfigure[Silla-nodo \label{fig:saddlenode}]{\includegraphics[scale=0.35]{sadlenode2.eps}}
\subfigure[Transcrítica \label{fig:trans}]{\includegraphics[scale=0.35]{trans2.eps}}
\caption{Ejemplos de bifurcaciones silla-nodo y transcrítica}
\end{figure}
Consideremos el sistema,

\begin{equation*}
\dot x = \mu - x^2,\  x, \mu \in \mathbb{R},  
\end{equation*}

Es fácil comprobar, que para $\mu =0$ el punto $(x = 0)$ es el único punto de equilibrio. Si linealizamos en torno a cero, obtenemos que el autovalor correspondiente sería cero.  No podemos por tanto determinar la estabilidad del sistema no lineal. Supongamos que tomamos ahora $\mu > 0$. Para cada valor de $\mu$, tendremos que los puntos de equilibrio toman valores $x^* = ( \pm \sqrt{\mu})$. Es decir, el sistema pasaría de tener un punto de equilibrio a tener dos. Si linealizamos el sistema en torno a los puntos de equilibrio, tendremos, que para uno de los puntos $x^*=+\sqrt{\mu}$ el sistema linealizado es asintóticamente estable: $ \delta\dot{x} = -2\sqrt{\mu}\delta x$. Para el otro punto, $x^*=-\sqrt{\mu}$, el sistema es asintóticamente inestable $ \delta \dot{x} = +2\sqrt{\mu}\delta x$.
Para el caso en 	que $\mu<0$ no existen en el sistema puntos de equilibrio.

El sistema descrito muestra un ejemplo de \emph{bifurcación}. El punto $(x,\mu)=(0,0)$ recibe el nombre de punto de bifurcación y el valor del parámetro $\mu = 0$ es un valor de bifurcación. Por último la figura \ref{fig:saddlenode} muestra el \emph{diagrama de bifurcación} de este sistema. En él, se ha representado la parábola $x=\pm \sqrt{\mu}$ correspondiente a la posición de los puntos de equilibrio del sistema en función de $\mu$. La rama representada con línea continua, corresponde a los nodos estables, mientras que la rama representada con una línea discontinua, corresponde a los nodos inestable. Sobre el diagrama, se han representado también las trayectorias que seguiría el sistema para diferentes valores iniciales de la variable $x$, las  aspas representan posiciones iniciales y los círculos las posiciones finales para el tiempo de simulación empleado. Es fácil ver cómo, para valores positivos de $\mu$  las trayectorias se alejan de la rama negativa de la parábola y, en su caso, tienden a converger en la positiva que contiene los puntos de equilibrio estables. Para $\mu =0$ los dos nodos colapsan en un único punto de equilibrio; condiciones iniciales negativas produce trayectorias que se alejan del origen, mientras que para condiciones iniciales positivas, las trayectorias convergen al origen. Para valores de $\mu<0$ no hay puntos de equilibrio y $x\to -\infty$ cuando $t\to \infty$.

Este tipo de bifurcaciones en las que a un lado de un valor del parámetro ($\mu$) no hay puntos de equilibrio y en el otro lado hay dos puntos de equilibrio, se conocen con el nombre de bifurcación silla-nodo (\emph{saddle-node}).  

\qed
\end{example}

\begin{example}[Bifurcación transcrítica] Consideremos ahora el sistema,

\begin{equation*}
\dot x = \mu x - x^2,\  x, \mu \in \mathbb{R},  
\end{equation*}
Al igual que en el ejemplo anterior, si hacemos $\mu=0$, el único punto de equilibrio es $x=0$ y de nuevo en este caso, el autovalor del sistema lineal vale cero. Si dejamos que $\mu$ tome valores distintos de cero, el sistema tiene dos puntos críticos para cada valor de $\mu$, $x^*=0$ y $x^*=\mu$. No es difícil comprobar que para el sistema linealizado tendremos dos situaciones distintas. Si $\mu>0$., el punto de equilibrio $x^*=\mu$ es estable y el punto de equilibrio $x^*=0$ es inestable. Para $\mu<0$ el carácter de los puntos críticos se invierte. La figura \ref{fig:trans} muestra el diagrama de bifurcación para este sistema.  Es fácil ver cómo para $\mu<0$ las trayectorias divergen de la recta $x=\mu$ y convergen, en su caso, a la recta $x=0$. Para $\mu>0$ la situación se invierte y ahora las trayectorias convergen a $x=\mu$ y divergen de $x=0$. Para $\mu=0$ los puntos de equilibrio colapsas en $x=0$. Este tipo de bifurcación se conoce con el nombre de bifurcación transcrítica
\qed
\end{example}
 
\begin{example}[Bifurcación en horquilla\footnote{en inglés: \emph{pichtfork}, la traducción es nuestra.}]
Para el sistema,
\begin{equation*}
\dot x = \mu x - x^3,\  x, \mu \in \mathbb{R},  
\end{equation*}
Se cumple de nuevo la condición de los ejemplos anteriores para $\mu=0$. Para $\mu>0$, se obtienen tres puntos críticos $x^*=0$ inestable, y $x^*=\pm\sqrt{\mu}$, ambos estables (compruébalo). Para $\mu<0$ solo tenemos un punto crítico $x^*=0$ que resulta ser estable, puesto que los otros corresponden a raíces imaginarias ($x^* = \pm j\sqrt{|\mu|}$). La figura \ref{fig:horquilla} muestra el diagrama de bifurcación para este sistema. La interpretación es similar a la de los dos ejemplos anteriores.
\begin{figure}
\subfigure[Horquilla \label{fig:horquilla}]{\includegraphics[scale=0.35]{pitchfork2.eps}}
\subfigure[No existe bifurcación \label{fig:nobif}]{\includegraphics[scale=0.35]{nobif2.eps}}

\caption{Ejemplo de bifurcación en horquilla y un caso en que no hay bifurcación}
\end{figure}

\qed
\end{example}
 
\begin{example}[No bifurcación] Para el sistema,
\begin{equation*}
\dot x = \mu - x^3,\  x, \mu \in \mathbb{R},  
\end{equation*}

Se cumple la misma condición que en los casos anteriores para $\mu=0$. Es fácil comprobar que para $\mu \neq 0$, el sistema sigue teniendo un único punto crítico $x^*=\sqrt[3]{\mu}$. No hay cambio por tanto, en el número de puntos críticos, ni tampoco en en el carácter del único punto crítico, que es siempre un nodo estable.  Podemos afirmar que en este caso no se da una bifurcación tal y como se muestra en el diagrama de bifurcación de la figura \ref{fig:nobif}.

\qed
\end{example}

 El término bifurcación no es fácil de definir. Una definición cualitativa e informal sería decir que una bifurcación para un punto de equilibrio de un sistema dinámico se produce cuando el cambio de un parámetro modifica el comportamiento del sistema en las proximidades del punto crítico.
 
Veamos por último un ejemplo del caso en que un sistema de segundo orden tiene dos autovalores imaginarios puros.
\begin{figure}
\centering
\subfigure[Diagrama de bifurcación]{\includegraphics[scale=0.7]{hopf2.eps}}\\
\subfigure[$\mu < 0$.]{\includegraphics[scale=0.35]{hopfsub.eps}}
\subfigure[$\mu >0$]{\includegraphics[scale=0.35]{hopfsuper.eps}}
\caption{Ejemplo de bifurcación de Hopf}
\label{fig:hopf}
\end{figure}

\begin{example}[Bifurcación de Hopf]\label{ex:hopf}
Consideremos el sistema,
\begin{align*}
\dot x_1 = x_1(\mu-x_1^2-x_2^2)-x_2\\
\dot x_2 = x_2(\mu-x_1^2-x_2^2) +x_1
\end{align*}

El único punto de equilibrio del sistema es $x^*= (0,0)$. Si linealizamos el sistema obtenemos en torno a dicho punto, obtenemos que el jacobiano vale,
\begin{align*}
	A = D_xf(x^*) &=  \begin{bmatrix}
		\mu &  -1 \\
		1 & \mu
	\end{bmatrix}
\end{align*}
Que tiene autovalores $\mu\pm j$. Estos autovalores son imaginarios puros para $\mu=0$ y cruzan el eje imaginario cuando $\mu$ pasa de negativo a positivo. Para $\mu<0$ el origen es un foco atractivo, para $\mu>0$ se convierte en un foco repulsivo.
Podemos analizarlo más fácilmente si transformamos el sistema a coordenadas polares, $x_1= r\cos(\theta)$, $x_2 = r\sin(\theta)$,
\begin{align*}
\dot r = \mu r - r^3\\
\dot \theta = 1
\end{align*}

En esta representación las variables aparecen desacopladas. El valor $r=0$, corresponde con el punto crítico del sistema. Para el caso en que $\mu>0$, tenemos que $r=\sqrt{\mu}$ hace nula la ecuación  $\dot r = \mu r - r^3$. Esta curva corresponde a un ciclo límite; en torno al origen de radio $\sqrt{\mu}$. Por tanto, para este caso, aunque el origen sea un punto inestable, el sistema converge al ciclo límite y es estable. La figura \ref{fig:hopf} Muestra el diagrama de bifurcación de este sistema.  En este caso, el punto de equilibrio, $x^*=(0,0)$ coincide con el eje $\mu$. Para $\mu$ negativo es atractivo y el sistema converge para cualquier condición inicial. Para $\mu$ positivo se ha representado ---junto al eje $\mu$ que ahora representa un punto de equilibrio inestable--   la forma del ciclo límite como un paraboloide. En este caso, el sistema converge, para cualquier condición inicial,  a la sección del paraboloide correspondiente al valor de $\mu$ del sistema.
\qed
\end{example}

Para terminar, reiterar que aquí solo se muestran algunos ejemplos de bifurcaciones. Un estudio completa del tema queda fuera de los objetivos de estos apuntes.
\newpage
\section*{Ejercicios}
\begin{enumerate}
\item El modelo de Volterra-Lotka estudia la evolución de un sistema formado por dos poblaciones una de depredadores y otra de presas que conforman un ecosistema cerrado.
\begin{align*}
&\dot x_1 = ax_1-cx_1x_2\\
&\dot x_2 =-bx_2 + dx_1x_2\\
&a,b,c,d \in \mathbb{R}^+,
\end{align*}
donde $x_1$ representa la población de presas; $x_2$ la de depredadores; $a$ la tasa de nacimiento de las presas, que es función de la cantidad de alimento que reciben; $b$ es la tasa de defunción de los depredadores;$c$ y $d$ modelan la interacción entre los depredadores y las presas.

\begin{enumerate}
\item Simular el modelo de Volterra-Lotka para parámetros fijos, por ejemplo $a=b=c=d=1$. Emplear para ello tanto	 Matlab como Simulink. Emplear distintas condiciones iniciales. Obtener tanto un gráfico de la evolución temporal de los estados como el diagrama de fases.
\item Modificar el modelo, de modo que el parámetro $a$ pase a ser un una función periódica $a = \frac{a_0}{2}\sin(\omega t)+\frac{a_0}{2}$. Estudiar el efecto de la frecuencia en el modelo.
\end{enumerate}
\item El modelo de Lorenz fue propuesto en 1963 por Edward Lorenz como un modelo simplificado de convección atmosférica.
\begin{align*}
&\dot x_1 = \sigma (x_2 -x_1)\\
&\dot x_2 = x_1 (\rho -x_3) - x_2\\
&\dot x_3 = x_1x_2 -\beta x_3\\
& \sigma, \rho, \beta \in \mathbb{R}^+
\end{align*}
Para los valores $\sigma  = 10$, $\beta = 8/3, \rho = 28$ el sistema exhibe soluciones caóticas; para casi todas las condiciones iniciales el sistema converge a un conjunto invariante conocido con el nombre de Atractor de Lorentz.
\begin{enumerate}
\item Utilizar tanto Matlab como Simulink para simular el modelo de Lorenz. Emplear para ello los parámetros indicados más arriba.
\item Comprobar mediante simulación,  que para $\rho < 1$, el sistema converge a su único  punto de equilibrio. Para $\rho = 1$ el sistema sufre una bifurcación de horquilla, Obtener los puntos de equilibrio del sistema y comprobarlo.
 \end{enumerate}
 
\item La siguiente ecuación diferencial define un modelo lineal, conocido a veces como el modelo Masa-Muelle-Amortiguador,
\begin{equation*}
m\ddot y + c \dot y + k y = F(t),
\end{equation*}

donde $y$ es la posición del sistema, $m$ representa la masa, $c$ es un coeficiente de amortiguamiento, y $k$ es una constante recuperadora, $F(t)$ representa una fuerza externa. El modelo es genérico en el sentido de que reproduce el símil mecánico de sistemas de muy diverso tipo.
\begin{enumerate}
\item Obtener el modelo equivalente en variable de estados, de modo que una variable de estado sea la posición del sistema $y$ y la otra la velocidad $\dot y$
\item Simular el modelo para valores $m=2.5Kg$, $c=0.6Ns/m$ y $k=0.4N/m$, empelando Matlab. Considerar los siguientes casos,
\item La señal de entrada $F$ es nula. Probar para distintos valores de las condiciones iniciales: $y(0) = 1$, $ \dot y(0) =0$; $y(0) = 0$, $\dot y(0) = 1$; $y(0) = 1$, $ \dot y(0) = 1$. Representar la evolución temporal de los estado durante un intervalo de $50s$.
\item La señal de entrada es una fuerza constante de $1N$.
\item \label{c} La señal de entrada  es una sinusoide de frecuencia $0.1 rad/s$. Representar en un mismo gráfico la evolución temporal de la señal de entrada y de la señal de salida en un intervalo de $200s$. ¿Qué desfase se observa entre la entrada y la salida para un tiempo mayor a $50s$? ¿Está estabilizada la señal de salida?
\item \label{d} repetir el apartado \ref{c}), para señales sinusoidales de frecuencias, $0.4 rad/s$ y $1 rad/s$ ¿Cuánto tarda en estabilizarse la señal en estos casos? ¿Qué conclusión se puede extraer del análisis de los resultados de estos dos últimos ejercicios?
\end{enumerate}

\item Empleando las ecuaciones \ref{eq:pi1} y \ref{eq:pi2} para el modelo de un péndulo invertido sin par externo,
\begin{enumerate}
\item Linealizar el sistema en torno al punto de equilibrio $\theta = \pi$.
\item simular mediante matlab tanto el sistema original como el linealizado. Emplear para ello $m = g = l = b = 1$. Considerar como condiciones iniciales para $X_1$ distintos ángulos cada vez más alejados de $\pi$ y p $x_2 = 0$ en todos los casos. Representar en un mismo gráfico la evolución temporal de $x_1$ para el sistema original y el linealizado. Discutir la validez de la aproximación lineal  en función del las condiciones iniciales empleadas.
\end{enumerate}

\item Dado el sistema,
\begin{align*}
\dot x_1 = -x_1(1-x_1^2-x_2^2)+x_2\\
\dot x_2 = -x_2(1-x_1^2-x_2^2) -x_1
\end{align*}

\begin{enumerate}
\item Obtener un diagrama de fases del sistema, empleando para ello Matlab. Comprobar gráficamente que el círculo $x_1^2+x_2^2 = 1$  marca el límite de la región de estabilidad del punto de equilibrio del sistema $(0,0)$. Pista: una manera clara de obtener el resultado es integrar para un intervalo de tiempo negativo, en ese caso, el límite de estabilidad se convierte en un ciclo límite.
\item Comprobar que el resultado de integrar el sistema para un tiempo negativo es idéntico a obtener las soluciones del sistema presentado en el ejemplo \ref{ex:hopf}, tomando $\mu=1$.
\end{enumerate}

\end{enumerate}
\chapter{Sistemas lineales}\label{lineales}

\section{Mapas lineales}

En este capítulo nos vamos a centrar en una clase de sistema llamado \emph{sistema linea en el espacio de estados}. Primero, necesitamos la noción de que es un \emph{mapa lineal}.

\begin{definition} Considera un mapa $H: V \to W$. Si $H$ preserva la operación suma y la multiplicación por un escalar, i.e.,
\begin{align}
	H(v_1+v_2) &= H(v_1) + H(v_2), \quad v_1, v_2\in\mathbb{V} \nonumber \\
	H(\alpha v_1) &= \alpha H(v_1), \quad \alpha\in\mathbb{K} \nonumber,
\end{align}
	entonces $H$ es un \emph{mapa lineal}.
\end{definition}

\subsection*{Ejercicio}
Comprueba si los siguientes mapas son lineales

\begin{enumerate}
	\item $H_1(v) := Av, A\in\mathbb{R}^{n\times n}, \quad v\in\mathbb{R}^n$
	\item $H_2(v) := \frac{\mathrm{d}}{\mathrm{dt}}(v(t)), \quad v\in\mathcal{C}^1$
	\item $H_3(v) := \int_0^T v(t) dt, \quad v\in\mathcal{C}^1, T\in\mathbb{R}_{\geq 0}$
	\item $H_4(v) := D(v) := v(t - T), \quad v\in\mathcal{C}^1, T\in\mathbb{R}_{\geq 0}$
	\item $H_5(v) := Av + b, \quad A\in\mathbb{R}^{n\times n}, v,b\in\mathbb{R}^n$
\end{enumerate}

\section{Sistemas continuos y lineales en el espacio de estados}

El siguiente sistema define un sistema continuo y lineal en el espacio de estados.

\begin{equation}
	\Sigma := \begin{cases}
	\dot x(t) &= A(t)x(t) + B(t)u(t), \quad x\in\mathbb{R}^n, A(t) \in \mathbb{R}^{n \times n} u\in\mathbb{R}^k, B(t) \in \mathbb{R}^{n \times k} \\
	y(t) &= C(t)x(t) + D(t)u(t), \quad y\in\mathbb{R}^m, C(t) \in \mathbb{R}^{m \times n}, D(t) \in \mathbb{R}^{m \times k}
	\end{cases}
	\label{eq: linsys}
\end{equation}

Se llama \emph{sistema LTI} a un sistema lineal y no variante en el tiempo (\emph{Linear Time-Independent}).

\subsection*{Ejercicio}
 Escribe un sistema continuo y lineal en el espacio de estados como un diagrama de bloques entrada/salida y comprueba que es un mapa lineal.

\subsection*{Ejercicio}
Interconecta sistemas continuos y lineales en el espacio de estados y comprueba que el sistema resultante es otro sistema continuo y lineal en el espacio de estados.

Reescribe como un único sistema lineal\footnote{Por abreviar, cuando no exista ambiguedad, llamaremos sistema lineal al sistema continuo y lineal en el espacio de estados} como en (\ref{eq: linsys}):

\begin{enumerate}
	\item La conexión en serie (o en cascada) de dos sistemas lineales, i.e., $y_1(t) = u_2(t)$.
	\item La conexión en paralelo de dos sistemas lineales, i.e., $y(t) = y_1(t) + y_2(t)$.
	\item La conexión realimentada, i.e., $u_1(t) = u(t) - y(t)$, asumiendo que $u, y, \in\mathbb{R}^k$.
\end{enumerate}

\begin{figure}
\centering
\begin{tikzpicture}[auto, node distance=3.5cm, >=latex']
	\node [input, name=input] {};
	\node [block, right of=input] (system) {$\Sigma_1$};
	\node [output, right of=system] (output) {};
	\draw [draw,->] (input) -- node {$u(t) = u_1(t)$} (system);
	\draw [->] (system) -- node [name=y] {$y_1(t)$}(output);
	\node [block, right of=output] (system2) {$\Sigma_2$};
	\node [output, right of=system2] (output2) {};
	\draw [draw,->] (output) -- node {$u_2(t)$} (system2);
	\draw [->] (system2) -- node [name=y] {$y_2(t) = y(t)$}(output2);
\end{tikzpicture}
	\caption{Conexión en serie de dos sistemas lineales y continuos en el espacio de estados.}
	\label{fig: series}
\end{figure}

\section{Solución a sistemas continuos lineales en el espacio de estados}
La solución a una ecuación diferencial ordinaria viene dada por la suma de dos soluciones: la solución a la parte homogénea, y la solución a la parte no homogénea.

\begin{equation}
	\dot x(t) = \underbrace{A(t)x(t)}_{\text{homogénea}} + \underbrace{B(t)u(t)}_{\text{no homogénea}}
	\label{eq: xdyn}
\end{equation}

\begin{theorem}{Serie de Peano-Barker.}
	
La solución única al sistema homogéneo $\dot x = Ax$ viene dada por
	\begin{equation}
		x(t) = \Phi(t,t_0)x(t_0), \quad x(t_0)\in\mathbb{R}^n, t\geq 0,
	\end{equation}
donde
	\begin{align}
		\Phi(t,t_0) := I + \int_{t_0}^t A(s_1)ds_1 + \int_{t_0}^t A(s_1) \int_{t_0}^{s_1} A(s_2)ds_2ds_1 \nonumber \\ + \int_{t_0}^t A(s_1) \int_{t_0}^{s_1} A(s_2)\int_{t_0}^{s_2} A(s_3) ds_3ds_2ds_1 + \dots . \label{eq: ser}
	\end{align}
\end{theorem}



Esbozo de la prueba: \\
Primero calculamos la siguiente derivada
	\begin{align}
		\frac{d}{dt}\Phi(t,t_0) &= A(t) + A(t)\int_{t_0}^{t}A(s_2)ds_2 \nonumber \\ &+ A(t)\int_{t_0}^t A(s_2) \int_{t_0}^{s_2} A(s_3)ds_3ds_2 + \dots \nonumber \\
		&= A(t) \Phi(t,t_0).
	\end{align}
	Afirmamos que la solución a la parte homogénea de (\ref{eq: xdyn}) es $x(t) = \Phi(t,t_0)x_0$ cuya derivada con respecto al tiempo es
\begin{align}
	\frac{d}{dt} x &= \frac{d}{dt}\Phi(t,t_0)x_0 \nonumber \\
	&= A(t) \Phi(t,t_0) x_0 \nonumber \\
	&= A(t)x(t),
\end{align}
lo cual prueba la identidad $\dot x = A(t)x(t)$ dado que $x(t) = \Phi(t,t_0)x_0$. Para terminar la prueba, necesitaríamos probar que la serie (\ref{eq: ser}) converge para todo $t\geq t_0$.

La matriz $\Phi(t,t_0)$ es llamada \textbf{\emph{matriz de transición de estados}}. Dada una condición inicial $x_0$, podemos predecir $x(t)$ en (\ref{eq: xdyn}) iterando $\Phi(t,t_0)$ en el caso de que no existiera ninguna interacción con el sistema, i.e., $u(t) = 0, t\geq t_0$.

\subsection{Propiedades}
\begin{enumerate}
	\item $\Phi(t, s)\Phi(s, \tau) = \Phi(t, \tau)$
	\item $\Phi(t, s)^{-1} = \Phi(s, t)$
	\item $\Phi(t, t) = I$
\end{enumerate}
\subsection*{Ejercicio}
Comprobar que
\begin{align}
	x(t) &= \Phi(t,t_0)x_0 + \int_{t_0}^t \Phi(t,\tau)B(\tau)u(\tau)d\tau  \label{eq: solx} \\
	y(t) &= C(t)\phi(t,t_0)x_0 + \int_{t_0}^t C(t)\Phi(t,\tau)B(\tau)u(\tau)d\tau + D(t)u(t), \label{eq: soly}
\end{align}
son las soluciones a

\begin{align}
	\dot x(t) &= A(t)x(t) + B(t)u(t)  \nonumber \\
	y(t) &= C(t)x(t) + D(t)u(t).  \nonumber
\end{align}

\section[Solución a sistemas LTI en el espacio de estados]{Solución a sistemas invariantes en el tiempo, continuos y lineales en el espacio de estados}\label{5:lti}

Comunmente conocidos como sistemas \emph{lti} (linear time invariant), son los sistemas en los que nos centraremos principalmente en el resto del curso. La matriz $\Phi(t,t_0)$ puede ser hallada analíticamente cuando $A$ es una matriz de coeficientes constantes. Si $A$ es constante, entonces podemos sacarla de las integrales en (\ref{eq: ser}), quedando
\begin{align}
	\Phi(t,t_0) := I + A \int_{t_0}^t ds_1 + A^2 \int_{t_0}^t \int_{t_0}^{s_1} ds_2ds_1 \nonumber \\ + A^3 \int_{t_0}^t \int_{t_0}^{s_1} \int_{t_0}^{s_2} ds_3ds_2ds_1 + \dots \label{eq: phi},
\end{align}
y observando que las siguientes integrales tienen solución analítica
\begin{align}
	\int_{t_0}^t ds_1 &= (t-t_0) \nonumber \\
	\int_{t_0}^t\int_{t_0}^{s_1} ds_2ds_1 &= \frac{(t-t_0)^2}{2} \nonumber \\
	\vdots \nonumber \\
	\int_{t_0}^t\int_{t_0}^{s_1} \cdots \int_{t_0}^{s_{k-2}}\int_{t_0}^{s_{k-1}}ds_k ds_{k-1} \cdots ds_2ds_1 &= \frac{(t-t_0)^k}{k!}, \nonumber
\end{align}
entonces tenemos que (\ref{eq: phi}) es calculada como
\begin{equation}
	\Phi(t,t_0) = \sum_{k=0}^{\infty} \frac{(t-t_0)^k}{k!}A^k,
\end{equation}
lo cual es familiar a la serie de Taylor de una función exponencial. Por ejemplo, para un escalar $x$, tenemos que $e^x := \sum_{k=0}^{\infty}\frac{1}{k!}x^k = 1 + x + \frac{x^2}{2} + \frac{x^3}{3!} + \dots $. De hecho, la definición de la \emph{exponencial de una matriz} es
\begin{equation}
	exp(A) = I + A + \frac{1}{2} A^2 + \frac{1}{3!} A^3 + \dots
\end{equation}
Fijemos $t_0 = 0$ por conveniencia, entonces
\begin{align}
	\Phi(t,0) &= I + tA + \frac{t^2}{2} A^2 + \frac{t^3}{3!} A^3 + \dots \nonumber \\
	&= exp(At),
\end{align}
por lo tanto, la solución a la parte homogénea (\ref{eq: xdyn}) teniendo $A$ con coeficientes constantes y fijando $t_0 = 0$ es
\begin{equation}
	x(t) = exp(At)x_0,\quad t\geq 0.
	\label{eq: xexp}
\end{equation}

Para continuar, necesitamos el siguiente resultado de álgebra lineal.
\begin{theorem}
\textbf{Forma de Jordan}. Para una matriz cuadrada $A\in\mathbb{C}^{n \times n}$, existe un cambio de base no singular (invertible) $P\in\mathbb{C}^{n \times n}$ que transforma $A$ en
\begin{equation}
	J = P^{-1}AP = \begin{bmatrix}
		J_1 & 0 & 0 & \dots & 0 \\
		0 & J_2 & 0 & \dots & 0 \\
		0 & 0 & J_3 & \dots & 0 \\
		\vdots & \vdots & \vdots & \cdots & \vdots \\
		0 & 0 & 0 & \cdots & J_l
	\end{bmatrix},
\end{equation}
donde $J_i$ es el bloque de Jordan con forma
	\begin{equation}
	J_i = \begin{bmatrix}
\lambda_i & 1 & 0 & \dots & 0 \\
		0 & \lambda_i & 1 & \dots & 0 \\
		0 & 0 & \lambda_i & \dots & 0 \\
		\vdots & \vdots & \vdots & \cdots & \vdots \\
		0 & 0 & 0 & \cdots & \lambda_i
	\end{bmatrix}_{n_i\times n_i},
	\end{equation}
	en donde cada $\lambda_i$ es un autovalor de $A$, y el número $l$ de bloques de Jordan es igual al número total de autovectores independientes de $A$. La matrix $J$ es única (descontando reordenación de filas/columnas) y es llamada la \textbf{forma normal de Jordan} de $A$.
\end{theorem}

Partiendo de la observación que $A = PJP^{-1}$ también, entonces es fácil probar que 
\begin{equation}
	A^k = PJ^kP^{-1},
\end{equation}
de tal manera que podamos calcular que
\begin{align}
	exp(At) &= P\left(\sum_{k=1}^\infty \frac{t^k}{k!} \begin{bmatrix}J_1^k & 0 & \cdots & 0 \\ 0 & J_2^k & \cdots & 0 \\ \vdots & \vdots & \cdots & \vdots \\ 0 & 0 & \cdots & J_l^k \end{bmatrix} \right) P^{-1} \nonumber \\
		&= P \begin{bmatrix}exp(J_1t) & 0 & \cdots & 0 \\ 0 & exp(J_2t) & \cdots & 0 \\ \vdots & \vdots & \cdots & \vdots \\ 0 & 0 & \cdots & exp(J_lt) \end{bmatrix} P^{-1}
\label{eq: expAJordan}
\end{align}

Observa que si $J$ es simplemente una matriz diagonal con los autovalores de $A$, i.e., $J_l = \lambda_l \in \mathbb{C}$, entonces $exp(J_lt) = e^{\lambda_lt} \in\mathbb{C}$ es un cálculo trivial. Es más, si $J$ es diagonal, entonces la solución (\ref{eq: xexp}) puede escribirse como
\begin{equation}
	x(t) = \sum_{i=1}^nc_i e^{\lambda_i t}p_i,
	\label{eq: xtdiag}
\end{equation}
donde $p_i\in\mathbb{R}^n$ es la columna $i$ de $P$ representando un autovector de $A$, y $c_i\in\mathbb{R}$ son coeficientes constantes acorde a la condición inicial $x(0)$.

Ahora, veamos las consecuencias de las siguientes dos suposiciones
\begin{enumerate}
	\item $J$ es diagonal.
	\item Todos los autovalores de $A$ tienen parte real negativa.
\end{enumerate}

Sabiendo que $\lim_{t\to\infty} e^{\lambda t} \to 0$ si $\lambda \in \mathbb{R}_{<0}$, entonces tenemos que $exp(At) \to 0$ según $t\to\infty$ si las dos previas suposiciones se dan. Si echamos un vistazo a (\ref{eq: xexp}) o (\ref{eq: xtdiag}, podemos concluir que 
\begin{equation}
	\lim_{t\to\infty} x(t) \to 0,
	\label{eq: xlim}
\end{equation}
por tanto, podemos predecir la evolución de $x(t)$ con sólamente mirar los autovalores de $A$. Si $J$ no es diagonal, podremos concluir más resultados. Lo veremos en la sección siguiente a la linearización de sistemas en el espacio de estados.


\section{Linearización de sistemas en el espacio de estados}\label{sec: linear}
Desafortunadamente, es realmente dificil (cuando no imposible) calcular una solución analítica para $x(t)$ e $y(t)$ para un sistema arbitrario $\Sigma$ como en (\ref{eq: sigma}). No obstante, hemos visto que sí se puede calcular una solución analítica para $x(t)$ e $y(t)$ cuando $\Sigma$ es un sistema invariante en el tiempo, continuo y lineal en el espacio de estados.

Será de gran utilidad encontrar una relación entre ambos sistemas.

Si $f(x,t)$ y $g(x,t)$ son reales analíticas en un entorno a un punto específico $(x^*,u^*)$, entonces podemos trabajar con aproximaciones de Taylor de $f(x,t)$ y $g(x,t)$ en ese mismo entorno. Cuando nos quedamos en orden uno en la aproximación es lo que se conoce como \emph{linearización}.
\begin{equation}
	\Sigma := \left.\begin{cases}
	\dot x(t) =& f(x(t),u(t)) \\ y(t) =& g(x(t),u(t))
	\end{cases}\right|_{x\approx x^*, u\approx u^*} \approx
	\begin{cases}
		x(t) &= x^* + \delta x(t) \\
		u(t) &= u^* + \delta u(t) \\
	\delta \dot x(t) &= A(t)\delta x(t) + B(t)\delta u(t) \\
	\delta y(t) &= C(t)\delta x(t) + D(t)\delta u(t)
	\end{cases}, \nonumber
\end{equation}
donde
\begin{align}
	A(t) &= \begin{bmatrix}
		\frac{\partial f_1}{\partial x_1} & \dots & \frac{\partial f_1}{\partial x_n} \\
		\vdots & \vdots & \vdots \\
		\frac{\partial f_n}{\partial x_1} & \dots & \frac{\partial f_n}{\partial x_n}
	\end{bmatrix}_{|_{x=x^*, u=u^*}} \quad
	&B(t) = \begin{bmatrix}
		\frac{\partial f_1}{\partial u_1} & \dots & \frac{\partial f_1}{\partial u_k} \\
		\vdots & \vdots & \vdots \\
		\frac{\partial f_k}{\partial u_1} & \dots & \frac{\partial f_k}{\partial u_k}
	\end{bmatrix}_{|_{x=x^*, u=u^*}} \nonumber \\
	C(t) &= \begin{bmatrix}
		\frac{\partial g_1}{\partial x_1} & \dots & \frac{\partial g_1}{\partial x_n} \\
		\vdots & \vdots & \vdots \\
		\frac{\partial g_m}{\partial x_1} & \dots & \frac{\partial g_m}{\partial x_n}
	\end{bmatrix}_{|_{x=x^*, u=u^*}} \quad
	&D(t) = \begin{bmatrix}
		\frac{\partial g_1}{\partial u_1} & \dots & \frac{\partial g_1}{\partial u_k} \\
		\vdots & \vdots & \vdots \\
		\frac{\partial g_m}{\partial u_1} & \dots & \frac{\partial g_m}{\partial u_k}
	\end{bmatrix}_{|_{x=x^*, u=u^*}}. \nonumber
\end{align}
Informalmente, estamos calculando la sensibilidad (hasta primer orden) de $f$ y $g$ cuando hacemos una variación pequeña de $x$ y $u$ alrededor de $(x^*,u^*)$. Como de pequeña ha de ser esa variación depende del sistema $\Sigma$. En particular, cuando diseñemos controladores basados en linearizar alrededor de un punto, daremos cotas para $\delta x$ y $\delta u$ de tal manera que el controlador pueda garantizar estabilidad.

\begin{example}[Linearización del péndulo invertido]
Más adelante, veremos que podemos diseñar una entrada de control $u(t)$, i.e., una señal que ha de seguir el torque $T$ en (\ref{eq: f}) de tal manera que $\theta$ y $\dot\theta$ converjan a unos valores constantes o trayectorias deseadas.

Por ejemplo, vamos a fijar un punto constante de interés $x^* = \begin{bmatrix}\theta^* \\ 0\end{bmatrix}$, por lo que la velocidad angular se marca a cero. Esta situación corresponde a una situación de equilibrio para el ángulo $\theta$. Para hallar el $u^*(t)$ en (\ref{eq: f}) necesario para tal equilibrio necesitamos que $\frac{\mathrm{d}}{\mathrm{dt}}\left(\begin{bmatrix}\theta \\ \dot\theta \end{bmatrix}\right) = \begin{bmatrix}0 \\ 0 \end{bmatrix}$. Una inspección a la dinámica (\ref{eq: dyn}) nos responde que
\begin{equation}
	u^* = T^* = -\frac{g}{l}\sin\theta^*,
\end{equation}
por ejemplo, para una posición totalmente vertical correspondiente a $\theta^* = 0$ tenemos que $T^*=0$, i.e., $x^* = \begin{bmatrix}0\\0\end{bmatrix}$ y $u^* = 0$.

El cáclulo de las matrices $A,B,C,$ y $D$ son los Jacobianos de $(\ref{eq: f})$ y $(\ref{eq: g})$, i.e.,

\begin{align}
\frac{\partial f_1}{\partial x_1} &= 0 \nonumber \\
\frac{\partial f_1}{\partial x_2} &= 1 \nonumber \\
\frac{\partial f_2}{\partial x_1} &= \frac{g}{l}\cos\theta \nonumber \\
\frac{\partial f_2}{\partial x_2} &= -\frac{b}{ml^2} \nonumber \\
\frac{\partial f_1}{\partial u_1} &= 0 \nonumber \\
\frac{\partial f_2}{\partial u_1} &= 1 \nonumber \\
\frac{\partial g_1}{\partial x_1} &= 1 \nonumber \\
\frac{\partial g_1}{\partial x_2} &= 0 \nonumber \\
\frac{\partial g_1}{\partial u_1} &= 0, \nonumber
\end{align}
por lo que podemos llegar a
\begin{align}
	\frac{\mathrm{d}}{\mathrm{dt}}\left(\begin{bmatrix}\delta\theta \\ \dot\delta\theta \end{bmatrix}\right) &= \begin{bmatrix}0 & 1 \\ \frac{g}{l}\cos\theta & -\frac{b}{ml^2} \end{bmatrix}_{|_{\theta=\theta^*}} \begin{bmatrix}\delta\theta \\ \dot\delta\theta \end{bmatrix} + \begin{bmatrix}0 \\ 1 \end{bmatrix} \delta T \nonumber \\
		\delta y &= \begin{bmatrix}1 & 0\end{bmatrix}\begin{bmatrix}\delta\theta \\ \dot\delta\theta \end{bmatrix} + 0 \, \delta T,
\end{align}
para modelar una aproximación a la dinámica de $x(t)$ y la salida $y(t)$ alrededor de los puntos $x^*$ y $u^*$.

\qed
\end{example}
\begin{comment}
\section{(Internal or Lyapunov) Stability}
\label{sec: sta}

Decimos que el sistema lineal (\ref{eq: linsys}) \emph{en el sentido de Lyapunov}
\begin{enumerate}
	\item es \emph{(marginalmente) estable} si para cada condición inicial $x_0$, entonces $x(t) = \Phi(t,t_0) x_0$ está acotada uniformamente para todo $t>t_0$.
	\item es \emph{asintóticamente estable} si además $x(t) \to 0$ según $t\to\infty$.
	\item es \emph{exponencialmente estable} si además $||x(t)|| \leq c e^{\lambda(t-t_0)}||x(t_0)||$ para algunas constantes $c,\lambda > 0$.
	\item is \emph{inestable} si no es marginalmente estable.
\end{enumerate}

%In control, it is very common to focus on \emph{error signals}, e.g., $e(t) := x(t) - x^*(t)$, where $x^*(t)$ is a trajectory goal. Note that if $x^*$ is constant, then $\dot e(t) = \dot x(t)$, and this is why we focus on having $x(t) \to 0$ as $t\to\infty$ in the above definitions for (\ref{eq: sigmalin}).

Centrémonos en sistemas \emph{lti}, es decir, cuando $A$ tiene coeficientes constantes o $\Phi(t,t_0) = e^{A(t-t_0)}$. Entonces, podemos establecer una clara relación entre los autovalores de $A$ y las definiciones de estabilidad en el sentido de Lyapunov únicamente inspeccionando la solución a $\dot x(t) = Ax(t)$ dada por (\ref{eq: solx}).

El sistema $\dot x(t) = Ax(t)$
\begin{enumerate}
	\item es marginalmente estable si y solo si todos los autovalores de $A$ tienen parte real negativa. Si algún autovalor tiene parte real nula, entonces su bloque de Jordan ha de ser $1\times 1$.
	\item es asintóticamente estable si y solo si todos los autovales de $A$ tienen estrictamente parte real negativa.
	\item es exponencialmente estable si es asintóticamente estable.
	\item es inestable si y solo si al menos un autovalor de $A$ tiene parte real positiva, o al menos uno de los autovalores con parte real nula tiene un bloque de Jordan mayor de $1\times 1$.
\end{enumerate}

Comprobando las soluciones (\ref{eq: solx})-(\ref{eq: soly}), podemos decir que si $A$ tiene coeficientes constantes y $\dot x = Ax$ es asintóticamente estable, entonces $x(t) \to \int_{t_0}^t e^{A(t-\tau)}B(\tau)u(\tau)d\tau$ según $t\to\infty$. 

\subsection{Estabilidad local de sistemas linearizados}
Si $x(t)$ es asintóticamente (exponencialmente) estable en $\dot x(t) = Ax(t)$, entonces, existe una única $P$ que satisface la \emph{ecuación de Lyapunov}
\begin{equation}
A^TP + PA = -Q, \quad \forall Q \succ 0.
	\label{eq: lya}
\end{equation}
Uno puede probar (\ref{eq: lya}) si considera
\begin{equation}
	P:= \int_0^\infty e^{A^Tt}Qe^{At}dt.
	\label{eq: P}
\end{equation}
Pista: Primero, sustituye $P$ en (\ref{eq: lya}), y después verifica el cálculo  $\frac{\mathrm{d}}{\mathrm{dt}}\left(e^{A^Tt}Qe^{At}\right)$. Si uno prueba que $P$ es única, entonces $P$ ha de ser positiva definida acorde a su definición (\ref{eq: P}).

Ahora vamos a considerar un sistema continuo, autónomo y no lineal en general
\begin{equation}
	\dot x(t) = f(x(t)), \quad x\in\mathbb{R}^n,
	\label{eq: non}
\end{equation}
con un punto de equilibrio $x^*\in\mathbb{R}^n$, i.e, $f(x^*) = 0$. La dinámica de $x(t)$ puede ser aproximada considerando $x(t) = x^* + \delta x(t)$ donde 
\begin{equation}
	\dot{\delta x(t)} = A\,\delta x(t), \quad A:=\frac{\partial f(x)}{\partial x}.
	\label{eq: delta}
\end{equation}

¿Cómo de buena es esta aproximación?

\begin{theorem}
	\label{thm: tayl}
	Asume que $f(x)$ is dos veces diferenciable. Si (\ref{eq: delta}) es exponencialmente estable, entonces, existe un entorno $\mathcal{B}$ alrededor de $x^*$ y constantes $c, \lambda > 0$ tal que para cada solución $x(t)$ del sistema (\ref{eq: non}) que empiece con $x(t_0)\in\mathcal{B}$, tenemos que
	\begin{equation}
	||x(t) - x^*|| \leq ce^{\lambda(t-t_0)} ||x(t_0) - x^*||, \quad \forall t\geq t_0.
	\end{equation}
\end{theorem}

\subsubsection{¿Cómo de grande es $\mathcal{B}$? ¿Podemos estimarlo? Esbozo de la prueba del Teorema \ref{thm: tayl}}

Como $f$ es dos veces diferenciable, de su desarrollo de Taylor tenemos que
\begin{equation}
	r(x) := f(x) - (f(x^*) + A(x - x^*)) = f(x) - A\,\delta x = O(||\delta x||^2),
\end{equation}
lo cual significa que existe una constante $c$ y una bola $\bar B$ alrededor de $x^*$ tal que 
\begin{equation}
	||r(x)|| \leq c||\delta x||^2, \quad x\in\bar B.
\end{equation}
Si el sistema linearizado es exponencialmente estable, tenemos que
\begin{equation}
A^TP + PA = -I.
\end{equation}
Ahora considera la siguiente señal escalar
\begin{equation}
	v(t) := (\delta x)^T P \delta x, \quad \forall t\geq 0.
\end{equation}
Observa que $\delta x(t) = x(t) - x^*$, entonces $\dot{\delta x(t)} = \dot x(t) = f(t)$. Por lo tanto, la derivada con respecto del tiempo de $v(t)$ satisface
\begin{align}
	\dot v &= f(x)^T P \delta x + (\delta x)^T P f(x) \nonumber \\
	&= (A\delta x + r(x))^T P \delta x + (\delta x)^T P (A\delta x + r(x)) \nonumber \\
	&= (\delta x)^T(A^T P + PA)\delta x + 2(\delta x)^T P r(x) \nonumber \\
	&= -||\delta x||^2 + 2(\delta x)^T P r(x) \nonumber \\
	&\leq -||\delta x||^2 + 2 ||P||\, ||\delta x|| \, ||r(x)||.
\end{align}

Sabemos que $v(t)$ es positiva excepto cuando $\delta x = 0$. Si podemos garantizar que $\dot v(t) < 0$ y que $\dot v(t) = 0$ solo cuando  $\delta x = 0$, entonces $v(t) \to 0$ as $t\to\infty$, lo cual implica que  $\delta x(t) \to 0$ as $t\to\infty$.

Ahora, si $x\in\mathcal{\bar B}$, entonces

\begin{equation}
	\dot v \leq -\Big(1 - 2c\,||P||\,||\delta x||\Big)||\delta x||^2,
\end{equation}
Por lo tanto, si la desviación  $\delta x$ es suficientemente pequeña, i.e., 
\begin{equation}
||\delta x|| < \frac{1}{2c||P||},
\end{equation}
entonces  $\dot v(t) < 0$ si $\delta x(0) \neq 0$ y $\delta x(0) < \frac{1}{2c||P||}$.

Podemos concluir que una estimación de $\mathcal B$ es
\begin{equation}
	\mathcal{B} := \{ \delta x : ||\delta x|| < \frac{1}{2c||P||} \}.
	\label{eq: Bregion}
\end{equation}

\section{Controlabilidad}
\subsection{Subespacios alcanzables y controlables}
Recordemos que cueando aplicamos una entrada genérica $u(\cdot)$ a (\ref{eq: linsys}), transferimos el sistema de un estado $x(t_0):=x_0$ a un estado $x(t_1):=x_1$, que además podemos calcular con la expresión
\begin{equation}
	x_1 = \Phi(t_1,t_0)x_0 + \int_{t_0}^{t_1} \Phi(t_1,\tau)B(\tau)u(\tau)d\tau,
\end{equation}
donde recordemos que $\Phi(\cdot)$ es la matriz de transición de estados del sistema.

Preguntas:
\begin{enumerate}
	\item ¿Qué estados puedo alcanzar desde $x_0$?
	\item ¿Existe siempre una entrada $u(\cdot)$ que transfiera un estado arbitrario $x_0$ a otro $x_1$?
\end{enumerate}

Estas dos preguntas llevan a acuñar las definiciones de (sub)espacios alcanzables y controlables.

\begin{definition}[Subespacio alcanzable]
	Dados dos instantes de tiempo $t_1>t_0\geq 0$, el subespacio alcanzable (o controlable desde el origen) $\mathcal{R}[t_0,t_1]$ consiste en todos los estados $x_1$ por los que existe una entrada $u:[t_0,t_1]\to \mathbb{R}^k$ que transfiere el estado $x_0 = 0$ a  $x_1 \in\mathbb{R}^n$; i.e.,
	\begin{equation}
		\mathcal{R}[t_0,t_1] := \Big\{x_1\in\mathbb{R}^n : \exists u(\cdot),\, x_1 = \int_{t_0}^{t_1} \Phi(t_1,\tau)B(\tau)u(\tau)d\tau \Big\}. \label{eq: rs}
	\end{equation}
\end{definition}

\begin{definition}[Subespacio controlable]
	Dados dos instantes de tiempo $t_1>t_0\geq 0$, el subespacio controlable (or controlable hacia el origen) $\mathcal{C}[t_0,t_1]$ consiste en todos los estados $x_0$ por los que existe una entrada $u:[t_0,t_1]\to \mathbb{R}^k$ que transfiera un estado $x_0\in\mathbb{R}^n$ a $x_1 = 0$; i.e.,
	\begin{equation}
		\mathcal{C}[t_0,t_1] := \Big\{x_0\in\mathbb{R}^n : \exists u(\cdot),\, 0 = \Phi(t_1,t_0)x_0 + \int_{t_0}^{t_1} \Phi(t_1,\tau)B(\tau)u(\tau)d\tau \Big\}.
	\end{equation}
\end{definition}

¿Cómo podemos calcular $\mathcal{R}[t_0,t_1]$ y $\mathcal{C}[t_0,t_1]$? Para ello vamos a introducir y explotar las siguientes dos matrices llamadas \emph{Gramianos}.

\begin{definition}[Gramianos de alcanzabilidad y controlabilidad]
	\begin{align}
		W_R(t_0,t_1) &:= \int_{t_0}^{t_1} \Phi(t_1,\tau)B(\tau)B(\tau)^T\Phi(t_1,\tau)^Td\tau \\
		W_C(t_0,t_1) &:= \int_{t_0}^{t_1} \Phi(t_0,\tau)B(\tau)B(\tau)^T\Phi(t_0,\tau)^Td\tau 
	\end{align}
\end{definition}

\begin{theorem}
Dados dos instantes de tiempo $t_1 > t_0 \geq 0$,
	\begin{align}
		\mathcal{R}[t_0,t_1] &= \text{Im}\{W_R(t_0,t_1)\} \label{RI} \\
		\mathcal{C}[t_0,t_1] &= \text{Im}\{W_C(t_0,t_1)\} \label{CI},
	\end{align}
	donde $\text{Im}\{A\}:= \Big\{y\in\mathbb{R}^m: \exists x\in\mathbb{R}^n, y = Ax\Big\}$ para una matriz $A\in\mathbb{R}^{m\times n}$.
\end{theorem}
\begin{proof}
	Solo vamos a probar (\ref{RI}) porque (\ref{CI}) tiene una prueba similar.
	Necesitamos mostrar ambas implicaciones: primero, si $x_1 \in \text{Im}\{W_R(t_0,t_1)$, entonces $x_1 \in \mathcal{R}[t_0,t_1]$; segundo, si $x_1 \in \mathcal{R}[t_0,t_1]$ entonces $x_1 \in \text{Im}\{W_R(t_0,t_1)$.\\
	Cuando  $x_1 \in \text{Im}\{W_R(t_0,t_1)$, existe un vector $\mu_1\in\mathbb{R}^n$ tal que
	\begin{equation}
	x_1 = W_R(t_0,t_1)\eta_1.
	\end{equation}
	Escoge $u(\tau) = B(\tau)^T\Phi(t_1, \tau)^T\eta_1$, y sustitúyelo en (\ref{eq: rs}), entonces tenemos que 
	\begin{equation}
	x_1 = \int_{t_0}^{t_1} \Phi(t_1,\tau)B(\tau) B(\tau)^T\Phi(t_1, \tau)^T \eta_1d\tau = W_R(t_0,t_1)\eta_1.
	\end{equation}
	Cuando $x_1 \in \mathcal{R}[t_0,t_1]$, existe una entrada $u(\cdot)$ para la cual 
	\begin{equation}
		x_1 = \int_{t_0}^{t_1} \Phi(t_1,\tau)B(\tau)u(\tau)d\tau.
		\label{x1}
	\end{equation}
	Si (\ref{x1}) es en $\text{Im}\{W_R(t_0,t_1)\}$, entonces $x_1^T\mu = 0, \, \mu \in \text{Ker}\{W_R(t_0,t_1)\}$\footnote{If $x\in\text{Ker}\{A^T\}$, por lo que $A^Tx = 0$. Si $y\in\text{Im}\{A\}$, entonces  $y = A\eta$. Por lo tanto, $x^Ty = x^TA\eta = \eta^TA^Tx = \eta \cdot 0 = 0$. Observa que $W_R^T = W_R$ por definción.} Vamos a calcular 
	\begin{equation}
		x_1^T\mu = \int_{t_0}^{t_1}u(\tau)^TB(\tau)^T\Phi(t_1,\tau)^T\mu \, d\tau. \label{eq: x1eta1}
	\end{equation}
	Y observando que 
	\begin{align}
	\mu \in \text{Ker}\{W_R(t_0,t_1)\} \implies \mu^TW_R(t_0,t_1)\mu &= 0 \nonumber \\ &= \int_{t_0}^{t_1}\mu^T \Phi(t_1,\tau)B(\tau)B(\tau)^T\Phi(t_1,\tau)^T \mu \, d\tau \nonumber \\ &= \int_{t_0}^{t_1} ||B(\tau)^T\Phi(t_1,\tau)^T \mu||^2 d \tau,
	\end{align}
	obtenemos que $B(\tau)^T\Phi(t_1,\tau)^T \mu  = 0$, llevando a (\ref{eq: x1eta1}) ser igual a cero.
\end{proof}
\begin{remark}
	Observa que hemos probado que $u(\tau) = B(\tau)^T\Phi(t_1,\tau)^T \eta_1$ puede ser como una entrada de control para transferir $x_0 = 0$ a $x_1\in\mathbb{R}^n$ en un tiempo finito $(t_1 - t_0)$. De hecho, esta es la señal de control \emph{en lazo abierto de mínima energía}.
\end{remark}
Vamos a ver este hecho en más detalle. Considera otra señal de control  $\bar u(t)$ tal que 
\begin{equation}
x_1 = \int_{t_0}^{t_1} \Phi(t_1,\tau)B(\tau)u(\tau)d\tau = \int_{t_0}^{t_1} \Phi(t_1,\tau)B(\tau)\bar u(\tau)d\tau.
\end{equation}
Para que sea cierto, debemos tener
For this to hold, we must have
\begin{equation}
	\int_{t_0}^{t_1} \Phi(t_1,\tau)B(\tau)v(\tau) \, d\tau = 0,
	\label{eq: ve}
\end{equation}
donde $v(\tau) = u(\tau) - \bar u(\tau)$.
Vamos a ver la energía asociada a $\bar u$
\begin{align}
\int_{t_0}^{t_1}||\bar u(\tau)||^2 d\tau &= \int_{t_0}^{t_1} || B(\tau)^T\Phi(t_1,\tau)^T \eta_1 + v(\tau)||^2 d\tau \nonumber \\ 
&= \eta_1^T W_R(t_0,t_1)\eta_1 + \int_{t_0}^{t_1} ||v(\tau)||^2 d \tau + 2\eta_1^T \int_{t_0}^{t_1}B(\tau)\Phi(t_1,\tau)v(\tau) d \tau,
\end{align}
donde el tercer término es cero por (\ref{eq: ve}). Por lo tanto, si $\bar u(t)$ difiere $v(t)$ de $u(t)$, gastará $\int_{t_0}^{t_1} ||v(\tau)||^2 d\tau$ más energía que $u(t)$.

\subsection{Matriz de controlabilidad para un sistema lti}

Consideremos un sistema lineal (\ref{eq: linsys}) con $A$ y $B$ con coeficientes constantes.

El teorema de Cayley-Hamilton nos permite escribir
\begin{equation}
	e^{At} = \sum_{i=0}^{n-1}\alpha_i(t)A^i, \quad \forall t\in\mathbb{R},
\end{equation}
para algunas funciones escalares apropiadas $\alpha_i(t)$. Tenemos también que si $A$ y $B$ tienen coeficientes constantes, entonces
\begin{align}
	x_1 &= \int_{0}^{t_0-t_1} e^{At}Bu(t) dt \nonumber \\
	&= \sum_{i=0}^{n-1} A^iB \Big(\int_{0}^{t_1-t_0}\alpha_i(t)u(t)dt \Big) \nonumber \\
	&= \mathcal{C} \begin{bmatrix}\int_{0}^{t_1-t_0}\alpha_0(t)u(t)dt \\ \vdots \\ \int_{0}^{t_1-t_0} \alpha_{n-1}(t)u(t)dt \end{bmatrix},
\end{align}
donde
\begin{equation}
\mathcal{C}:=\begin{bmatrix}B & AB & A^2B & \cdots & A^{n-1}B
\end{bmatrix}_{n\times (kn)},
	\label{eq: conmat}
\end{equation}
es la llamada matrix de controlabilidad del sistema lti.
\begin{remark}Observa que $\mathcal{C}[t_0,t_1]$ y $\mathcal{C}$  son diferentes objetos. El primero es un (sub)espacio, mientras que el segundo es una matriz.
\end{remark}

Observa que acabamos de probar que la imagen de $\mathcal{C}$ es la misma que la imagen de $W_R(t_0,t_1)$. La siguiente afirmación más general puede probarse también:
\begin{theorem}
	\label{thm: spcon}
Dados dos instantes de tiempo $t_0, t_1$, con $t_1 > t_0$, tenemos que
	\begin{equation}
		\mathcal{R}[t_0,t_1] = \text{Im}\{W_R(t_0,t_1)\} = \text{Im}\{\mathcal{C}\} = \text{Im}\{W_C(t_0,t_1)\} = \mathcal{C}[t_0,t_1],
	\end{equation}
\end{theorem}

Podemos extraer dos importantes consecuencias del Teorema \ref{thm: spcon} para sistemas lti. ¡Observa que $\text{Im}\{\mathcal{C}\}$ no depende de ninguna variable tiempo!

\begin{enumerate}
	\item \emph{Reversibilidad temporal}: Si uno puede alcanzar $x_1$ desde el origen, entonces uno puede alcanzar el origen desde $x_1$. Es decir, los subespacios de alcanzabilidad y controlabilidad son el mismo subespacio.
	\item \emph{Escalado de tiempo}: La alcanzabilidad y la controlabilidad no dependen del tiempo. Si uno puede transferir el estado de $x_0$ and $x_1$ en $t$ segundos, entonces también puedes hacerlo en $\bar t \neq t$ segundos.
\end{enumerate}

\begin{definition}
	Dados dos instantes de tiempo $t_1 > t_0 \geq 0$, el par $(A,B)$ del sistema (\ref{eq: linsys}) se dice \emph{alcanzable} en $[t_0, t_1]$ si $\mathcal{R}[t_0,t_1] = \mathbb{R}^n$. Es decir, si podemos alcanzar cualquier estado en tiempo finito partiendo desde el origen.
\end{definition}

\begin{definition}
	\label{def: con}
	Dados dos instantes de tiempo $t_1 > t_0$, el par del sistema (\ref{eq: linsys}) se dice \emph{controlable} en $[t_0, t_1]$ si $\mathcal{C}[t_0,t_1] = \mathbb{R}^n$. Es decir, si podemos alcanzar el origen partiendo desde cualquier estado en tiempo finito.
\end{definition}

\subsection{Tests de controlabilidad}
El siguiente teorema es el resultado de combinar la Definición \ref{def: con} con el Teorema \ref{thm: spcon}:
\begin{theorem}
	El par (constante) $(A,B)$ es controlable si y solo si el rango de $\mathcal{C}$ es $n$.
\end{theorem}

El siguiente teorema se puede comprobar numéricamente a partir de los siguientes resultados.
\begin{theorem}
	\label{thm: atbt}
	El par (constante) $(A,B)$ es controlable si y solo si no existe ningún autovector de $A^T$ en el kernel de $B^T$.
\end{theorem}
El siguiente teorema es una reescritura del anterior.
\begin{theorem}
El par (constante) $(A,B)$ es controlable si y solo si el dango de $\begin{bmatrix}A-\lambda I & B\end{bmatrix}$ es $n$.
\end{theorem}

\begin{remark}
	Observa que el tener un sistema asintóticamente estable no implica el tener 
	un sistema lti controlable. Por ejemplo, 
\begin{equation}
	\dot x = \begin{bmatrix}-3 & 0 \\ 0 & -7\end{bmatrix}x + \begin{bmatrix}0 \\ 1\end{bmatrix}u, \quad x\in\mathbb{R}^2, u\in\mathbb{R}.
\end{equation}
	El kernel de $B^T$ es generado por $\begin{bmatrix}1 \\ 0\end{bmatrix}$, el cual es proporcional a un autovector de $A = A^T$. Entonces, el sistema no es controlable acorde al teorema \ref{thm: atbt}. Observa, que no tenemos autoridad ninguna sobre $x_1$ a través de $u$. Sin embargo, es asintóticamente estable. Uno puede ver que $x_{\{1,2\}}(t) \to 0$ según $t\to\infty$ cuando $u = 0$. Recuerda que la controlabilidad trata de transferir estados en \textbf{tiempo finito}.
\end{remark}

\section{Estabilización de un sistema lti por realimentación de estados}
\label{sec: reak}
\subsection{Test de Lyapunov para la estabilización de un sistema lti}
\begin{definition}
	Si el sistema (\ref{eq: linsys}) es lti, entonces es estabilizable si existe una entrada $u(t)$ para cualquier $x(0)$ tal que $x(t)\to 0$ as $t\to\infty$.
\end{definition}
Esta definición es una versión de \emph{sistema controlable} pero para tiempo infinito. En el siguiente teorema veremos que \emph{estabilizable} es menos restrictivo que \emph{controlable}.

\begin{theorem}
	Si el sistema (\ref{eq: linsys}) es lti, entonces es estabilizable si y solo si todos los autovectores de $A^T$ correspondientes a autovalores con parte real no negativa pertenecen al kernel de $B^T$.
\end{theorem}

La proyección de $x(t)$ sobre el espacio generado por los autovectores de $A^T$ asociados a autovalores con parte real negativa van a cero sin necesidad de la \emph{asistencia} de ninguna entrada. Entonces, la entrada $u(t)$ debe asistir a las proyecciones de $x(t)$ en el resto de autovectores de $A^T$. Por ejemplo,

\begin{equation}
	\dot x = \begin{bmatrix}-3 & 0 \\ 0 & 7\end{bmatrix}x + \begin{bmatrix}0 \\ 1\end{bmatrix}u, \quad x\in\mathbb{R}^2, u\in\mathbb{R},
\end{equation}
mientras que no tenemos ninguna \emph{autoridad} sobre $x_1(t)$, podemos emplear $u(t)$ para lleva a $x_2(t)$ a cero de tal manera que $x(t)\to 0$ según $t\to\infty$.
\begin{theorem}
	Si el sistema (\ref{eq: linsys}) es lti, entonces es estabilizable si y solo si hay una matriz positiva definida $P$ a la siguiente desigualdad
	\begin{equation}
	AP + PA^T - BB^T \prec 0
		\label{eq: lb}
	\end{equation}
\end{theorem}
\begin{proof}
	Solo vamos a ver una dirección de la prueba. No confundir (\ref{eq: lb}) con la ecuación de Lyapunov\footnote{Observa el orden de las matrices traspuestas, y observa también el signo opuesto de $-BB^T$ y $+I$ en las dos ecuaciones.} $PA+A^TP \prec -I$ en (\ref{eq: lya}).

Conisdera $x$ como un autovector asociado al autovalor $\lambda$ de $A^T$ con parte real no negativa. Entonces,
	\begin{equation}
	x^*(AP+PA^T)x < x^*BB^Tx = ||B^Tx||^2,
		\label{eq: aux}
	\end{equation}
	donde $x^*$ es el complejo conjugado de $x$. Pero el lado izquierdo de (\ref{eq: aux}) es igual a 
	\begin{equation}
		(A^T(x^*)^T)^TPx + x^*PA^Tx = \lambda^*x^*Px + \lambda x^*Px = 2\text{Real}\{\lambda\}x^*Px.
	\end{equation}
	Como $P$ es positiva definida, y $\text{Real}\{\lambda\} \geq 0$, podemos concluir que 
\begin{equation}
0 \leq 2\text{Real}\{\lambda\}x^*Px < ||B^Tx||^2,
\end{equation}
y por tanto $x$ no debe pertenecer al kernel de $B$, ya que si tendríamos que 
\begin{equation}
0 \leq 2\text{Real}\{\lambda\}x^*Px < 0,
\end{equation}
y eso no es posible.
\end{proof}

\subsection{Controlador por realimentación de estados}
Realimentación de estados implica el escoger una señal de entrada que dependa únicamente de los estados del sistema, es decir, diseñar una
\begin{equation}
u(t) = -Kx(t),
	\label{eq: kx}
\end{equation}
donde $K\in\mathbb{R}^{n\times k}$, tal que $x(t) \to 0$ en (\ref{eq: linsys}) exponencialmente rápido según $t\to\infty$.

Define la ganancia matriz de control
\begin{equation}
	K:=\frac{1}{2}B^TP^{-1}, \label{eq: K}
\end{equation}
donde $P$ se calcula en (\ref{eq: lb}) si y solo si un sistema lti es estabilizable. Por lo tanto, podemos rescribir (\ref{eq: lb}) como
\begin{equation}
	(A - \frac{1}{2}BB^TP^{-1})P + P(A - \frac{1}{2}BB^TP^{-1})^T = (A-BK)P+P(A-BK)^T \prec 0,
\end{equation}
y multiplicando a la izquierda y derecha por $Q:=P^{-1}$ tenemos que
\begin{equation}
	Q(A-BK)+(A-BK)^TQ \prec 0,
\end{equation}
la cual es una ecuación de Lyapunov. Por lo tanto, podemos concluir que $(A-BK)$ tiene todos sus autovalores \emph{estables}, esto es, si uno escoge la entrada
\begin{equation}
	u = -Kx = -\frac{1}{2}B^TP^{-1}x, \label{eq: conK}
\end{equation}
en el sistema lti estabilizable, entonces $x(t) \to 0$ exponencialmente rápido según $t\to\infty$.

Encontrar la matrix $P$ en (\ref{eq: K}) requiere resolver la \emph{linear matrix inequality} (LMI) en (\ref{eq: lb}). Para ello existen herramientas numéricas \emph{LMI solvers} disponibles en Matlab o Python.

Si el sistema lti es controlable (recordemos que es más restrictivo que estabilizable), la matriz $K$ en (\ref{eq: kx}) puede explotar el Teorema \ref{thm: atbt}. Es decir, podemos encontrar una matriz $K$ tal que $(A-BK)$ tenga sus autovalores donde nosotros queramos por diseño. Por ejemplo, a través del Teorema de Ackermann.
\begin{theorem}
	\label{thm: ack}
	Si el sistema lti es controlable, entonces $(A-BK)$ tiene como autovalores un conjunto deseado $\Lambda$ si
	\begin{equation}
		K = \begin{bmatrix}0 & \dots & 0 & 0 & 1\end{bmatrix}\mathcal{C}^{-1} \Delta(A),
	\end{equation}
	en donde $\mathcal{C}$ es la matrix de controlabilidad (\ref{eq: conmat}) y $\Delta$ es el polinomio característico que satisface $\Lambda$.
\end{theorem}
Obviamente, al requerir la inversa de la matriz de contolabilidad, se requiere por tanto que $\mathcal{C}$ sea de rango máximo. Es decir, el sistema lti ha de ser controlable para poder aplicar el Teorema \ref{thm: ack}.


\section{Observabilidad en sistemas lti}
\subsection{Subespacio inobservable y el Gramiano de observabilidad}
\begin{definition}
	El subespacio inobservable $\mathcal{UO}$ de un sistema lti consiste en todos aquellos estados que satisfacen
	\begin{equation}
	C e^{A} x_0 = 0.
		\label{eq: ce}
	\end{equation}
\end{definition}
Esta definición está motivada por los siguientes hechos.
Recuerda que en (\ref{eq: soly}) podemos derivar que
\begin{align}
	y(t) &= Ce^{At}x_0 + \int_0^tCe^{A(t-\tau)}Bu(\tau)d\tau + Du(t) \nonumber \\
	\tilde y(t) &:= y(t) - \int_0^tCe^{A(t-\tau)}Bu(\tau)d\tau - Du(t) = Ce^{At}x_0. \label{eq: y}
\end{align}
En el lado izquierdo de (\ref{eq: y}) tenemos un par entrada/salida, y en el lado derecho de (\ref{eq: y}) tenemos el estado inicial $x_0$. De (\ref{eq: y}) podemos observar dos propiedades interesantes:
\begin{enumerate}
	\item Cuando un particular $x_0$ es compatible con un par entrada/salida, entonces todo estado inicial de la forma $x_0 + x_u, \, x_u\in\mathcal{UO}$ también es compatible con la misma entrada/salida.
	\item Cuando $\mathcal{UO}$ contiene únicamente el vector cero, entonces existe al menos un estado inicial $x_0$ compatible con un par entrada/salida.
\end{enumerate}

\begin{definition}
	Un sistema lti es observable si su $\mathcal{UO}$ contiene solo el vector cero.
\end{definition}

\begin{definition}
	Dados dos instantes de tiempo $t_1>t_0\geq 0$, el Gramiano de observabilidad viene definido por
	\begin{equation}
		W_O(t_0,t_1) := \int_{t_0}^{t_1}e^{A^T(\tau - t_0)} C^T Ce^{A(\tau - t_0)}d\tau
	\end{equation}
\end{definition}
Partiendo de (\ref{eq: ce}), se puede llegar a
\begin{equation}
	\operatorname{Ker}W_O(t_0,t_1) = \mathcal{UO}.
\end{equation}

\subsection{Tests de observabilidad}
Lo siguiente se puede demostrar formalmente apoyándonos en el teorema de Cayley-Hamilton. Por ejemplo, para ver por qué paramos en $(n-1)$. Vamos a ver la sensibilidad con respecto de los estados $x$ de $y$ y de sus derivadas con respecto del tiempo \footnote{Observa como $B$ y $D$ no juegan ningún papel aquí} cuando $u(t) = 0$.
\begin{align}
	y(t) = Cx(t) &\implies \dot y(t) = C\dot x(t) = CAx(t) \implies \ddot y(t) = C A^2 x(t) \quad \dots \nonumber \\ &\implies \frac{\mathrm{d}^{n-1}y}{\mathrm{dt}^{n-1}} = CA^{n-1}x(t), \nonumber
\end{align}
Si no queremos perder ninguna información sobre la señal $x(t)$, entonces la matriz
\begin{equation}
	\mathcal{O} = \begin{bmatrix}C \\ CA \\ \vdots \\ CA^{n-1}\end{bmatrix}_{(kn)\times n}, \label{eq: O}
\end{equation}
debe de ser de rango máximo en sus columnas. Vamos a introducir los siguientes resultados equivalentes
\begin{theorem}
	Un sistema lti es observable si y solo si el rango de $\mathcal{O}$ es igual a $n$.
\end{theorem}
\begin{theorem}
	Un sistema lti es observable si y solo si no hay ningún autovector de $A$ en el kernel de $C$.
\end{theorem}

Fíjate que de (\ref{eq: O}) podemos derivar un test de controlabilidad
\begin{equation}
	\mathcal{O}^T = \begin{bmatrix}C^T & A^TC^T & \cdots & (A^{n-1})^TC^T \end{bmatrix}_{n \times (kn)}, 
\end{equation}
que vendría dado por el siguiente sistema lti \emph{dual}
\begin{equation}
	\Sigma_{\text{dual}} := \begin{cases}
		\dot{\bar x}(t) &= A^T \bar x(t) + C^T \bar u(t) \\
		\bar y(t) &= B^T\bar x(t) + D^T\bar u(t)
	\end{cases}.
\label{eq: sigmadual}
\end{equation}
Por lo tanto, podemos formular el siguiente resultado
\begin{theorem}
	Un sistema lti es observable si y solo si su sistema dual (\ref{eq: sigmadual}) es controlable.
\end{theorem}
No haría falta ningún test nuevo para concluir la observabilidad de un sistema lti. Simplemente, construimos su sistema dual y estudiamos su controlabilidad.

\section{Estimación de estados en sistemas lti}
El estimador más simple consiste en hacer una copia de la dinámica del sistema lti
\begin{equation}
	\Sigma_{\text{estimator}} := \begin{cases}
		\dot{\hat x}(t) &= A \hat x(t) + B u(t) \\
		\hat y(t) &= C\hat x(t) + D u(t)
	\end{cases},
\label{eq: sigmaest}
\end{equation}
en donde $\hat x\in\mathbb{R}^nm y\in\mathbb{R}^m$ será nuestra estimación de los estados y de la salida del sistema lti respectivamente. Ahora, definamos la señal de error
\begin{equation}
e(t) := \hat x(t) - x(t),
\end{equation}
por lo tanto, cuando el error $e$ sea cero, querrá decir que estamos estimando los estados del sistema lti correctamente. Vamos a ver, la dinámica de la señal de error
\begin{equation}
	\dot e(t) = \dot{\hat x}(t) - \dot x(t) = A\hat x + Bu - Ax - Bu = Ae(t),
\end{equation}
por lo que $e(t)\to 0$ según $t\to\infty$ exponencialmente rápido si $A$ es una matriz \emph{estable} para toda entrada $u(t)$.

¿Y si $A$ no es una matriz estable? Entonces considera el siguiente estimador
\begin{equation}
	\Sigma_{\text{estimator2}} := \begin{cases}
		\dot{\hat x}(t) &= A \hat x(t) + B u(t) - L(\hat y(t) - y(t)) \\
		\hat y(t) &= C\hat x(t) + D u(t)
	\end{cases},
\label{eq: sigmaest2}
\end{equation}
en donde tenemos dos entradas $u$ y $(\hat y - y)$ a la dinámica de $\hat x$, y $L\in\mathbb{R}^{n\times m}$ es una matriz de ganancias. Observa que la señal $y$ viene del sistema lti original, y es algo que podemos medir al ser su salida. Ahora, veamos la nueva dinámica para la señal de error $e(t)$.
\begin{equation}
	\dot e = A\hat x + Bu - L(\hat y - y) - (Ax + Bu) = (A-LC)e,\label{eq: ed} 
\end{equation}
por lo tanto, si $(A-LC)$ es una matriz de estabilidad, entonces $e(t)\to 0$ según $t\to\infty$ exponencialmente rápido para cualquier señal $u(t)$.

Para el cálculo de $L$ bastaría con calcular $K$ para el sistema dual utilizando los resultados de la sección \ref{sec: reak}. En particular $L = K^T_{\text{dual}}$, y $K = L^T_{\text{dual}}$.

A continuación, los resultados \emph{duales} para observabilidad.
\begin{theorem}
	Un sistema lti es \emph{detectable} si y solo si los autovectores de $A$ correspondientes a autovalores inestables no están en el kernel de $C$.
\end{theorem}
Si un sistema es detectable, entonces su dual es estabilizable, y viceversa.
\begin{theorem}
	Cuando un par $(A,C)$ es detectable, entonces es siempre posible encontrar una matriz $L$ tal que $(A-LC)$ es una matriz de estabilidad.
\end{theorem}
\begin{theorem}
	Cuando un par $(A,C)$ es observable, entonces es siempre posible encontrar una matrix $L$ tal que los autovalores de $(A-LC)$ puedan estar donde queramos.
\end{theorem}

\section{Estabilización de sistemas lti con realimentación a través de su salida}

Si $C$ no es invertible, entonces no tenemos un cálculo directo de los estados $x$ de un sistema lti; por lo tanto no podemos aplicar la señal de control $u = -Kx$ como en la sección \ref{sec: realk}.

¿Qué ocure si el sistema lti es observable, o al menos detectable? Entonces, vamos a mostrar que podemos utilizar como controlador la señal
\begin{equation}
	u = -K \hat x, \label{eq: uh}
\end{equation}
donde $\hat x$ son los estados de nuestro estimador (\ref{eq: sigmaest2}). Vamos a aplicar (\ref{eq: uh}) a un sistema lti, por lo que la dinámica será
\begin{equation}
	\dot x = Ax - BK\hat x = Ax - BK(e + x) = (A -BK)x -BKe,
\end{equation}
que junto con (\ref{eq: ed}) nos lleva al siguiente sistema autónomo
\begin{equation}
	\begin{bmatrix}\dot x \\ \dot e\end{bmatrix} = \begin{bmatrix}A-BK & -BK \\ 0 & A-LC\end{bmatrix}\begin{bmatrix}x \\ e\end{bmatrix},
\end{equation}
el cual es un sistema triangular, y será exponencialmente estable porque $(A-BK)$ y $(A-LC)$ son matrices diseñadas para tener todos sus autovalores estables.

\section{Regulador cuadrático lineal o LQR}
Si el par $(A,B)$ es controlable, entonces hemos visto que podemos colocar los autovalores de $(A-BK)$ donde queramos. Entonces uno podría preguntarse ¿cuál es el mejor lugar para los autovalores?

El \emph{mejor} lugar responde al siguiente criterio de diseño. Considera que tenemos la siguiente salida de interés
\begin{equation}
	z(t) = G x(t) + H u(t) \label{eq: z}.
\end{equation}
Por supuesto $z(t)\in\mathbb{R}^l$ puede ser la señal de salida $y(t)$, pero no tiene por qué. En particular, debemos distinguir que $y(t)$ la usaremos para alimentar el estimador/controlador, y $z(t)$ la usaremos para optimizar nuestro criterio sobre \emph{el mejor} lugar para los autovalores de $(A-BK)$.

El problema LQR está definido de la siguiente manera:
\begin{problem}
	Encontrar una entrada $u(t), t\in[0,\infty)$ que minimice la siguiente función de coste
\begin{equation}
	J := \int_0^\infty z(t)^T \bar Q z(t) + \rho \, u(t)^T \bar R u(t) dt,
	\label{eq: J}
\end{equation}
	donde $\bar Q\in\mathbb{R}^{l\times l}$ y $\bar R\in\mathbb{R}^{m\times m}$ son matrices positivas definidas, y $\rho$ es una constante positiva para relativizar ambos términos en (\ref{eq: J}).
\end{problem}

Como $z = Gx + Hu$, entonces $J$ puede ser rescrito como
\begin{equation}
	J := \int_0^\infty x(t)^T Q x(t) + \, u(t)^T R u(t) + 2x(t)^T N u(t)dt,
	\label{eq: J2},
\end{equation}
donde $Q = G^T\bar Q G$, $R = H^T\bar QH + \rho \bar R$, y $N = G^T\bar Q H$.

Observa como $\bar Q$, $\bar R$ y $\rho$ determinan como de importante es el minimizar los elementos de $z$ y $u$.

Si somos capaces de encontrar una matriz positiva definida $P$ tal que satisface la \emph{ecuación algebraica de Ricatti}
\begin{equation}
	A^TP + PA + Q - (PB + N)R^{-1}(B^TP + N^T) = 0,
\end{equation}
entonces
\begin{theorem}
	Si $A - BR^{-1}(B^TP+N^T)$ es una matriz de estabilidad. Entonces, la señal de entrada
	\begin{equation}
	u(t) = -R^{-1}(B^TP+N^T) x(t),
	\end{equation}
	minimiza $J$, es más, $J = x(0)^T P x(0)$.
\end{theorem}

La \emph{regla de Bryson} nos orienta para unos valores razonables de $\bar Q$ y $\bar R$:
\begin{align}
	\bar Q_{ii} &= \frac{1}{\text{valor máximo aceptable de}\, z_i^2} \nonumber \\
	\bar R_{jj} &= \frac{1}{\text{valor máximo aceptable de}\, u_j^2}. \nonumber
\end{align}


\section{Resumen para la estabilización de un punto en un sistema no lineal}
Dado el siguiente sistema no lineal
\begin{equation}
	\Sigma :=
\begin{cases}
	\dot x &= f(x,u) \\
	y &= g(x,u)
\end{cases}, \label{eq: sisnl}
\end{equation}
donde $x\in\mathbb{R}^n$ es el estado del sistema, $u\in\mathbb{R}^k$ es la entrada al sistema, $y\in\mathbb{R}^m$ es la salida del sistema, y $f(x,u): \mathbb{R}^n\times\mathbb{R}^k \to \mathbb{R}^n$ y  $g(x,u): \mathbb{R}^n\times \times\mathbb{R}^k \to \mathbb{R}^m$ son funciones arbitrarias.

El siguiente algoritmo nos permite estabilizar un punto arbitrario $x^*$ del sistema (\ref{eq: sisnl}).

\begin{enumerate}
	\item Escoge un punto de interés $x^*$, y entonces calcula $u^*$ de tal manera que  $f(x^*,u^*) = 0$.
	\item Lineariza (\ref{eq: sisnl}) alrededor de  $x^*$ y $u^*$, i.e., calcula sus Jacobianos $A:=\frac{\partial f(x,u)}{\partial x}$ y $B:=\frac{\partial f(x,u)}{\partial u}$ y evalúalos en $x=x^*$ y $u=u^*$. Observa, que si un Jacobiano no existe, e.g., $f(x,u)$ no es real analítica, entonces, no podemos continuar.
	\item Comprobar si $(A,B)$ es controlable. Si este test falla, entonces comprueba si es almenos estabilizable. Si ambos tests fallan, no podemos continuar.
	\item Calcular el Jacobiano $C:=\frac{\partial g(x,u)}{\partial x}$. Si $C$ no es invertible, entonces necesitaremos un observador/estimador. Si $C$ es invertible, entonces ves al paso \ref{step}.
	\item Si necesitamos un estimador, entonces comprueba que $(A,C)$ es observable. Si no, comprueba que $(A,C)$ es detectable. Si ambos tests fallan, no podemos continuar.
	\item Si $(A,C)$ es observable, entonces contruye el estimador (\ref{eq: sigmaest2}), con $D:=\frac{\partial g(x,u)}{\partial u}$, y calcula $L$ de tal manera que $(A-LC)$ sea una matriz estable. Por ejemplo, con el Teorema \ref{thm: ack}.
	\item Si $(A,C)$ no es observable pero sí detectable, entonces, calcula $L$ a través del sistema dual con el test de Lyapunov para la estabilización como en (\ref{eq: K}). Recuerda, que entonces $L = K^T$.
	\item \label{step} Si el par $(A,B)$ es controlable, entonces podemos calcular $K$ de tal manera que $(A-BK)$ sea una matriz estable. Por ejemplo, con el Teorema \ref{thm: ack}.
	\item Si $(A,B)$ no es controlable, pero estabilizable. Entonces calcula $K$ a través del test de Lyapunov para la estabilización como en (\ref{eq: K}).
	\item Felicidades! Escoge\footnote{Recuerda que nos referimos muchas veces a $u(t)$ como la entrada al sistema lti, para el sistema lineal no olvides que es $u(t) = u^*(t) + \delta u(t)$, i.e., para el sistema no lineal tendríamos que $u(t) = u^* -K\delta x$.}  $u(t) = -Kx(t)$ o $u(t) = -K\hat x(t)$ si has necesitado de un estimador.
	\item Por último recuerda que el controlador ha sido diseñado para un sistema no lineal. Por lo que existe una región dada en (\ref{eq: Bregion}) que garantiza la convergencia. Fuera de ahí, no podemos decir nada con seguridad.
\end{enumerate}

\section{Seguimiento por parte de la salida de una consigna constante}

Considera el siguiente sistema lti con dos señales de salida
\begin{equation}
	\Sigma := \begin{cases}
	\dot x(t) &= Ax(t) + Bu(t), \quad x\in\mathbb{R}^n, u\in\mathbb{R}^k \\
	y_1(t) &= C_1x(t), \quad y_1\in\mathbb{R}^m \\
	y_2(t) &= C_2x(t), \quad y_2\in\mathbb{R}^l
	\end{cases}.
	\label{eq: linsys2}
\end{equation}
Hasta ahora nos hemos centrado en hacer el origen del vector de estados en (\ref{eq: linsys2}) estable, y por lo tanto las señales de salida $y_{1,2}(t) = C_{1,2}x(t)$ convergen a cero si $x(t) \to 0$ según $t\to\infty$.

Por mantener esta sección corta, vamos a considerar que $C_1 = I$, es decir, que podemos medir todos los estados $x$ a partir de la salida $y_1$. Vamos a utilizar esta señal para después diseñar un controlador de realimentación de estados. Si $C_1$ fuera una matriz arbitraria, pero el par $(A,C_1)$ fuera observable, la técnica explicada en esta sección seguiría siendo aplicable con el uso de un estimador.

En esta sección vamos a centrarnos en que dada una consigna constante $y_d\in\mathbb{R}^l$, entonces que el objetivo sea $y_2(t) \to y_d$ según $t\to\infty$.

\begin{remark}
Observa que para que $y_2(t)$ pueda alcanzar un valor arbitrario $y_d$, entonces ha de existir un estado $x_d\in\mathbb{R}^n$ tal que $y_d = C_2x_d$. Es decir, necesitamos la condición de que $C_2$ sea de rango máximo para poder imponer sin problemas un $y_d$ arbitrario.
\end{remark}

Considera el cambio de variable $\tilde x(t) = x(t) - x_d$, entonces el sistema (\ref{eq: linsys2}) se puede reescribir como
\begin{equation}
	\Sigma := \begin{cases}
		\dot{\tilde x}(t) &= A\tilde x(t) + Bu(t) + Ax_d \\
	\tilde y_1(t) &= \tilde x(t)  \\
	\tilde y_2(t) &= C_2\tilde x(t)
	\end{cases},
	\label{eq: linsys3}
\end{equation}
y si existe una solución $u^*$ tal que $Bu^* = -Ax_d$, entonces con una entrada $u = \tilde u + u^*$ tenemos que
\begin{equation}
	\Sigma := \begin{cases}
		\dot{\tilde x}(t) &= A\tilde x(t) + B\tilde u(t)  \\
	\tilde y_1(t) &= \tilde x(t)  \\
	\tilde y_2(t) &= C_2\tilde x(t) 
	\end{cases},
	\label{eq: linsys4}
\end{equation}
para el cual podemos encontrar una matriz $K\in\mathbb{R}^{n\times n}$ tal que $u(t) = K (x(t) - x_d) + u^*$ haga el origen de (\ref{eq: linsys4}) asintóticamente estable. Observa que si $\tilde y_2 \to 0$ según $t\to\infty$, entonces $y_2(t) \to y_d$ también. También observa que para el diseño de $K$, las matrices $A$ y $B$ de (\ref{eq: linsys4}) son las mismas que en (\ref{eq: linsys2}). Vamos a resumir este resultado en el siguiente teorema
\begin{theorem}
	Dado el sistema lti (\ref{eq: linsys2}) con $C_1 = I$, entonces $y_2(t) \to y_d\in\mathbb{R}^l$ según $t\to\infty$ con la siguiente ley de control
	$$
	u(t) = K (x(t) - x_d) + u^*,
	$$
	donde $x_d$ y $u^*$ han de existir como soluciones a 
$$
	\begin{cases}
		y_d &= C_2x_d \\
		Bu^* &= -Ax_d
	\end{cases}.
	$$\label{thm: ff}
\end{theorem}


\subsubsection{Control integral}
El Teorema (\ref{thm: ff}) se basa fundamentalmente en que hay que conocer con exactitud las matrices $A$ y $B$ para poder calcular una señal en \emph{lazo abierto} $u^*$. Cualquier error de modelado puede hacernos calcular la $u^*$ equivocada y consecuentemente $y_2(t)$ puede no converger al valor deseado $y_d$.

Vamos a utilizar la técnica conocida como \emph{control integral} para garantizar que $y_2(t) \to y_d$ aún bajo errores de modelado. Definamos la siguiente señal (integral)
\begin{equation}
	e_i(t) := \int_0^t (y_2(t) - y_d) \mathrm{dt},
	\label{eq: ei}
\end{equation}
cuya dinámica viene dada por
\begin{equation}
	\dot e_i(t) = y_2(t) - y_d + e_i(0), \quad e_i(0)\in\mathbb{R}^l,
	\label{eq: dei}
\end{equation}
donde $e_i(0)$ por conveniencia tenemos la libertad de igualarla a cero. Observa que cuando $y_2 = y_d$, entonces (\ref{eq: ei}) es cero, i.e., la señal $e_i(t)$ está en equilibrio. Vamos a apilar las señales $x$ y $e_i$ y analizar su dinámica.
\begin{equation}
	\begin{bmatrix}\dot x \\ \dot e_i\end{bmatrix} = \begin{bmatrix}A & 0 \\ C_2 & 0\end{bmatrix}\begin{bmatrix}x \\ e_i\end{bmatrix} + \begin{bmatrix}B \\ 0 \end{bmatrix} u + \begin{bmatrix}0 \\ -I\end{bmatrix} y_d.
	\label{eq: xei}
\end{equation}
Ahora vamos a proponer la siguiente ley de control $u = -\begin{bmatrix}K & K_I\end{bmatrix}\begin{bmatrix}x \\ e_i \end{bmatrix} = -Kx -K_Ie_i$, entonces tenemos que
	\begin{equation}
	\begin{bmatrix}\dot x \\ e_i \end{bmatrix} = \begin{bmatrix}A-BK & -K_I \\ C_2 & 0\end{bmatrix}\begin{bmatrix} x \\ e_i \end{bmatrix} + \begin{bmatrix}0 \\ -I\end{bmatrix} y_d.
	\label{eq: xeta2}
	\end{equation}

Si los autovalores de la matriz $\begin{bmatrix}A-BK & -K_I \\ C_2 & 0\end{bmatrix}$ son estables, entonces el sistema (\ref{eq: xeta2}) es exponencialmente estable pero el equilibrio no estará en el origen ya que está forzado por el término $\begin{bmatrix}0 \\ -I\end{bmatrix} y_d$. Independientemente del nuevo equilibrio\footnote{Si $\begin{bmatrix}A-BK & -K_I \\ C_2 & 0\end{bmatrix}$ es una matriz de estabilidad, y $C_2$ es de rango máximo, el conjunto de equilibrio de (\ref{eq: xeta2}) puede demostrarse como $\mathcal{E} := \{x, e \, : \, C_2x = y_d, \, x = (A-BK)^{-1}K_Ie, \, y_d\in\mathbb{R}^l\}$, es decir $[x(t),e(t)] \to \mathcal{E}$ según $t\to\infty$.
	} \emph{forzado}, lo que si es cierto es que en el equilibrio tenemos que $\dot e_i = 0$, por lo que $y_2 = y_d$. La salida $y_2$ ha alcanzado el valor deseado y es tolerante a errores de modelado en $A$ y $B$ ya que la dinámica de $e_i(t)$ no depende de ellos. No obstante, grandes errores de modelado, por ejemplo para $\tilde A$ y $\tilde B$ podría hacer la matriz $\begin{bmatrix}\tilde A-\tilde BK & -K_I \\ C_2 & 0\end{bmatrix}$ con autovalores con parte real positiva.

¿Podemos encontrar $K = \begin{bmatrix}K & K_I\end{bmatrix}$ tal que $\begin{bmatrix}A-BK & -K_I \\ C_2 & 0\end{bmatrix}$ pueda tener los autovalores donde nosotros queramos? Para ello entonces hay que responder a la siguiente pregunta:

¿Es el par $\left(\begin{bmatrix}A & 0 \\ C_2 & 0\end{bmatrix}, \begin{bmatrix}B \\ 0 \end{bmatrix}\right)$ controlable (o al menos estabilizable)? Si la respuesta es afirmativa, entonces podemos encontrar una matriz $K\in\mathbb{R}^{(n+l)\times(n+l)}$ tal que $u = -K \begin{bmatrix}x \\ e_i \end{bmatrix}$ haga el sistema (\ref{eq: xeta2}) exponencialmente estable.


Observa que si $v_i$ es autovector de $A^T$ con autovalor $\lambda_i$, entonces $v_i^T A = v_i^T\lambda_i$, es decir $\begin{bmatrix}v_i^T & 0\end{bmatrix} \begin{bmatrix}A & 0 \\ C_2 & 0\end{bmatrix} = \begin{bmatrix}v_i^TA & 0\end{bmatrix} = \lambda_i\begin{bmatrix}v_i^T & 0\end{bmatrix}$. Es decir, $\begin{bmatrix}v_i^T & 0\end{bmatrix}$ es autovector de $\begin{bmatrix}A & 0 \\ C_2 & 0\end{bmatrix}$. Aplicando el Teorema \ref{thm: atbt} podemos comprobar que el par $\left(\begin{bmatrix}A & 0 \\ C_2 & 0\end{bmatrix}, \begin{bmatrix}B \\ 0 \end{bmatrix}\right)$ sería controlable si
	\begin{equation}
		\begin{bmatrix}v_i^T & 0\end{bmatrix} \begin{bmatrix}B \\ 0 \end{bmatrix} = v_i^TB \neq 0,
	\end{equation} y eso es solo posible si el par $(A,B)$ es controlable. Es decir, el control integral no ha variado las propiedades de controlabilidad del sistema original. Podemos resumir los resultados alcanzados con el siguiente teorema
\begin{theorem}
	Considera el sistema lti (\ref{eq: linsys2}) con $C_1 = I$ y $C_2$ con rango máximo. Considera la señal integral $e_i(t)$ en (\ref{eq: ei}) para un valor deseado $y_d\in\mathbb{R}^l$ para $y_2(t)$. Entonces, existe una matrix $K \in\mathbb{R}^{(n+l)\times(n+l)}$ tal que la señal de entrada $u = -K \begin{bmatrix}x \\ e_i\end{bmatrix}$ ocasiona que $y_2(t) \to y_d$ asintóticamente según $t\to\infty$ si y solo si el par $(A,B)$ en (\ref{eq: linsys2}) es controlable (estabilizable).
\end{theorem}
\end{comment}

\chapter{Realimentación para el control de sistemas lineales}\label{crllineales}

\section{Estabilidad interna o de Lyapunov}
\label{sec: sta}

Decimos que el sistema lineal (\ref{eq: linsys}) \emph{en el sentido de Lyapunov}
\begin{enumerate}
	\item es \emph{(marginalmente) estable} si para cada condición inicial $x_0$, entonces $x(t) = \Phi(t,t_0) x_0$ está acotada uniformamente para todo $t>t_0$.
	\item es \emph{asintóticamente estable} si además $x(t) \to 0$ según $t\to\infty$.
	\item es \emph{exponencialmente estable} si además $||x(t)|| \leq c e^{\lambda(t-t_0)}||x(t_0)||$ para algunas constantes $c,\lambda > 0$.
	\item es \emph{inestable} si no es marginalmente estable.
\end{enumerate}

%In control, it is very common to focus on \emph{error signals}, e.g., $e(t) := x(t) - x^*(t)$, where $x^*(t)$ is a trajectory goal. Note that if $x^*$ is constant, then $\dot e(t) = \dot x(t)$, and this is why we focus on having $x(t) \to 0$ as $t\to\infty$ in the above definitions for (\ref{eq: sigmalin}).

Centrémonos en sistemas \emph{lti}, es decir, cuando $A$ tiene coeficientes constantes o $\Phi(t,t_0) = e^{A(t-t_0)}$. Entonces, podemos establecer una clara relación entre los autovalores de $A$ y las definiciones de estabilidad en el sentido de Lyapunov únicamente inspeccionando la solución a $\dot x(t) = Ax(t)$ dada por (\ref{eq: solx}).

El sistema $\dot x(t) = Ax(t)$
\begin{enumerate}
	\item es marginalmente estable si y solo si todos los autovalores de $A$ tienen parte real no positiva. Si algún autovalor tiene parte real nula, entonces su bloque de Jordan ha de ser $1\times 1$.
	\item es asintóticamente estable si y solo si todos los autovales de $A$ tienen estrictamente parte real negativa.
	\item es exponencialmente estable si es asintóticamente estable.
	\item es inestable si y solo si al menos un autovalor de $A$ tiene parte real positiva, o al menos uno de los autovalores con parte real nula tiene un bloque de Jordan mayor de $1\times 1$.
\end{enumerate}

Comprobando las soluciones (\ref{eq: solx})-(\ref{eq: soly}), podemos decir que si $A$ tiene coeficientes constantes y $\dot x = Ax$ es asintóticamente estable, entonces $x(t) \to \int_{t_0}^t e^{A(t-\tau)}B(\tau)u(\tau)d\tau$ según $t\to\infty$. 

\subsection{Estabilidad local de sistemas linearizados}\label{lyapulin}
Si $x(t)$ es asintóticamente (exponencialmente) estable en $\dot x(t) = Ax(t)$, entonces, existe una única $P$ que satisface la \emph{ecuación de Lyapunov}
\begin{equation}
A^TP + PA = -Q, \quad \forall Q \succ 0.
	\label{eq: lya}
\end{equation}
Uno puede probar (\ref{eq: lya}) si considera
\begin{equation}
	P:= \int_0^\infty e^{A^Tt}Qe^{At}dt.
	\label{eq: P}
\end{equation}
Pista: Primero, sustituye $P$ en (\ref{eq: lya}), y después verifica el cálculo  $\frac{\mathrm{d}}{\mathrm{dt}}\left(e^{A^Tt}Qe^{At}\right)$. Si uno prueba que $P$ es única, entonces $P$ ha de ser positiva definida acorde a su definición (\ref{eq: P}).

Ahora vamos a considerar un sistema continuo, autónomo y no lineal en general
\begin{equation}
	\dot x(t) = f(x(t)), \quad x\in\mathbb{R}^n,
	\label{eq: non}
\end{equation}
con un punto de equilibrio $x^*\in\mathbb{R}^n$, i.e, $f(x^*) = 0$. La dinámica de $x(t)$ puede ser aproximada considerando $x(t) = x^* + \delta x(t)$ donde 
\begin{equation}
	\dot{\delta} x(t) = A\,\delta x(t), \quad A:=\frac{\partial f(x)}{\partial x}.
	\label{eq: delta}
\end{equation}

¿Cómo de buena es esta aproximación?

\begin{theorem}
	\label{thm: tayl}
	Asume que $f(x)$ is dos veces diferenciable. Si (\ref{eq: delta}) es exponencialmente estable, entonces, existe un entorno $\mathcal{B}$ alrededor de $x^*$ y constantes $c, \lambda > 0$ tal que para cada solución $x(t)$ del sistema (\ref{eq: non}) que empiece con $x(t_0)\in\mathcal{B}$, tenemos que
	\begin{equation}
	||x(t) - x^*|| \leq ce^{\lambda(t-t_0)} ||x(t_0) - x^*||, \quad \forall t\geq t_0.
	\end{equation}
\end{theorem}

\subsubsection{¿Cómo de grande es $\mathcal{B}$? ¿Podemos estimarlo? Esbozo de la prueba del Teorema \ref{thm: tayl}}

Como $f$ es dos veces diferenciable, de su desarrollo de Taylor tenemos que
\begin{equation}
	r(x) := f(x) - (f(x^*) + A(x - x^*)) = f(x) - A\,\delta x = O(||\delta x||^2),
\end{equation}
lo cual significa que existe una constante $c$ y una bola $\bar B$ alrededor de $x^*$ tal que 
\begin{equation}
	||r(x)|| \leq c||\delta x||^2, \quad x\in\bar B.
\end{equation}
Si el sistema linearizado es exponencialmente estable, tenemos que
\begin{equation}
A^TP + PA = -I.
\end{equation}
Ahora considera la siguiente señal escalar
\begin{equation}
	v(t) := (\delta x)^T P \delta x, \quad \forall t\geq 0.
\end{equation}
Observa que $\delta x(t) = x(t) - x^*$, entonces $\dot{\delta x(t)} = \dot x(t) = f(t)$. Por lo tanto, la derivada con respecto del tiempo de $v(t)$ satisface
\begin{align}
	\dot v &= f(x)^T P \delta x + (\delta x)^T P f(x) \nonumber \\
	&= (A\delta x + r(x))^T P \delta x + (\delta x)^T P (A\delta x + r(x)) \nonumber \\
	&= (\delta x)^T(A^T P + PA)\delta x + 2(\delta x)^T P r(x) \nonumber \\
	&= -||\delta x||^2 + 2(\delta x)^T P r(x) \nonumber \\
	&\leq -||\delta x||^2 + 2 ||P||\, ||\delta x|| \, ||r(x)||.
\end{align}

Sabemos que $v(t)$ es positiva excepto cuando $\delta x = 0$. Si podemos garantizar que $\dot v(t) < 0$ y que $\dot v(t) = 0$ solo cuando  $\delta x = 0$, entonces $v(t) \to 0$ as $t\to\infty$, lo cual implica que  $\delta x(t) \to 0$ as $t\to\infty$.

Ahora, si $x\in\mathcal{\bar B}$, entonces

\begin{equation}
	\dot v \leq -\Big(1 - 2c\,||P||\,||\delta x||\Big)||\delta x||^2,
\end{equation}
Por lo tanto, si la desviación  $\delta x$ es suficientemente pequeña, i.e., 
\begin{equation}
||\delta x|| < \frac{1}{2c||P||},
\end{equation}
entonces  $\dot v(t) < 0$ si $\delta x(0) \neq 0$ y $||\delta x(0)|| < \frac{1}{2c||P||}$.

Podemos concluir que una estimación de $\mathcal B$ es
\begin{equation}
	\mathcal{B} := \{ \delta x : ||\delta x|| < \frac{1}{2c||P||} \}.
	\label{eq: Bregion}
\end{equation}

\section{Controlabilidad}
\subsection{Subespacios alcanzables y controlables}
Recordemos que cuando aplicamos una entrada genérica $u(\cdot)$ a (\ref{eq: linsys}), transferimos el sistema de un estado $x(t_0):=x_0$ a un estado $x(t_1):=x_1$, que además podemos calcular con la expresión
\begin{equation}
	x_1 = \Phi(t_1,t_0)x_0 + \int_{t_0}^{t_1} \Phi(t_1,\tau)B(\tau)u(\tau)d\tau,
\end{equation}
donde recordemos que $\Phi(\cdot)$ es la matriz de transición de estados del sistema.

Preguntas:
\begin{enumerate}
	\item ¿Qué estados puedo alcanzar desde $x_0$?
	\item ¿Existe siempre una entrada $u(\cdot)$ que transfiera un estado arbitrario $x_0$ a otro $x_1$?
\end{enumerate}

Estas dos preguntas llevan a acuñar las definiciones de (sub)espacios alcanzables y controlables.

\begin{definition}[Subespacio alcanzable]
	Dados dos instantes de tiempo $t_1>t_0\geq 0$, el subespacio alcanzable (o controlable desde el origen) $\mathcal{R}[t_0,t_1]$ consiste en todos los estados $x_1$ para los que existe una entrada $u:[t_0,t_1]\to \mathbb{R}^k$ que transfiere el estado $x_0 = 0$ a  $x_1 \in\mathbb{R}^n$; i.e.,
	\begin{equation}
		\mathcal{R}[t_0,t_1] := \Big\{x_1\in\mathbb{R}^n : \exists u(\cdot),\, x_1 = \int_{t_0}^{t_1} \Phi(t_1,\tau)B(\tau)u(\tau)d\tau \Big\}. \label{eq: rs}
	\end{equation}
\end{definition}

\begin{definition}[Subespacio controlable]
	Dados dos instantes de tiempo $t_1>t_0\geq 0$, el subespacio controlable (or controlable hacia el origen) $\mathcal{C}[t_0,t_1]$ consiste en todos los estados $x_0$ por los que existe una entrada $u:[t_0,t_1]\to \mathbb{R}^k$ que transfiera un estado $x_0\in\mathbb{R}^n$ a $x_1 = 0$; i.e.,
	\begin{equation}
		\mathcal{C}[t_0,t_1] := \Big\{x_0\in\mathbb{R}^n : \exists u(\cdot),\, 0 = \Phi(t_1,t_0)x_0 + \int_{t_0}^{t_1} \Phi(t_1,\tau)B(\tau)u(\tau)d\tau \Big\}.
	\end{equation}
\end{definition}

¿Cómo podemos calcular $\mathcal{R}[t_0,t_1]$ y $\mathcal{C}[t_0,t_1]$? Para ello vamos a introducir y explotar las siguientes dos matrices llamadas \emph{Gramianos}.

\begin{definition}[Gramianos de alcanzabilidad y controlabilidad]
	\begin{align}
		W_R(t_0,t_1) &:= \int_{t_0}^{t_1} \Phi(t_1,\tau)B(\tau)B(\tau)^T\Phi(t_1,\tau)^Td\tau \\
		W_C(t_0,t_1) &:= \int_{t_0}^{t_1} \Phi(t_0,\tau)B(\tau)B(\tau)^T\Phi(t_0,\tau)^Td\tau 
	\end{align}
\end{definition}

\begin{theorem}
Dados dos instantes de tiempo $t_1 > t_0 \geq 0$,
	\begin{align}
		\mathcal{R}[t_0,t_1] &= \text{Im}\{W_R(t_0,t_1)\} \label{RI} \\
		\mathcal{C}[t_0,t_1] &= \text{Im}\{W_C(t_0,t_1)\} \label{CI},
	\end{align}
	donde $\text{Im}\{A\}:= \Big\{y\in\mathbb{R}^m: \exists x\in\mathbb{R}^n, y = Ax\Big\}$ para una matriz $A\in\mathbb{R}^{m\times n}$.
\end{theorem}
\begin{proof}
	Solo vamos a probar (\ref{RI}) porque (\ref{CI}) tiene una prueba similar.
	Necesitamos mostrar ambas implicaciones: primero, si $x_1 \in \text{Im}\{W_R(t_0,t_1)$, entonces $x_1 \in \mathcal{R}[t_0,t_1]$; segundo, si $x_1 \in \mathcal{R}[t_0,t_1]$ entonces $x_1 \in \text{Im}\{W_R(t_0,t_1)$.\\
	Cuando  $x_1 \in \text{Im}\{W_R(t_0,t_1)$, existe un vector $\mu_1\in\mathbb{R}^n$ tal que
	\begin{equation}
	x_1 = W_R(t_0,t_1)\eta_1.
	\end{equation}
	Escoge $u(\tau) = B(\tau)^T\Phi(t_1, \tau)^T\eta_1$, y sustitúyelo en (\ref{eq: rs}), entonces tenemos que 
	\begin{equation}
	x_1 = \int_{t_0}^{t_1} \Phi(t_1,\tau)B(\tau) B(\tau)^T\Phi(t_1, \tau)^T \eta_1d\tau = W_R(t_0,t_1)\eta_1.
	\end{equation}
	Cuando $x_1 \in \mathcal{R}[t_0,t_1]$, existe una entrada $u(\cdot)$ para la cual 
	\begin{equation}
		x_1 = \int_{t_0}^{t_1} \Phi(t_1,\tau)B(\tau)u(\tau)d\tau.
		\label{x1}
	\end{equation}
	Si (\ref{x1}) es en $\text{Im}\{W_R(t_0,t_1)\}$, entonces $x_1^T\mu = 0, \, \mu \in \text{Ker}\{W_R(t_0,t_1)\}$\footnote{If $x\in\text{Ker}\{A^T\}$, por lo que $A^Tx = 0$. Si $y\in\text{Im}\{A\}$, entonces  $y = A\eta$. Por lo tanto, $x^Ty = x^TA\eta = \eta^TA^Tx = \eta \cdot 0 = 0$. Observa que $W_R^T = W_R$ por definción.} Vamos a calcular 
	\begin{equation}
		x_1^T\mu = \int_{t_0}^{t_1}u(\tau)^TB(\tau)^T\Phi(t_1,\tau)^T\mu \, d\tau. \label{eq: x1eta1}
	\end{equation}
	Y observando que 
	\begin{align}
	\mu \in \text{Ker}\{W_R(t_0,t_1)\} \implies \mu^TW_R(t_0,t_1)\mu &= 0 \nonumber \\ &= \int_{t_0}^{t_1}\mu^T \Phi(t_1,\tau)B(\tau)B(\tau)^T\Phi(t_1,\tau)^T \mu \, d\tau \nonumber \\ &= \int_{t_0}^{t_1} ||B(\tau)^T\Phi(t_1,\tau)^T \mu||^2 d \tau,
	\end{align}
	obtenemos que $B(\tau)^T\Phi(t_1,\tau)^T \mu  = 0$, llevando a (\ref{eq: x1eta1}) ser igual a cero.
\end{proof}
\begin{remark}
	Observa que hemos probado que $u(\tau) = B(\tau)^T\Phi(t_1,\tau)^T \eta_1$ puede ser como una entrada de control para transferir $x_0 = 0$ a $x_1\in\mathbb{R}^n$ en un tiempo finito $(t_1 - t_0)$. De hecho, esta es la señal de control \emph{en lazo abierto de mínima energía}.
\end{remark}
Vamos a ver este hecho en más detalle. Considera otra señal de control  $\bar u(t)$ tal que 
\begin{equation}
x_1 = \int_{t_0}^{t_1} \Phi(t_1,\tau)B(\tau)u(\tau)d\tau = \int_{t_0}^{t_1} \Phi(t_1,\tau)B(\tau)\bar u(\tau)d\tau.
\end{equation}
Para que sea cierto, debemos tener
For this to hold, we must have
\begin{equation}
	\int_{t_0}^{t_1} \Phi(t_1,\tau)B(\tau)v(\tau) \, d\tau = 0,
	\label{eq: ve}
\end{equation}
donde $v(\tau) = u(\tau) - \bar u(\tau)$.
Vamos a ver la energía asociada a $\bar u$
\begin{align}
\int_{t_0}^{t_1}||\bar u(\tau)||^2 d\tau &= \int_{t_0}^{t_1} || B(\tau)^T\Phi(t_1,\tau)^T \eta_1 + v(\tau)||^2 d\tau \nonumber \\ 
&= \eta_1^T W_R(t_0,t_1)\eta_1 + \int_{t_0}^{t_1} ||v(\tau)||^2 d \tau + 2\eta_1^T \int_{t_0}^{t_1}B(\tau)\Phi(t_1,\tau)v(\tau) d \tau,
\end{align}
donde el tercer término es cero por (\ref{eq: ve}). Por lo tanto, si $\bar u(t)$ difiere $v(t)$ de $u(t)$, gastará $\int_{t_0}^{t_1} ||v(\tau)||^2 d\tau$ más energía que $u(t)$.

\subsection{Matriz de controlabilidad para un sistema lti}

Consideremos un sistema lineal (\ref{eq: linsys}) con $A$ y $B$ con coeficientes constantes.

El teorema de Cayley-Hamilton nos permite escribir
\begin{equation}
	e^{At} = \sum_{i=0}^{n-1}\alpha_i(t)A^i, \quad \forall t\in\mathbb{R},
\end{equation}
para algunas funciones escalares apropiadas $\alpha_i(t)$. Tenemos también que si $A$ y $B$ tienen coeficientes constantes, entonces
\begin{align}
	x_1 &= \int_{0}^{t_0-t_1} e^{At}Bu(t) dt \nonumber \\
	&= \sum_{i=0}^{n-1} A^iB \Big(\int_{0}^{t_1-t_0}\alpha_i(t)u(t)dt \Big) \nonumber \\
	&= \mathcal{C} \begin{bmatrix}\int_{0}^{t_1-t_0}\alpha_0(t)u(t)dt \\ \vdots \\ \int_{0}^{t_1-t_0} \alpha_{n-1}(t)u(t)dt \end{bmatrix},
\end{align}
donde
\begin{equation}
\mathcal{C}:=\begin{bmatrix}B & AB & A^2B & \cdots & A^{n-1}B
\end{bmatrix}_{n\times (kn)},
	\label{eq: conmat}
\end{equation}
es la llamada matrix de controlabilidad del sistema lti.
\begin{remark}Observa que $\mathcal{C}[t_0,t_1]$ y $\mathcal{C}$  son diferentes objetos. El primero es un (sub)espacio, mientras que el segundo es una matriz.
\end{remark}

Observa que acabamos de probar que la imagen de $\mathcal{C}$ es la misma que la imagen de $W_R(t_0,t_1)$. La siguiente afirmación más general puede probarse también:
\begin{theorem}
	\label{thm: spcon}
Dados dos instantes de tiempo $t_0, t_1$, con $t_1 > t_0$, tenemos que
	\begin{equation}
		\mathcal{R}[t_0,t_1] = \text{Im}\{W_R(t_0,t_1)\} = \text{Im}\{\mathcal{C}\} = \text{Im}\{W_C(t_0,t_1)\} = \mathcal{C}[t_0,t_1],
	\end{equation}
\end{theorem}

Podemos extraer dos importantes consecuencias del Teorema \ref{thm: spcon} para sistemas lti. ¡Observa que $\text{Im}\{\mathcal{C}\}$ no depende de ninguna variable tiempo!

\begin{enumerate}
	\item \emph{Reversibilidad temporal}: Si uno puede alcanzar $x_1$ desde el origen, entonces uno puede alcanzar el origen desde $x_1$. Es decir, los subespacios de alcanzabilidad y controlabilidad son el mismo subespacio.
	\item \emph{Escalado de tiempo}: La alcanzabilidad y la controlabilidad no dependen del tiempo. Si uno puede transferir el estado de $x_0$ and $x_1$ en $t$ segundos, entonces también puedes hacerlo en $\bar t \neq t$ segundos.
\end{enumerate}

\begin{definition}
	Dados dos instantes de tiempo $t_1 > t_0 \geq 0$, el par $(A,B)$ del sistema (\ref{eq: linsys}) se dice \emph{alcanzable} en $[t_0, t_1]$ si $\mathcal{R}[t_0,t_1] = \mathbb{R}^n$. Es decir, si podemos alcanzar cualquier estado en tiempo finito partiendo desde el origen.
\end{definition}

\begin{definition}
	\label{def: con}
	Dados dos instantes de tiempo $t_1 > t_0$, el par del sistema (\ref{eq: linsys}) se dice \emph{controlable} en $[t_0, t_1]$ si $\mathcal{C}[t_0,t_1] = \mathbb{R}^n$. Es decir, si podemos alcanzar el origen partiendo desde cualquier estado en tiempo finito.
\end{definition}

\subsection{Tests de controlabilidad}
El siguiente teorema es el resultado de combinar la Definición \ref{def: con} con el Teorema \ref{thm: spcon}:
\begin{theorem}
	El par (constante) $(A,B)$ es controlable si y solo si el rango de $\mathcal{C}$ es $n$.
\end{theorem}

El siguiente teorema se puede comprobar numéricamente a partir de los siguientes resultados.
\begin{theorem}
	\label{thm: atbt}
	El par (constante) $(A,B)$ es controlable si y solo si no existe ningún autovector de $A^T$ en el kernel de $B^T$.
\end{theorem}
El siguiente teorema es una reescritura del anterior.
\begin{theorem}
El par (constante) $(A,B)$ es controlable si y solo si el rango de $\begin{bmatrix}A-\lambda I & B\end{bmatrix}$ es $n$.
\end{theorem}

\begin{remark}
	Observa que el tener un sistema asintóticamente estable no implica el tener 
	un sistema lti controlable. Por ejemplo, 
\begin{equation}
	\dot x = \begin{bmatrix}-3 & 0 \\ 0 & -7\end{bmatrix}x + \begin{bmatrix}0 \\ 1\end{bmatrix}u, \quad x\in\mathbb{R}^2, u\in\mathbb{R}.
\end{equation}
	El kernel de $B^T$ es generado por $\begin{bmatrix}1 \\ 0\end{bmatrix}$, el cual es proporcional a un autovector de $A = A^T$. Entonces, el sistema no es controlable acorde al teorema \ref{thm: atbt}. Observa, que no tenemos autoridad ninguna sobre $x_1$ a través de $u$. Sin embargo, es asintóticamente estable. Uno puede ver que $x_{\{1,2\}}(t) \to 0$ según $t\to\infty$ cuando $u = 0$. Recuerda que la controlabilidad trata de transferir estados en \textbf{tiempo finito}.
\end{remark}

\section{Estabilización de un sistema lti por realimentación de estados}
\label{sec: reak}
\subsection{Test de Lyapunov para la estabilización de un sistema lti}
\begin{definition}
	Si el sistema (\ref{eq: linsys}) es lti, entonces es estabilizable si existe una entrada $u(t)$ para cualquier $x(0)$ tal que $x(t)\to 0$ as $t\to\infty$.
\end{definition}
Esta definición es una versión de \emph{sistema controlable} pero para tiempo infinito. En el siguiente teorema veremos que \emph{estabilizable} es menos restrictivo que \emph{controlable}.

\begin{theorem}
	Si el sistema (\ref{eq: linsys}) es lti, entonces es estabilizable si y solo si no existe ningún autovector de $A^T$ cuyo autovalor tenga parte real no negativa en el kernel de $B^T$.
\end{theorem}

Dicho de otra manera, la proyección de $x(t)$ sobre el espacio generado por los autovectores de $A^T$ asociados a autovalores con parte real negativa van a cero sin necesidad de la \emph{asistencia} de ninguna entrada. Entonces, la entrada $u(t)$ debe asistir a las proyecciones de $x(t)$ en el resto de autovectores de $A^T$. Por ejemplo,

\begin{equation}
	\dot x = \begin{bmatrix}-3 & 0 \\ 0 & 7\end{bmatrix}x + \begin{bmatrix}0 \\ 1\end{bmatrix}u, \quad x\in\mathbb{R}^2, u\in\mathbb{R},
\end{equation}
mientras que no tenemos ninguna \emph{autoridad} sobre $x_1(t)$, podemos emplear $u(t)$ para lleva a $x_2(t)$ a cero de tal manera que $x(t)\to 0$ según $t\to\infty$.
\begin{theorem}
	Si el sistema (\ref{eq: linsys}) es lti, entonces es estabilizable si y solo si hay una matriz positiva definida $P$ a la siguiente desigualdad
	\begin{equation}
	AP + PA^T - BB^T \prec 0
		\label{eq: lb}
	\end{equation}
\end{theorem}
\begin{proof}
	Solo vamos a ver una dirección de la prueba. No confundir (\ref{eq: lb}) con la ecuación de Lyapunov\footnote{Observa el orden de las matrices traspuestas, y observa también el signo opuesto de $-BB^T$ y $+I$ en las dos ecuaciones.} $PA+A^TP \prec -I$ en (\ref{eq: lya}).

Conisdera $x$ como un autovector asociado al autovalor $\lambda$ de $A^T$ con parte real no negativa. Entonces,
	\begin{equation}
	x^*(AP+PA^T)x < x^*BB^Tx = ||B^Tx||^2,
		\label{eq: aux}
	\end{equation}
	donde $x^*$ es el complejo conjugado de $x$. Pero el lado izquierdo de (\ref{eq: aux}) es igual a 
	\begin{equation}
		(A^T(x^*)^T)^TPx + x^*PA^Tx = \lambda^*x^*Px + \lambda x^*Px = 2\text{Real}\{\lambda\}x^*Px.
	\end{equation}
	Como $P$ es positiva definida, y $\text{Real}\{\lambda\} \geq 0$, podemos concluir que 
\begin{equation}
0 \leq 2\text{Real}\{\lambda\}x^*Px < ||B^Tx||^2,
\end{equation}
y por tanto $x$ debe pertenecer al kernel de $B$, si no tendríamos que 
\begin{equation}
0 \leq 2\text{Real}\{\lambda\}x^*Px < 0,
\end{equation}
y eso no es posible.
\end{proof}

\subsection{Controlador por realimentación de estados}\label{sec: realk}
Realimentación de estados implica el escoger una señal de entrada que dependa únicamente de los estados del sistema, es decir, diseñar una
\begin{equation}
u(t) = -Kx(t),
	\label{eq: kx}
\end{equation}
donde $K\in\mathbb{R}^{n\times k}$, tal que $x(t) \to 0$ en (\ref{eq: linsys}) exponencialmente rápido según $t\to\infty$.

Define la ganancia matriz de control
\begin{equation}
	K:=\frac{1}{2}B^TP^{-1}, \label{eq: K}
\end{equation}
donde $P$ se calcula en (\ref{eq: lb}) si y solo si un sistema lti es estabilizable. Por lo tanto, podemos rescribir (\ref{eq: lb}) como
\begin{equation}
	(A - \frac{1}{2}BB^TP^{-1})P + P(A - \frac{1}{2}BB^TP^{-1})^T = (A-BK)P+P(A-BK)^T \prec 0,
\end{equation}
y multiplicando a la izquierda y derecha por $Q:=P^{-1}$ tenemos que
\begin{equation}
	Q(A-BK)+(A-BK)^TQ \prec 0,
\end{equation}
la cual es una ecuación de Lyapunov. Por lo tanto, podemos concluir que $(A-BK)$ tiene todos sus autovalores \emph{estables}, esto es, si uno escoge la entrada
\begin{equation}
	u = -Kx = -\frac{1}{2}B^TP^{-1}x, \label{eq: conK}
\end{equation}
en el sistema lti estabilizable, entonces $x(t) \to 0$ exponencialmente rápido según $t\to\infty$.

Encontrar la matrix $P$ en (\ref{eq: K}) requiere resolver la \emph{linear matrix inequality} (LMI) en (\ref{eq: lb}). Para ello existen herramientas numéricas \emph{LMI solvers} disponibles en Matlab o Python.

Si el sistema lti es controlable (recordemos que es más restrictivo que estabilizable), la matriz $K$ en (\ref{eq: kx}) puede explotar el Teorema \ref{thm: atbt}. Es decir, podemos encontrar una matriz $K$ tal que $(A-BK)$ tenga sus autovalores donde nosotros queramos por diseño. Por ejemplo, a través del Teorema de Ackermann.
\begin{theorem}
	\label{thm: ack}
	Si el sistema lti es controlable, entonces $(A-BK)$ tiene como autovalores un conjunto deseado $\Lambda$ si
	\begin{equation}
		K = \begin{bmatrix}0 & \dots & 0 & 0 & 1\end{bmatrix}\mathcal{C}^{-1} \Delta(A),
	\end{equation}
	en donde $\mathcal{C}$ es la matrix de controlabilidad (\ref{eq: conmat}) y $\Delta$ es el polinomio característico que satisface $\Lambda$.
\end{theorem}
Obviamente, al requerir la inversa de la matriz de contolabilidad, se requiere por tanto que $\mathcal{C}$ sea de rango máximo. Es decir, el sistema lti ha de ser controlable para poder aplicar el Teorema \ref{thm: ack}.


\section{Observabilidad en sistemas lti}
\subsection{Subespacio inoservable y el Gramiano de observabilidad}
\begin{definition}
	El subespacio inoservable $\mathcal{UO}$ de un sistema lti consiste en todos aquellos estados que satisfacen
	\begin{equation}
	C e^{A} x_0 = 0.
		\label{eq: ce}
	\end{equation}
\end{definition}
Esta definición está motivada por los siguientes hechos.
Recuerda que en (\ref{eq: soly}) podemos derivar que
\begin{align}
	y(t) &= Ce^{At}x_0 + \int_0^tCe^{A(t-\tau)}Bu(\tau)d\tau + Du(t) \nonumber \\
	\tilde y(t) &:= y(t) - \int_0^tCe^{A(t-\tau)}Bu(\tau)d\tau - Du(t) = Ce^{At}x_0. \label{eq: y}
\end{align}
En el lado izquierdo de (\ref{eq: y}) tenemos un par entrada/salida, y en el lado derecho de (\ref{eq: y}) tenemos el estado inicial $x_0$. De (\ref{eq: y}) podemos observar dos propiedades interesantes:
\begin{enumerate}
	\item Cuando un particular $x_0$ es compatible con un par entrada/salida, entonces todo estado inicial de la forma $x_0 + x_u, \, x_u\in\mathcal{UO}$ también es compatible con la misma entrada/salida.
	\item Cuando $\mathcal{UO}$ contiene únicamente el vector cero, entonces existe al menos un estado inicial $x_0$ compatible con un par entrada/salida.
\end{enumerate}

\begin{definition}
	Un sistema lti es observable si su $\mathcal{UO}$ contiene solo el vector cero.
\end{definition}

\begin{definition}
	Dados dos instantes de tiempo $t_1>t_0\geq 0$, el Gramiano de observabilidad viene definido por
	\begin{equation}
		W_O(t_0,t_1) := \int_{t_0}^{t_1}e^{A^T(\tau - t_0)} C^T Ce^{A(\tau - t_0)}d\tau
	\end{equation}
\end{definition}
Partiendo de (\ref{eq: ce}), se puede llegar a
\begin{equation}
	\operatorname{Ker}W_O(t_0,t_1) = \mathcal{UO}.
\end{equation}

\subsection{Tests de observabilidad}
Lo siguiente se puede demostrar formalmente apoyándonos en el teorema de Cayley-Hamilton. Por ejemplo, para ver por qué paramos en $(n-1)$. Vamos a ver la sensibilidad con respecto de los estados $x$ de $y$ y de sus derivadas con respecto del tiempo \footnote{Observa como $B$ y $D$ no juegan ningún papel aquí} cuando $u(t) = 0$.
\begin{align}
	y(t) = Cx(t) &\implies \dot y(t) = C\dot x(t) = CAx(t) \implies \ddot y(t) = C A^2 x(t) \quad \dots \nonumber \\ &\implies \frac{\mathrm{d}^{n-1}y}{\mathrm{dt}^{n-1}} = CA^{n-1}x(t), \nonumber
\end{align}
Si no queremos perder ninguna información sobre la señal $x(t)$, entonces la matriz
\begin{equation}
	\mathcal{O} = \begin{bmatrix}C \\ CA \\ \vdots \\ CA^{n-1}\end{bmatrix}_{(kn)\times n}, \label{eq: O}
\end{equation}
debe de ser de rango máximo en sus columnas. Vamos a introducir los siguientes resultados equivalentes
\begin{theorem}
	Un sistema lti es observable si y solo si el rango de $\mathcal{O}$ es igual a $n$.
\end{theorem}
\begin{theorem}
	Un sistema lti es observable si y solo si no hay ningún autovector de $A$ en el kernel de $C$.
\end{theorem}

Fíjate que de (\ref{eq: O}) podemos derivar un test de controlabilidad
\begin{equation}
	\mathcal{O}^T = \begin{bmatrix}C^T & A^TC^T & \cdots & (A^{n-1})^TC^T \end{bmatrix}_{n \times (kn)}, 
\end{equation}
que vendría dado por el siguiente sistema lti \emph{dual}
\begin{equation}
	\Sigma_{\text{dual}} := \begin{cases}
		\dot{\bar x}(t) &= A^T \bar x(t) + C^T \bar u(t) \\
		\bar y(t) &= B^T\bar x(t) + D^T\bar u(t)
	\end{cases}.
\label{eq: sigmadual}
\end{equation}
Por lo tanto, podemos formular el siguiente resultado
\begin{theorem}
	Un sistema lti es observable si y solo si su sistema dual (\ref{eq: sigmadual}) es controlable.
\end{theorem}
No haría falta ningún test nuevo para concluir la observabilidad de un sistema lti. Simplemente, construimos su sistema dual y estudiamos su controlabilidad.

\section{Estimación de estados en sistemas lti}
El estimador más simple consiste en hacer una copia de la dinámica del sistema lti
\begin{equation}
	\Sigma_{\text{estimator}} := \begin{cases}
		\dot{\hat x}(t) &= A \hat x(t) + B u(t) \\
		\hat y(t) &= C\hat x(t) + D u(t)
	\end{cases},
\label{eq: sigmaest}
\end{equation}
en donde $\hat x\in\mathbb{R}^nm y\in\mathbb{R}^m$ será nuestra estimación de los estados y de la salida del sistema lti respectivamente. Ahora, definamos la señal de error
\begin{equation}
e(t) := \hat x(t) - x(t),
\end{equation}
por lo tanto, cuando el error $e$ sea cero, querrá decir que estamos estimando los estados del sistema lti correctamente. Vamos a ver, la dinámica de la señal de error
\begin{equation}
	\dot e(t) = \dot{\hat x}(t) - \dot x(t) = A\hat x + Bu - Ax - Bu = Ae(t),
\end{equation}
por lo que $e(t)\to 0$ según $t\to\infty$ exponencialmente rápido si $A$ es una matriz \emph{estable} para toda entrada $u(t)$.

¿Y si $A$ no es una matriz estable? Entonces considera el siguiente estimador
\begin{equation}
	\Sigma_{\text{estimator2}} := \begin{cases}
		\dot{\hat x}(t) &= A \hat x(t) + B u(t) - L(\hat y(t) - y(t)) \\
		\hat y(t) &= C\hat x(t) + D u(t)
	\end{cases},
\label{eq: sigmaest2}
\end{equation}
en donde tenemos dos entradas $u$ y $(\hat y - y)$ a la dinámica de $\hat x$, y $L\in\mathbb{R}^{n\times m}$ es una matriz de ganancias. Observa que la señal $y$ viene del sistema lti original, y es algo que podemos medir al ser su salida. Ahora, veamos la nueva dinámica para la señal de error $e(t)$.
\begin{equation}
	\dot e = A\hat x + Bu - L(\hat y - y) - (Ax + Bu) = (A-LC)e,\label{eq: ed} 
\end{equation}
por lo tanto, si $(A-LC)$ es una matriz de estabilidad, entonces $e(t)\to 0$ según $t\to\infty$ exponencialmente rápido para cualquier señal $u(t)$.

Para el cálculo de $L$ bastaría con calcular $K$ para el sistema dual utilizando los resultados de la sección \ref{sec: reak}. En particular $L = K^T_{\text{dual}}$, y $K = L^T_{\text{dual}}$.

A continuación, los resultados \emph{duales} para observabilidad.
\begin{theorem}
	Un sistema lti es \emph{detectable} si y solo si los autovectores de $A$ correspondientes a autovalores inestables no están en el kernel de $C$.
\end{theorem}
Si un sistema es detectable, entonces su dual es estabilizable, y viceversa.
\begin{theorem}
	Cuando un par $(A,C)$ es detectable, entonces es siempre posible encontrar una matriz $L$ tal que $(A-LC)$ es una matriz de estabilidad.
\end{theorem}
\begin{theorem}
	Cuando un par $(A,C)$ es observable, entonces es siempre posible encontrar una matrix $L$ tal que los autovalores de $(A-LC)$ puedan estar donde queramos.
\end{theorem}

\section[Estabilización por realimentación de la salida]{Estabilización de sistemas lti con realimentación a través de su salida}

Si $C$ no es invertible, entonces no tenemos un cálculo directo de los estados $x$ de un sistema lti; por lo tanto no podemos aplicar la señal de control $u = -Kx$ como en la sección \ref{sec: realk}.

¿Qué ocure si el sistema lti es observable, o al menos detectable? Entonces, vamos a mostrar que podemos utilizar como controlador la señal
\begin{equation}
	u = -K \hat x, \label{eq: uh}
\end{equation}
donde $\hat x$ son los estados de nuestro estimador (\ref{eq: sigmaest2}). Vamos a aplicar (\ref{eq: uh}) a un sistema lti, por lo que la dinámica será
\begin{equation}
	\dot x = Ax - BK\hat x = Ax - BK(e + x) = (A -BK)x -BKe,
\end{equation}
que junto con (\ref{eq: ed}) nos lleva al siguiente sistema autónomo
\begin{equation}
	\begin{bmatrix}\dot x \\ \dot e\end{bmatrix} = \begin{bmatrix}A-BK & -BK \\ 0 & A-LC\end{bmatrix}\begin{bmatrix}x \\ e\end{bmatrix},
\end{equation}
el cual es un sistema triangular, y será exponencialmente estable porque $(A-BK)$ y $(A-LC)$ son matrices diseñadas para tener todos sus autovalores estables.

\section{Regulador cuadrático lineal o LQR}
Si el par $(A,B)$ es controlable, entonces hemos visto que podemos colocar los autovalores de $(A-BK)$ donde queramos. Entonces uno podría preguntarse ¿cuál es el mejor lugar para los autovalores?

El \emph{mejor} lugar responde al siguiente criterio de diseño. Considera que tenemos la siguiente salida de interés
\begin{equation}
	z(t) = G x(t) + H u(t) \label{eq: z}.
\end{equation}
Por supuesto $z(t)\in\mathbb{R}^l$ puede ser la señal de salida $y(t)$, pero no tiene por qué. En particular, debemos distinguir que $y(t)$ la usaremos para alimentar el estimador/controlador, y $z(t)$ la usaremos para optimizar nuestro criterio sobre \emph{el mejor} lugar para los autovalores de $(A-BK)$.

El problema LQR está definido de la siguiente manera:
\begin{problem}
	Encontrar una entrada $u(t), t\in[0,\infty)$ que minimice la siguiente función de coste
\begin{equation}
	J := \int_0^\infty z(t)^T \bar Q z(t) + \rho \, u(t)^T \bar R u(t) dt,
	\label{eq: J}
\end{equation}
	donde $\bar Q\in\mathbb{R}^{l\times l}$ y $\bar R\in\mathbb{R}^{m\times m}$ son matrices positivas definidas, y $\rho$ es una constante positiva para relativizar ambos términos en (\ref{eq: J}).
\end{problem}

Como $z = Gx + Hu$, entonces $J$ puede ser rescrito como
\begin{equation}
	J := \int_0^\infty x(t)^T Q x(t) + \, u(t)^T R u(t) + 2x(t)^T N u(t)dt,
	\label{eq: J2},
\end{equation}
donde $Q = G^T\bar Q G$, $R = H^T\bar QH + \rho \bar R$, y $N = G^T\bar Q H$.

Observa como $\bar Q$, $\bar R$ y $\rho$ determinan como de importante es el minimizar los elementos de $z$ y $u$.

Si somos capaces de encontrar una matriz positiva definida $P$ tal que satisface la \emph{ecuación algebraica de Ricatti}
\begin{equation}
	A^TP + PA + Q - (PB + N)R^{-1}(B^TP + N^T) = 0,
\end{equation}
entonces
\begin{theorem}
	Si $A - BR^{-1}(B^TP+N^T)$ es una matriz de estabilidad. Entonces, la señal de entrada
	\begin{equation}
	u(t) = -R^{-1}(B^TP+N^T) x(t),
	\end{equation}
	minimiza $J$, es más, $J = x(0)^T P x(0)$.
\end{theorem}

La \emph{regla de Bryson} nos orienta para unos valores razonables de $\bar Q$ y $\bar R$:
\begin{align}
	\bar Q_{ii} &= \frac{1}{\text{valor máximo aceptable de}\, z_i^2} \nonumber \\
	\bar R_{jj} &= \frac{1}{\text{valor máximo aceptable de}\, u_j^2}. \nonumber
\end{align}

En el capitulo siguiente (opcional) se desarrollan más los conceptos de control óptimo introducidos aquí.


\section{Resumen para la estabilización de un punto en un sistema no lineal}
Dado el siguiente sistema no lineal
\begin{equation}
	\Sigma :=
\begin{cases}
	\dot x &= f(x,u) \\
	y &= g(x,u)
\end{cases}, \label{eq: sisnl}
\end{equation}
donde $x\in\mathbb{R}^n$ es el estado del sistema, $u\in\mathbb{R}^k$ es la entrada al sistema, $y\in\mathbb{R}^m$ es la salida del sistema, y $f(x,u): \mathbb{R}^n\times\mathbb{R}^k \to \mathbb{R}^n$ y  $g(x,u): \mathbb{R}^n\times \times\mathbb{R}^k \to \mathbb{R}^m$ son funciones arbitrarias.

El siguiente algoritmo nos permite estabilizar un punto arbitrario $x^*$ del sistema (\ref{eq: sisnl}).

\begin{enumerate}
	\item Escoge un punto de interés $x^*$, y entonces calcula $u^*$ de tal manera que  $f(x^*,u^*) = 0$.
	\item Lineariza (\ref{eq: sisnl}) alrededor de  $x^*$ y $u^*$, i.e., calcula sus Jacobianos $A:=\frac{\partial f(x,u)}{\partial x}$ y $B:=\frac{\partial f(x,u)}{\partial u}$ y evalúalos en $x=x^*$ y $u=u^*$. Observa, que si un Jacobiano no existe, e.g., $f(x,u)$ no es real analítica, entonces, no podemos continuar.
	\item Comprobar si $(A,B)$ es controlable. Si este test falla, entonces comprueba si es almenos estabilizable. Si ambos tests fallan, no podemos continuar.
	\item Calcular el Jacobiano $C:=\frac{\partial g(x,u)}{\partial x}$. Si $C$ no es invertible, entonces necesitaremos un observador/estimador. Si $C$ es invertible, entonces ves al paso \ref{step}.
	\item Si necesitamos un estimador, entonces comprueba que $(A,C)$ es observable. Si no, comprueba que $(A,C)$ es detectable. Si ambos tests fallan, no podemos continuar.
	\item Si $(A,C)$ es observable, entonces contruye el estimador (\ref{eq: sigmaest2}), con $D:=\frac{\partial g(x,u)}{\partial u}$, y calcula $L$ de tal manera que $(A-LC)$ sea una matriz estable. Por ejemplo, con el Teorema \ref{thm: ack}.
	\item Si $(A,C)$ no es observable pero sí detectable, entonces, calcula $L$ a través del sistema dual con el test de Lyapunov para la estabilización como en (\ref{eq: K}). Recuerda, que entonces $L = K^T$.
	\item \label{step} Si el par $(A,B)$ es controlable, entonces podemos calcular $K$ de tal manera que $(A-BK)$ sea una matriz estable. Por ejemplo, con el Teorema \ref{thm: ack}.
	\item Si $(A,B)$ no es controlable, pero estabilizable. Entonces calcula $K$ a través del test de Lyapunov para la estabilización como en (\ref{eq: K}).
	\item Felicidades! Escoge\footnote{Recuerda que nos referimos muchas veces a $u(t)$ como la entrada al sistema lti, para el sistema lineal no olvides que es $u(t) = u^*(t) + \delta u(t)$, i.e., para el sistema no lineal tendríamos que $u(t) = u^* -K\delta x$.}  $u(t) = -Kx(t)$ o $u(t) = -K\hat x(t)$ si has necesitado de un estimador.
	\item Por último recuerda que el controlador ha sido diseñado para un sistema no lineal. Por lo que existe una región dada en (\ref{eq: Bregion}) que garantiza la convergencia. Fuera de ahí, no podemos decir nada con seguridad.
\end{enumerate}

\section{Seguimiento por parte de la salida de una consigna constante}

Considera el siguiente sistema lti con dos señales de salida
\begin{equation}
	\Sigma := \begin{cases}
	\dot x(t) &= Ax(t) + Bu(t), \quad x\in\mathbb{R}^n, u\in\mathbb{R}^k \\
	y_1(t) &= C_1x(t), \quad y_1\in\mathbb{R}^m \\
	y_2(t) &= C_2x(t), \quad y_2\in\mathbb{R}^l
	\end{cases}.
	\label{eq: linsys2}
\end{equation}
Hasta ahora nos hemos centrado en hacer el origen del vector de estados en (\ref{eq: linsys2}) estable, y por lo tanto las señales de salida $y_{1,2}(t) = C_{1,2}x(t)$ convergen a cero si $x(t) \to 0$ según $t\to\infty$.

Por mantener esta sección corta, vamos a considerar que $C_1 = I$, es decir, que podemos medir todos los estados $x$ a partir de la salida $y_1$. Vamos a utilizar esta señal para después diseñar un controlador de realimentación de estados. Si $C_1$ fuera una matriz arbitraria, pero el par $(A,C_1)$ fuera observable, la técnica explicada en esta sección seguiría siendo aplicable con el uso de un estimador.

En esta sección vamos a centrarnos en que dada una consigna constante $y_d\in\mathbb{R}^l$, entonces que el objetivo sea $y_2(t) \to y_d$ según $t\to\infty$.

\begin{remark}
Observa que para que $y_2(t)$ pueda alcanzar un valor arbitrario $y_d$, entonces ha de existir un estado $x_d\in\mathbb{R}^n$ tal que $y_d = C_2x_d$. Es decir, necesitamos la condición de que $C_2$ sea de rango máximo para poder imponer sin problemas un $y_d$ arbitrario.
\end{remark}

Considera el cambio de variable $\tilde x(t) = x(t) - x_d$, entonces el sistema (\ref{eq: linsys2}) se puede reescribir como
\begin{equation}
	\Sigma := \begin{cases}
		\dot{\tilde x}(t) &= A\tilde x(t) + Bu(t) + Ax_d \\
	\tilde y_1(t) &= \tilde x(t)  \\
	\tilde y_2(t) &= C_2\tilde x(t)
	\end{cases},
	\label{eq: linsys3}
\end{equation}
y si existe una solución $u^*$ tal que $Bu^* = -Ax_d$, entonces con una entrada $u = \tilde u + u^*$ tenemos que
\begin{equation}
	\Sigma := \begin{cases}
		\dot{\tilde x}(t) &= A\tilde x(t) + B\tilde u(t)  \\
	\tilde y_1(t) &= \tilde x(t)  \\
	\tilde y_2(t) &= C_2\tilde x(t) 
	\end{cases},
	\label{eq: linsys4}
\end{equation}
para el cual podemos encontrar una matriz $K\in\mathbb{R}^{n\times n}$ tal que $u(t) = K (x(t) - x_d) + u^*$ haga el origen de (\ref{eq: linsys4}) asintóticamente estable. Observa que si $\tilde y_2 \to 0$ según $t\to\infty$, entonces $y_2(t) \to y_d$ también. También observa que para el diseño de $K$, las matrices $A$ y $B$ de (\ref{eq: linsys4}) son las mismas que en (\ref{eq: linsys2}). Vamos a resumir este resultado en el siguiente teorema
\begin{theorem}
	Dado el sistema lti (\ref{eq: linsys2}) con $C_1 = I$, entonces $y_2(t) \to y_d\in\mathbb{R}^l$ según $t\to\infty$ con la siguiente ley de control
	$$
	u(t) = K (x(t) - x_d) + u^*,
	$$
	donde $x_d$ y $u^*$ han de existir como soluciones a 
$$
	\begin{cases}
		y_d &= C_2x_d \\
		Bu^* &= -Ax_d
	\end{cases}.
	$$\label{thm: ff}
\end{theorem}


\subsubsection{Control integral}
El Teorema (\ref{thm: ff}) se basa fundamentalmente en que hay que conocer con exactitud las matrices $A$ y $B$ para poder calcular una señal en \emph{lazo abierto} $u^*$. Cualquier error de modelado puede hacernos calcular la $u^*$ equivocada y consecuentemente $y_2(t)$ puede no converger al valor deseado $y_d$.

Vamos a utilizar la técnica conocida como \emph{control integral} para garantizar que $y_2(t) \to y_d$ aún bajo errores de modelado. Definamos la siguiente señal (integral)
\begin{equation}
	e_i(t) := \int_0^t (y_2(t) - y_d) \mathrm{dt},
	\label{eq: ei}
\end{equation}
cuya dinámica viene dada por
\begin{equation}
	\dot e_i(t) = y_2(t) - y_d + e_i(0), \quad e_i(0)\in\mathbb{R}^l,
	\label{eq: dei}
\end{equation}
donde $e_i(0)$ por conveniencia tenemos la libertad de igualarla a cero. Observa que cuando $y_2 = y_d$, entonces (\ref{eq: ei}) es cero, i.e., la señal $e_i(t)$ está en equilibrio. Vamos a apilar las señales $x$ y $e_i$ y analizar su dinámica.
\begin{equation}
	\begin{bmatrix}\dot x \\ \dot e_i\end{bmatrix} = \begin{bmatrix}A & 0 \\ C_2 & 0\end{bmatrix}\begin{bmatrix}x \\ e_i\end{bmatrix} + \begin{bmatrix}B \\ 0 \end{bmatrix} u + \begin{bmatrix}0 \\ -I\end{bmatrix} y_d.
	\label{eq: xei}
\end{equation}
Ahora vamos a proponer la siguiente ley de control $u = -\begin{bmatrix}K & K_I\end{bmatrix}\begin{bmatrix}x \\ e_i \end{bmatrix} = -Kx -K_Ie_i$, entonces tenemos que
	\begin{equation}
	\begin{bmatrix}\dot x \\ \dot e_i \end{bmatrix} = \begin{bmatrix}A-BK & -BK_I \\ C_2 & 0\end{bmatrix}\begin{bmatrix} x \\ e_i \end{bmatrix} + \begin{bmatrix}0 \\ -I\end{bmatrix} y_d.
	\label{eq: xeta2}
	\end{equation}

Si los autovalores de la matriz $\begin{bmatrix}A-BK & -BK_I \\ C_2 & 0\end{bmatrix}$ son estables, entonces el sistema (\ref{eq: xeta2}) es exponencialmente estable pero el equilibrio no estará en el origen ya que está forzado por el término $\begin{bmatrix}0 \\ -I\end{bmatrix} y_d$. Independientemente del nuevo equilibrio\footnote{Si $\begin{bmatrix}A-BK & -BK_I \\ C_2 & 0\end{bmatrix}$ es una matriz de estabilidad, y $C_2$ es de rango máximo, el conjunto de equilibrio de (\ref{eq: xeta2}) puede demostrarse como $\mathcal{E} := \{x, e \, : \, C_2x = y_d, \, x = (A-BK)^{-1}K_Ie, \, y_d\in\mathbb{R}^l\}$, es decir $[x(t),e(t)] \to \mathcal{E}$ según $t\to\infty$.
	} \emph{forzado}, lo que si es cierto es que en el equilibrio tenemos que $\dot e_i = 0$, por lo que $y_2 = y_d$. La salida $y_2$ ha alcanzado el valor deseado y es tolerante a errores de modelado en $A$ y $B$ ya que la dinámica de $e_i(t)$ no depende de ellos. No obstante, grandes errores de modelado, por ejemplo para $\tilde A$ y $\tilde B$ podría hacer la matriz $\begin{bmatrix}\tilde A-\tilde BK & -BK_I \\ C_2 & 0\end{bmatrix}$ con autovalores con parte real positiva.

¿Podemos encontrar $K = \begin{bmatrix}K & K_I\end{bmatrix}$ tal que $\begin{bmatrix}A-BK & -BK_I \\ C_2 & 0\end{bmatrix}$ pueda tener los autovalores donde nosotros queramos? Para ello entonces hay que responder a la siguiente pregunta:

¿Es el par $\left(\begin{bmatrix}A & 0 \\ C_2 & 0\end{bmatrix}, \begin{bmatrix}B \\ 0 \end{bmatrix}\right)$ controlable (o al menos estabilizable)? Si la respuesta es afirmativa, entonces podemos encontrar una matriz $K\in\mathbb{R}^{(n+l)\times(n+l)}$ tal que $u = -K \begin{bmatrix}x \\ e_i \end{bmatrix}$ haga el sistema (\ref{eq: xeta2}) exponencialmente estable.

\begin{example}[Sistema mecánico elemental amortiguado] Consideremos el sistema dinámico,
\begin{equation*}
\begin{bmatrix}
\dot x_1 \\
\dot x_2
\end{bmatrix} =
\begin{bmatrix}
0 & 1\\
0 & -\mu
\end{bmatrix} 
\begin{bmatrix}
x_1 \\
x_2
\end{bmatrix}+ 
\begin{bmatrix}
0 \\ 1
\end{bmatrix}u
\end{equation*}
Podemos pensar en $x_1$ como la posición de un móvil que se mueve sobre una línea recta y en $x_2$ como su velocidad. Supongamos que intentamos hacerle alcanzar una posición final $x_{1d}$ y una velocidad final $x_{2d} \neq 0$. Es evidente que los dos objetivos son imposibles de conseguir simultáneamente. Supongamos que lo intentamos. Para ello ampliamos nuestro sistema de modo que que contenga la señal de error, empleando la ecuación (\ref{eq: xei})  con $C_2 = I ;\,  e_i = x_i-x_{id}$ obtenemos,

\begin{equation*}
\begin{bmatrix}
\dot x_1 \\
\dot x_2\\
\dot e_1\\
\dot e_2
\end{bmatrix} =
\underbrace{
\begin{bmatrix}
0 & 1 &0 & 0\\
0 & -\mu & 0 & 0\\
1 & 0 &0 &0\\
0 &1 &0 & 0
\end{bmatrix}}_{\bar{A}=\begin{bmatrix}
A & 0\\
C_2 & 0
\end{bmatrix}}
\begin{bmatrix}
x_1 \\
x_2\\
e_1\\
e_2
\end{bmatrix} + 
\underbrace{\begin{bmatrix}
0 \\
1\\
0\\
0
\end{bmatrix}}_{\bar{B}=\begin{bmatrix}B\\ 0
\end{bmatrix}}u+
\begin{bmatrix}
0 &0 \\
0 & 0\\
-1 & 0\\
0 &-1
\end{bmatrix}
\begin{bmatrix}
x_{1d}\\x_{2d}
\end{bmatrix}
\end{equation*}

Podemos ahora comprobar si el sistema ampliado es controlable,
\begin{equation*}
\mathcal{C} = [\bar{B},\ \bar{A} \bar{B},\ \bar{A}^2\bar{B},\ \bar{A}^3\bar{B}]  = 
\begin{bmatrix}
0 & 1 & -\mu & \mu^2\\
1& -\mu & \mu^2 & -\mu^3\\
0 & 0 & 1 & -\mu\\
0 & 1 & -\mu & \mu^2
\end{bmatrix}.
\end{equation*}
Es suficiente comparar la fila primera y la última para comprobar que la matriz de controlabilidad tiene rango menor que cuatro, con lo que el sistema no es controlable.

Se deja como ejercicio comprobar que si tratamos de controlar solo la posición final del móvil, el sistema ampliado sí es controlable; el móvil alcanza la posición deseada y se para.

\qed
\end{example}





 



\include{ejercicio}
\include{optimo}
\include{estabilidad}
\include{control_nonlinear}
\include{apendice1}


\printindex
\end{document} 
